%auto-ignore

\section{Additional Ablation Studies}
\label{appendix:sec:more_ablation_studies}

\subsection{Effect of Number of Training Steps}
\label{sec:num_training_steps}

Figure~\ref{fig:step_abalation} presents MNLI Dev accuracy after fine-tuning from a checkpoint that has been pre-trained for $k$ steps. This allows us to answer the following questions:

%auto-ignore

\begin{figure}[b]
\begin{tikzpicture}
 \begin{axis}[
   width=0.95\columnwidth,
   height=0.7\columnwidth,
   legend cell align=left,
   legend style={at={(1, 0)},anchor=south east,font=\scriptsize},
   mark options={mark size=3},
   font=\scriptsize,
   xmin=0, xmax=1000,
   ymin=75, ymax=85,
   xtick={200,400,600,800,1000},
   ymajorgrids=true,
   xmajorgrids=true,
   xlabel style={yshift=0.5ex,},
   xlabel=Pre-training Steps (Thousands) ,
   ylabel=MNLI Dev Accuracy,
   ylabel style={yshift=-0.5ex,}]
    \addplot[mark=triangle,g-blue] plot coordinates {
      (30, 78.6)
      (50, 79.6)
      (100, 80.5)
      (200, 82.2)
      (400, 83.2)
      (600, 84.0)
      (800, 84.3)
      (1000, 84.4)
    };
    \addlegendentry{\bertbase (Masked LM)}
    \addplot[mark=x,g-red] plot coordinates {
      (30, 79.4)
      (50, 79.8)
      (100, 80.3)
      (200, 81.0)
      (400, 81.7)
      (600, 81.9)
      (800, 82.1)
      (1000, 82.2)
    };
    \addlegendentry{\bertbase (Left-to-Right)}
     \end{axis}
\end{tikzpicture}
\caption{Ablation over number of training steps. This shows the MNLI accuracy after fine-tuning, starting from model parameters that have been pre-trained for $k$ steps. The x-axis is the value of $k$.}
\label{fig:step_abalation}
\end{figure}

\begin{enumerate}
\item Question: Does BERT really need such a large amount of pre-training (128,000 words/batch * 1,000,000 steps) to achieve high fine-tuning accuracy? \\
Answer: Yes, \bertbase achieves almost 1.0\% additional accuracy on MNLI when trained on 1M steps compared to 500k steps.
\item Question: Does MLM pre-training converge slower than LTR pre-training, since only 15\% of words are predicted in each batch rather than every word? \\
Answer: The MLM model does converge slightly slower than the LTR model. However, in terms of absolute accuracy the MLM model begins to outperform the LTR model almost immediately.
\end{enumerate}

\subsection{Ablation for Different Masking Procedures}
\label{appendix:sec:different_masks}

In Section~\ref{sec:pretraining_tasks}, we mention that BERT uses a mixed strategy for masking the target tokens when pre-training with the masked language
model (MLM) objective. The following is an ablation study to evaluate the effect of different masking strategies.

Note that the purpose of the masking strategies is to reduce the mismatch
between pre-training and fine-tuning, as the {\tt [MASK]} symbol never appears during the fine-tuning stage. We report the Dev results for both
MNLI and NER. For NER, we report both fine-tuning and feature-based approaches,
as we expect the mismatch will be amplified for the feature-based approach as
the model will not have the chance to adjust the representations.

%auto-ignore
\begin{table}[ht]
\begin{center}
{\small
\begin{tabular}{@{}rrrccc@{}}
  \toprule
  \multicolumn{3}{c}{Masking Rates} & \multicolumn{3}{c}{Dev Set Results} \\
  \cmidrule(r{0.2cm}){1-3}
  \cmidrule(l{0.2cm}){4-6}
  \textsc{Mask} &	\textsc{Same}&	\textsc{Rnd}	& {MNLI} &	\multicolumn{2}{c}{NER}\\
   &  & & {\footnotesize Fine-tune} &   {\footnotesize Fine-tune}	& {\footnotesize Feature-based}\\
    \cmidrule(r{0.2cm}){1-3}
  \cmidrule(l{0.1cm}r{0.1cm}){4-4}
  \cmidrule(l{0.2cm}){5-6}
  
  80\%	&10\%	&10\%	&84.2	&95.4	&94.9\\
100\%	&0\%	&0\%	&84.3	&94.9	&94.0\\
80\%	&0\%	&20\%	&84.1	&95.2	&94.6\\
80\%	&20\%	&0\%	&84.4	&95.2	&94.7\\
0\%	&20\%	&80\%	&83.7	&94.8	&94.6 \\
0\%	&0\%	&100\%	&83.6	&94.9	&94.6\\
\bottomrule
\end{tabular}
} % small
\end{center}
\caption{\label{tab:mask_ablation} Ablation over different masking strategies.}
\end{table}



The results are presented in Table~\ref{tab:mask_ablation}. In the table,
 \textsc{Mask} means that we replace the target token with the {\tt [MASK]} symbol
for MLM; \textsc{Same} means that we keep the target token as is; \textsc{Rnd}
means that we replace the target token with another random token. 

The numbers in the left part of the table represent the probabilities of the specific strategies used during MLM pre-training (BERT uses 80\%, 10\%, 10\%). The right part of the paper represents the Dev set results. For the feature-based
approach, we concatenate the last 4 layers of BERT as the features, which
was shown to be the best approach in Section~\ref{sec:ner}.

From the table it can be seen that fine-tuning is surprisingly robust to
different masking strategies. However, as expected, using only the \textsc{Mask} strategy
was problematic when applying the feature-based approach to NER. Interestingly,
using only the \textsc{Rnd} strategy performs much worse than our strategy as well.






