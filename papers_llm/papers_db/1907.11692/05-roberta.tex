\section{\ourmodel{}} \label{sec:roberta}

%auto-ignore
\section{Ablation Studies}
\label{sec:ablation}
In this section, we perform ablation experiments over a number of facets of BERT in order to better understand their relative importance. Additional ablation studies can be found in Appendix~\ref{appendix:sec:more_ablation_studies}.

\subsection{Effect of Pre-training Tasks}

\label{sec:task_ablation}
We demonstrate the importance of the deep bidirectionality of BERT by evaluating two pre-training objectives using exactly the same pre-training data, fine-tuning scheme, and hyperparameters as \bertbase:
\vspace{0.3cm}
\\
\noindent\textbf{No NSP}: A bidirectional model which is trained using the ``masked LM'' (MLM) but without the ``next sentence prediction'' (NSP) task.\\
\noindent\textbf{LTR \& No NSP}: A left-context-only model which is trained using a standard Left-to-Right (LTR) LM, rather than an MLM. The left-only constraint was also applied at fine-tuning, because  removing it introduced a pre-train/fine-tune mismatch that degraded downstream performance. Additionally, this model was pre-trained without the NSP task. This is directly comparable to OpenAI GPT, but using our larger training dataset, our input representation, and our fine-tuning scheme.
%auto-ignore
\begin{table}[t]
\small
 \begin{tabular}{@{}lccccc@{}}
    \toprule
              & \multicolumn{5}{c}{Dev Set} \\
   Tasks & MNLI-m & QNLI & MRPC & SST-2 & SQuAD     \\
         & (Acc) & (Acc) & (Acc) & (Acc) & (F1)     \\
     \midrule
\bertbase       & 84.4 & 88.4 & 86.7 & 92.7 & 88.5 \\
No NSP          & 83.9 & 84.9 & 86.5 & 92.6 & 87.9 \\
LTR \& No NSP   & 82.1 & 84.3 & 77.5 & 92.1 & 77.8 \\
\quad + BiLSTM  & 82.1 & 84.1 & 75.7 & 91.6 & 84.9 \\
     \bottomrule
   \end{tabular}
   \caption{Ablation over the pre-training tasks using the \bertbase architecture. ``No NSP'' is trained without the next sentence prediction task. ``LTR \& No NSP'' is trained as a left-to-right LM without the next sentence prediction, like OpenAI GPT. ``+ BiLSTM'' adds a randomly initialized BiLSTM on top of the ``LTR + No NSP'' model during fine-tuning.
   }
   \label{tab:task_ablation}    
\end{table}

We first examine the impact brought by the NSP task. In Table~\ref{tab:task_ablation}, we show that removing NSP hurts performance significantly on QNLI, MNLI, and SQuAD 1.1. Next, we evaluate the impact of training bidirectional representations by comparing ``No NSP'' to ``LTR \& No NSP''. The LTR model performs worse than the MLM model on all tasks, with large drops on MRPC and SQuAD.

For SQuAD it is intuitively clear that a LTR model will perform poorly at token predictions, since the token-level hidden states have no right-side context.
In order to make a good faith attempt at strengthening the LTR system, we added a randomly initialized BiLSTM on top. This does significantly improve results on SQuAD, but the results are still far worse than those of the pre-trained bidirectional models. The BiLSTM hurts performance on the GLUE tasks. 

We recognize that it would also be possible to train separate LTR and RTL models and represent each token as the concatenation of the two models, as ELMo does. However: (a) this is twice as expensive as a single bidirectional model; (b) this is non-intuitive for tasks like QA, since the RTL model would not be able to condition the answer on the question; (c) this it is strictly less powerful than a deep bidirectional model, since it can use both left and right context at every layer.


\subsection{Effect of Model Size}
\label{sec:model_size_ablation}

In this section, we explore the effect of model size on fine-tuning task accuracy. We trained a number of BERT models with a differing number of layers, hidden units, and attention heads, while otherwise using the same hyperparameters and training procedure as described previously.

Results on selected GLUE tasks are shown in Table~\ref{tab:size_ablation}. In this table, we report the average Dev Set accuracy from 5 random restarts of fine-tuning. We can see that larger models lead to a strict accuracy improvement across all four datasets, even for MRPC which only has 3,600 labeled training examples, and is substantially different from the pre-training tasks. It is also perhaps surprising that we are able to achieve such significant improvements on top of models which are already quite large relative to the existing literature. For example, the largest Transformer explored in \citet{vaswani-etal:2017:_atten} is (L=6, H=1024, A=16) with 100M parameters for the encoder, and the largest Transformer we have found in the literature is (L=64, H=512, A=2) with 235M parameters \cite{alrfou:2018}. By contrast, \bertbase contains 110M parameters and \bertlarge contains 340M parameters.

%auto-ignore
\begin{table}[b]
\begin{center}
{\small
\begin{tabular}{@{}rrrcccc@{}}
  \toprule
  \multicolumn{3}{c}{Hyperparams}      &      & \multicolumn{3}{c}{Dev Set Accuracy} \\
  \midrule
  \#L & \#H &\#A & LM (ppl) & MNLI-m & MRPC &SST-2               \\
  \midrule
  \
   3 &  768 & 12 & 5.84 & 77.9 & 79.8 & 88.4 \\
   6 &  768 &  3 & 5.24 & 80.6 & 82.2 & 90.7 \\
   6 &  768 & 12 & 4.68 & 81.9 & 84.8 & 91.3 \\
  12 &  768 & 12 & 3.99 & 84.4 & 86.7 & 92.9 \\
  12 & 1024 & 16 & 3.54 & 85.7 & 86.9 & 93.3 \\
  24 & 1024 & 16 & 3.23 & 86.6 & 87.8 & 93.7 \\
\bottomrule
\end{tabular}
} % small
\end{center}
\caption{\label{tab:size_ablation} Ablation over BERT model size. \#L = the number of layers; \#H = hidden size; \#A = number of attention heads. ``LM (ppl)'' is the masked LM perplexity of held-out training data.}
\end{table}


It has long been known that increasing the model size will lead to continual improvements on large-scale tasks such as machine translation and language modeling, which is demonstrated by the LM perplexity of held-out training data shown in Table~\ref{tab:size_ablation}. However, we believe that this is the first work to demonstrate convincingly that scaling to extreme model sizes also leads to large improvements on very small scale tasks, provided that the model has been sufficiently pre-trained. \citet{peters2018dissecting} presented mixed results on the downstream task impact of increasing the pre-trained bi-LM size from two to four layers and \citet{melamud2016context2vec} mentioned in passing that increasing hidden dimension size from 200 to 600 helped, but increasing further to 1,000 did not bring further improvements. Both of these prior works used a feature-based approach --- we hypothesize that when the model is fine-tuned directly on the downstream tasks and uses only a very small number of randomly initialized additional parameters, the task-specific models can benefit from the larger, more expressive pre-trained representations even when downstream task data is very small. 



\subsection{Feature-based Approach with BERT}
\label{sec:ner}
All of the BERT results presented so far have used the fine-tuning approach, where a simple classification layer is added to the pre-trained model, and all parameters are jointly fine-tuned on a downstream task. However, the feature-based approach, where fixed features are extracted from the pre-trained model, has certain advantages. First, not all
%NLP 
tasks can be easily represented by a Transformer encoder architecture, and therefore require a task-specific model architecture to be added. Second, there are major computational benefits to
%being able to 
pre-compute an expensive representation of the training data once and then run many experiments with 
%less expensive 
cheaper
models on top of this representation. 

In this section, we compare the two approaches by applying BERT to the CoNLL-2003 Named Entity Recognition (NER) task~\cite{tjong-de:2003}. In the input to BERT, we use a case-preserving WordPiece model, and we include the maximal document context provided by the data. Following standard practice, we formulate this as a tagging task but do not use a CRF layer in the output. We use the representation of the first sub-token as the input to the token-level classifier over the NER label set.


To ablate the fine-tuning approach, we apply the feature-based approach by extracting the activations from one or more layers {\it without} fine-tuning any parameters of BERT. These contextual embeddings are used as input to a randomly initialized two-layer 768-dimensional BiLSTM before the classification layer.

Results are presented in Table~\ref{tab:pretrained_embeddings}. \bertlarge performs competitively with state-of-the-art methods. The best performing method concatenates the token representations from the top four hidden layers of the pre-trained Transformer, which is only 0.3 F1 behind fine-tuning the entire model. This demonstrates that BERT is effective for both fine-tuning and feature-based approaches.


%auto-ignore

\begin{table}[t]
\small
\centering
 \begin{tabular}{@{}lcc@{}}
\toprule
System & Dev F1 & Test F1 \\
\midrule
ELMo~\cite{peters-etal:2018:_deep}& 95.7 & 92.2 \\
CVT~\cite{clark2018semi} & - & 92.6 \\
CSE~\cite{akbik2018contextual} & - & {\bf 93.1} \\
\midrule
Fine-tuning approach & & \\
\;\;\;\bertlarge  & 96.6 & 92.8 \\
\;\;\;\bertbase& 96.4 & 92.4 \\
\midrule
Feature-based approach (\bertbase) &  &  \\
\;\;\;Embeddings & 91.0 &- \\
\;\;\;Second-to-Last Hidden   & 95.6&- \\
\;\;\;Last Hidden            & 94.9&- \\
\;\;\;Weighted Sum Last Four Hidden        & 95.9&- \\
\;\;\;Concat Last Four Hidden        & 96.1&- \\
\;\;\;Weighted Sum All 12 Layers        & 95.5&- \\
\bottomrule
\end{tabular}
\caption{CoNLL-2003 Named Entity Recognition results. Hyperparameters were selected using the Dev set. The reported Dev and Test scores are averaged over 5 random restarts using those hyperparameters.
}
\label{tab:ner_results}    
\label{tab:pretrained_embeddings}    
\end{table}

In the previous section we propose modifications to the BERT pretraining procedure that improve end-task performance.
We now aggregate these improvements and evaluate their combined impact.
We call this configuration \textbf{\ourmodel{}} for \underline{\textbf{R}}obustly \underline{\textbf{o}}ptimized \underline{\textbf{BERT}} \underline{\textbf{a}}pproach.
Specifically, \ourmodel{} is trained with dynamic masking (Section~\ref{sec:dynamic_masking}), \textsc{full-sentences} without NSP loss (Section~\ref{sec:model_input_nsp}), large mini-batches (Section~\ref{sec:large_batches}) and a larger byte-level BPE (Section~\ref{sec:bpe}).

Additionally, we investigate two other important factors that have been under-emphasized in previous work: (1) the data used for pretraining, and (2) the number of training passes through the data.
For example, the recently proposed XLNet architecture~\cite{yang2019xlnet} is pretrained using nearly 10 times more data than the original BERT~\cite{devlin2018bert}.
It is also trained with a batch size eight times larger for half as many optimization steps, thus seeing four times as many sequences in pretraining compared to BERT.

To help disentangle the importance of these factors from other modeling choices (e.g., the pretraining objective), we begin by training \ourmodel{} following the \bertlarge{} architecture ($L=24$, $H=1024$, $A=16$, 355M parameters).
We pretrain for 100K steps over a comparable \textsc{BookCorpus} plus \textsc{Wikipedia} dataset as was used in \newcite{devlin2018bert}.
We pretrain our model using 1024 V100 GPUs for approximately one day.

\paragraph{Results}

We present our results in Table~\ref{tab:ablation}.
When controlling for training data, we observe that \ourmodel{} provides a large improvement over the originally reported \bertlarge{} results, reaffirming the importance of the design choices we explored in Section~\ref{sec:design}.

Next, we combine this data with the three additional datasets described in Section~\ref{sec:data}.
We train \ourmodel{} over the combined data with the same number of training steps as before (100K).
In total, we pretrain over 160GB of text.
We observe further improvements in performance across all downstream tasks, validating the importance of data size and diversity in pretraining.\footnote{Our experiments conflate increases in data size and diversity. We leave a more careful analysis of these two dimensions to future work.}

\begin{table*}[t]
\begin{center}
\begin{tabular}{lcccccccccc}
\toprule
& \bf MNLI & \bf QNLI & \bf QQP & \bf RTE & \bf SST & \bf MRPC & \bf CoLA & \bf STS & \bf WNLI & \bf Avg \\
\midrule 
\multicolumn{10}{l}{\textit{Single-task single models on dev}}\\
\bertlarge{} & 86.6/- & 92.3 & 91.3 & 70.4 & 93.2 & 88.0 & 60.6 & 90.0 & - & -\\
\xlnetlarge{} & 89.8/- & 93.9 & 91.8 & 83.8 & 95.6 & 89.2 & 63.6 & 91.8 & - & -\\
\ourmodel{} & \textbf{90.2}/\textbf{90.2} & \textbf{94.7} & \textbf{92.2} & \textbf{86.6} & \textbf{96.4} & \textbf{90.9} & \textbf{68.0} & \textbf{92.4} & \textbf{91.3} & - \\
\midrule
\multicolumn{10}{l}{\textit{Ensembles on test (from leaderboard as of July 25, 2019)}} \\
ALICE & 88.2/87.9 & 95.7 & \textbf{90.7} & 83.5 & 95.2 & 92.6 & \textbf{68.6} & 91.1 & 80.8 & 86.3 \\
MT-DNN & 87.9/87.4 & 96.0 & 89.9 & 86.3 & 96.5 & 92.7 & 68.4 & 91.1 & 89.0 & 87.6 \\
XLNet  & 90.2/89.8 & 98.6 & 90.3 & 86.3 & \textbf{96.8} & \textbf{93.0} & 67.8 & 91.6 & \textbf{90.4} & 88.4 \\
\ourmodel{} & \textbf{90.8/90.2} & \textbf{98.9} & 90.2 & \textbf{88.2} & 96.7 & 92.3 & 67.8 & \textbf{92.2} & 89.0 & \bf 88.5 \\
\bottomrule
\end{tabular}
\end{center}
\caption{
Results on GLUE. All results are based on a 24-layer architecture.
\bertlarge{} and \xlnetlarge{} results are from \newcite{devlin2018bert} and \newcite{yang2019xlnet}, respectively.
\ourmodel{} results on the development set are a median over five runs.
\ourmodel{} results on the test set are ensembles of \emph{single-task} models.
For RTE, STS and MRPC we finetune starting from the MNLI model instead of the baseline pretrained model.
Averages are obtained from the GLUE leaderboard.
}
\label{tab:roberta_glue}
\end{table*}

Finally, we pretrain \ourmodel{} for significantly longer, increasing the number of pretraining steps from 100K to 300K, and then further to 500K.
We again observe significant gains in downstream task performance, and the 300K and 500K step models outperform \xlnetlarge{} across most tasks.
We note that even our longest-trained model does not appear to overfit our data and would likely benefit from additional training.

In the rest of the paper, we evaluate our best \ourmodel{} model on the three different benchmarks: GLUE, SQuaD and RACE.
Specifically we consider \ourmodel{} trained for 500K steps over all five of the datasets introduced in Section~\ref{sec:data}.

\subsection{GLUE Results} \label{sec:results_glue}

For GLUE we consider two finetuning settings.
In the first setting (\emph{single-task, dev}) we finetune \ourmodel{} separately for each of the GLUE tasks, using only the training data for the corresponding task.
We consider a limited hyperparameter sweep for each task, with batch sizes $\in \{16, 32\}$ and learning rates $\in \{1e-5, 2e-5, 3e-5\}$, with a linear warmup for the first 6\% of steps followed by a linear decay to 0.
We finetune for 10 epochs and perform early stopping based on each task's evaluation metric on the dev set.
The rest of the hyperparameters remain the same as during pretraining.
In this setting, we report the median development set results for each task over five random initializations, without model ensembling.

In the second setting (\emph{ensembles, test}), we compare \ourmodel{} to other approaches on the test set via the GLUE leaderboard.
While many submissions to the GLUE leaderboard depend on multi-task finetuning, \textbf{our submission depends only on single-task finetuning}.
For RTE, STS and MRPC we found it helpful to finetune starting from the MNLI single-task model, rather than the baseline pretrained \ourmodel{}.
We explore a slightly wider hyperparameter space, described in the Appendix, and ensemble between 5 and 7 models per task.

\paragraph{Task-specific modifications}

Two of the GLUE tasks require task-specific finetuning approaches to achieve competitive leaderboard results.

\underline{QNLI}:
Recent submissions on the GLUE leaderboard adopt a pairwise ranking formulation for the QNLI task, in which candidate answers are mined from the training set and compared to one another, and a single (question, candidate) pair is classified as positive~\cite{liu2019mtdnn,liu2019improving,yang2019xlnet}.
This formulation significantly simplifies the task, but is not directly comparable to BERT~\cite{devlin2018bert}.
Following recent work, we adopt the ranking approach for our test submission, but for direct comparison with BERT we report development set results based on a pure classification approach.

\underline{WNLI}: We found the provided NLI-format data to be challenging to work with.
Instead we use the reformatted WNLI data from SuperGLUE~\cite{wang2019superglue}, which indicates the span of the query pronoun and referent.
We finetune \ourmodel{} using the margin ranking loss from \newcite{kocijan2019surprisingly}.
For a given input sentence, we use spaCy~\cite{spacy2} to extract additional candidate noun phrases from the sentence and finetune our model so that it assigns higher scores to positive referent phrases than for any of the generated negative candidate phrases.
One unfortunate consequence of this formulation is that we can only make use of the positive training examples, which excludes over half of the provided training examples.\footnote{While we only use the provided WNLI training data, our results could potentially be improved by augmenting this with additional pronoun disambiguation datasets.}

\paragraph{Results}

We present our results in Table~\ref{tab:roberta_glue}.
In the first setting (\emph{single-task, dev}), \ourmodel{} achieves state-of-the-art results on all 9 of the GLUE task development sets.
Crucially, \ourmodel{} uses the same masked language modeling pretraining objective and architecture as \bertlarge{}, yet consistently outperforms both \bertlarge{} and \xlnetlarge{}.
This raises questions about the relative importance of model architecture and pretraining objective, compared to more mundane details like dataset size and training time that we explore in this work.

In the second setting (\emph{ensembles, test}), we submit \ourmodel{} to the GLUE leaderboard and achieve state-of-the-art results on 4 out of 9 tasks and the highest average score to date.
This is especially exciting because \ourmodel{} does not depend on multi-task finetuning, unlike most of the other top submissions.
We expect future work may further improve these results by incorporating more sophisticated multi-task finetuning procedures.

\subsection{SQuAD Results} \label{sec:results_squad}

We adopt a much simpler approach for SQuAD compared to past work.
In particular, while both BERT~\cite{devlin2018bert} and XLNet~\cite{yang2019xlnet} augment their training data with additional QA datasets, \textbf{we only finetune \ourmodel{} using the provided SQuAD training data}.
\newcite{yang2019xlnet} also employed a custom layer-wise learning rate schedule to finetune XLNet, while we use the same learning rate for all layers.

For SQuAD v1.1 we follow the same finetuning procedure as \newcite{devlin2018bert}.
For SQuAD v2.0, we additionally classify whether a given question is answerable; we train this classifier jointly with the span predictor by summing the classification and span loss terms.

\paragraph{Results}

\begin{table}[t]
\begin{center}
\begin{tabular}{lcccc}
\toprule
\multirow{2}{*}{\bf Model} & \multicolumn{2}{c}{\bf SQuAD 1.1} &\multicolumn{2}{c}{\bf SQuAD 2.0} \\
&  EM &  F1 &  EM &  F1  \\
\midrule
\multicolumn{5}{l}{\textit{Single models on dev, w/o data augmentation}}\\
\bertlarge{} &  84.1&90.9&79.0&81.8\\
\xlnetlarge{} &\bf{89.0}& 94.5&86.1&88.8\\
\ourmodel{} & 88.9 & \bf{94.6} & \bf{86.5} &\bf{89.4}\\
\midrule
\multicolumn{5}{l}{\textit{Single models on test (as of July 25, 2019)}}\\
\multicolumn{3}{l}{\xlnetlarge{}} & 86.3$^{\dag}$ & 89.1$^{\dag}$ \\
\multicolumn{3}{l}{\ourmodel{}} & 86.8 & 89.8 \\
\multicolumn{3}{l}{XLNet + SG-Net Verifier} & \textbf{87.0}$^{\dag}$ & \textbf{89.9}$^{\dag}$ \\
\bottomrule
\end{tabular}
\end{center}
\caption{
Results on SQuAD.
$\dag$ indicates results that depend on additional external training data.
\ourmodel{} uses only the provided SQuAD data in both dev and test settings.
BERT$_{\textsc{large}}$ and XLNet$_{\textsc{large}}$ results are from \newcite{devlin2018bert} and \newcite{yang2019xlnet}, respectively.
}
\label{tab:roberta_squad}
\end{table}

We present our results in Table~\ref{tab:roberta_squad}.
On the SQuAD v1.1 development set, \ourmodel{} matches the state-of-the-art set by XLNet.
On the SQuAD v2.0 development set, \ourmodel{} sets a new state-of-the-art, improving over XLNet by 0.4 points (EM) and 0.6 points (F1).

We also submit \ourmodel{} to the public SQuAD 2.0 leaderboard and evaluate its performance relative to other systems.
Most of the top systems build upon either BERT~\cite{devlin2018bert} or XLNet~\cite{yang2019xlnet}, both of which rely on additional external training data.
In contrast, our submission does not use any additional data.

Our single \ourmodel{} model outperforms all but one of the single model submissions, and is the top scoring system among those that do not rely on data augmentation.

\subsection{RACE Results} \label{sec:results_race}

In RACE, systems are provided with a passage of text, an associated question, and four candidate answers. Systems are required to classify which of the four candidate answers is correct.

We modify \ourmodel{} for this task by concatenating each candidate answer with the corresponding question and passage.
We then encode each of these four sequences and pass the resulting \emph{[CLS]} representations through a fully-connected layer, which is used to predict the correct answer.
We truncate question-answer pairs that are longer than 128 tokens and, if needed, the passage so that the total length is at most 512 tokens.


\begin{table}[t]
\begin{center}
\begin{tabular}{lccc}
\toprule
\bf Model & \bf Accuracy & \bf Middle & \bf High \\
\midrule
\multicolumn{4}{l}{\textit{Single models on test (as of July 25, 2019)}}\\
\bertlarge{} &  72.0 & 76.6 & 70.1 \\
\xlnetlarge{} & 81.7 & 85.4 & 80.2 \\
\midrule
\ourmodel{} & \bf{83.2} &  \bf{86.5} & \bf{81.3}\\
\bottomrule
\end{tabular}
\end{center}
\caption{Results on the RACE test set. BERT$_{\textsc{large}}$ and XLNet$_{\textsc{large}}$ results are from \newcite{yang2019xlnet}.}
\label{tab:roberta_race}
\end{table}


Results on the RACE test sets are presented in Table~\ref{tab:roberta_race}.
\ourmodel{} achieves state-of-the-art results on both middle-school and high-school settings.


