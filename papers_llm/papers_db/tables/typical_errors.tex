\begin{table}[]
\centering
\caption{Typical Errors}
\label{tab:examples}
\begin{adjustbox}{max width=\textwidth}
\begin{tabular}{@{}c|l@{}}
\toprule
    Prompt      & \begin{tabular}[c]{@{}l@{}}\# Get the weather forecast data in \{zipcode\} area\\ curl -X GET 'https://api.openweathermap.org/data/2.5/forecast?zip=\{zipcode\}\&appid=\{API\_KEY\}\{optional\_params\}'\\ \\ Given this API, tell me the temperature for the upcoming week in zip code 11962?\end{tabular}                                                                                                                                                                                                                                                                                                                                                                                                                                                         \\
\cmidrule(lr){1-2}
Expected results & curl -X GET 'https://api.openweathermap.org/data/2.5/forecast?zip=11962\&appid=\{API\_KEY\} \\
\midruleWrong 
Verbosity     & \begin{tabular}[c]{@{}l@{}}To get the temperature for the upcoming week in zip code 11962, you need to replace \{zipcode\} with 11962 and \{API\_KEY\} with \\your actual API key. You can also add optional parameters like units (metric or imperial) to get the temperature in Celsius or Fahrenheit. \\ Here's an example using the metric system:\\ \\ ```bash\\ curl -X GET 'https://api.openweathermap.org/data/2.5/forecast?zip=11962\&appid=YOUR\_API\_KEY\&units=metric | jq -r '.list[].main.temp''\\ ```\\ \\ Replace YOUR\_API\_KEY with your actual API key. This command will return a JSON object containing the weather forecast data \\ for the upcoming week in the 11962 area. You can then parse the JSON object to extract the temperature information for each day.\end{tabular} \\
% \midrule
% llama-30b & \textbackslash n\textbackslash n\textbackslash n\textbackslash n\textbackslash n\textbackslash n\textbackslash n\textbackslash n\textbackslash n\textbackslash n\textbackslash n\textbackslash n\textbackslash n\textbackslash n\textbackslash n\textbackslash n\textbackslash n\textbackslash n\textbackslash n\textbackslash n\textbackslash n\textbackslash n\textbackslash n\textbackslash n\textbackslash n\textbackslash n\textbackslash n\textbackslash n                                                                                                                                                                                                                                                                                                                                                                                                                                                                                                                                                                                                                                                                                                                                    \\
% \midrule
% neox-20b  & \begin{tabular}[c]{@{}l@{}}You can use the following code:\\ import requests\\ \\ zipcode = '11962'\\ api\_key = 'your\_api\_key'\\ \\ url = 'https://api.openweathermap.org/data/2.5/forecast?zip=\{zipcode\}\&appid=\{api\_key\}'\\ \\ response = requests.get(url.format(zipcode=zipcode, api\_key=api\_key))\\ \\ print(response.json())\end{tabular}                                                                                                                                                                                                                                                                                                                                                                                                           \\
\cmidrule(lr){1-2}
Wrong API   & \begin{tabular}[c]{@{}l@{}} curl -X GET 'https://api.openweathermap.org/data/2.5/{\color{red}\textbf{air\_pollution}}?city=11962\&appid=\{API\_KEY\}\{optional\_params\}'\end{tabular}         \\
\cmidrule(lr){1-2}
Wrong Args.   & \begin{tabular}[c]{@{}l@{}}curl -X GET 'https://api.openweathermap.org/data/2.5/forecast?{\color{red}\textbf{city}}=11962\&appid=\{API\_KEY\}\{optional\_params\}'\end{tabular}         \\
\bottomrule
\end{tabular}
\end{adjustbox}
\end{table}



% \begin{table}[]
% \centering
% \caption{Examples}
% \label{tab:all_models}
% \begin{adjustbox}{max width=\textwidth}
% \begin{tabular}{@{}l|l|l@{}}
%           & \begin{tabular}[c]{@{}l@{}}\# Get the weather forecast data in \{zipcode\} area\\ curl -X GET 'https://api.openweathermap.org/data/2.5/forecast?zip=\{zipcode\}\&appid=\{API\_KEY\}\{optional\_params\}'\\ \\ Given this API, tell me the temperature for the upcoming week in zip code 11962? Answer in shell command only\end{tabular}                                                                                                                                                                                                                                                                                                                                                                                                                                                                                                                                                                                        & \begin{tabular}[c]{@{}l@{}}\# Get the weather forecast data in \{zipcode\} area\\ curl -X GET 'https://api.openweathermap.org/data/2.5/forecast?zip=\{zipcode\}\&appid=\{API\_KEY\}\{optional\_params\}'\\ \\ Given this API, tell me the temperature for the upcoming week in zip code 11962?\end{tabular}                                                                                                                                                                                                                                                                                                                                                                                                                                                         \\
% Chat-gpt  & \begin{tabular}[c]{@{}l@{}}Sure, here's the shell command to get the temperature for the upcoming week in zip code 11962 using the OpenWeatherMap API:\\ \\ curl -s 'https://api.openweathermap.org/data/2.5/forecast?zip=11962\&units=imperial\&appid=\{API\_KEY\}' | jq '.list{[}{]}.main.temp'\\ This command uses curl to send a GET request to the OpenWeatherMap API with the zip parameter set to 11962, the units parameter set to imperial (to get the temperature in Fahrenheit), and the appid parameter set to your API key.\\ \\ The output of this command is piped to jq, which is a command-line JSON processor. The jq expression .list{[}{]}.main.temp extracts the temp field from the main object in each item in the list array of the JSON response.\\ \\ The output will be the temperature values for the upcoming week at 3-hour intervals, so you'll get 56 temperature values in total.\end{tabular} & \begin{tabular}[c]{@{}l@{}}To get the temperature for the upcoming week in zip code 11962, you can make the following cURL request, replacing \{API\_KEY\} with your OpenWeatherMap API key:\\ \\ curl -X GET 'https://api.openweathermap.org/data/2.5/forecast?zip=11962\&appid=\{API\_KEY\}\&units=imperial'\\ This request will return a JSON response that includes the weather forecast data for the next five days, with temperature values in Fahrenheit. You can parse the JSON response to extract the temperature values for each day.\end{tabular}                                                                                                                                                                                                       \\
% gpt-4     & curl -X GET 'https://api.openweathermap.org/data/2.5/forecast?zip=11962\&appid=\{API\_KEY\}\&units=imperial'                                                                                                                                                                                                                                                                                                                                                                                                                                                                                                                                                                                                                                                                                                                                                                                                                    & \begin{tabular}[c]{@{}l@{}}To get the temperature for the upcoming week in zip code 11962, you need to replace \{zipcode\} with 11962 and \{API\_KEY\} with your actual API key. You can also add optional parameters like units (metric or imperial) to get the temperature in Celsius or Fahrenheit. Here's an example using the metric system:\\ \\ ```bash\\ curl -X GET 'https://api.openweathermap.org/data/2.5/forecast?zip=11962\&appid=YOUR\_API\_KEY\&units=metric'\\ ```\\ \\ Replace YOUR\_API\_KEY with your actual API key. This command will return a JSON object containing the weather forecast data for the upcoming week in the 11962 area. You can then parse the JSON object to extract the temperature information for each day.\end{tabular} \\
% llama-30b & \textless{}empty\textgreater{}                                                                                                                                                                                                                                                                                                                                                                                                                                                                                                                                                                                                                                                                                                                                                                                                                                                                                                  & \textless{}empty\textgreater{}                                                                                                                                                                                                                                                                                                                                                                                                                                                                                                                                                                                                                                                                                                                                      \\
% neox-20b  & \begin{tabular}[c]{@{}l@{}}You can use the following command:\\ curl -s 'https://api.openweathermap.org/data/2.5/forecast?zip=11962\&appid=\textless{}API\_KEY\textgreater{}\&units=metric\&cnt=7' | grep -oP '(?\textless{}=temp\textbackslash{}s*=\textbackslash{}s*)\textbackslash{}d+(?=\textbackslash{}s*C)'\\ \\ The output is:\\ \\ 25\end{tabular}                                                                                                                                                                                                                                                                                                                                                                                                                                                                                                                                                                      & \begin{tabular}[c]{@{}l@{}}You can use the following code:\\ import requests\\ \\ zipcode = '11962'\\ api\_key = 'your\_api\_key'\\ \\ url = 'https://api.openweathermap.org/data/2.5/forecast?zip=\{zipcode\}\&appid=\{api\_key\}'\\ \\ response = requests.get(url.format(zipcode=zipcode, api\_key=api\_key))\\ \\ print(response.json())\end{tabular}                                                                                                                                                                                                                                                                                                                                                                                                           \\
% codegen   & curl -X GET 'https://api.openweathermap.org/data/2.5/forecast?zip=11962\&appid=\{API\_KEY\}\{optional\_params\}'                                                                                                                                                                                                                                                                                                                                                                                                                                                                                                                                                                                                                                                                                                                                                                                                                & \begin{tabular}[c]{@{}l@{}}\# Get the weather forecast data in zip code 11962\\ curl -X GET 'https://api.openweathermap.org/data/2.5/forecast?zip=11962\&appid=\{API\_KEY\}\{optional\_params\}'\end{tabular}                                                                                                                                                                                                                                                                                                                                                                                                                                                                                                                                                      
% \end{tabular}
% \end{adjustbox}
% \end{table}