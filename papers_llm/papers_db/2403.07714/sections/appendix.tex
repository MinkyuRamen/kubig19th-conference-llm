
\section{Comparison of Reported and Reproduced Performance}
Detailed comparison scores of reported and reproduced performance are shown in \Cref{tab:performance_comparison}.

\begin{table}[ht!]
    \centering
    \small
    \begin{tabular}{lcc}
        \toprule
        {\textbf{Method}} & Reported & Reproduced \\
        \midrule
         GPT 3.5 Turbo 0613 + CoT & 41.5 & 35.2 \textcolor{red}{{\tiny 
         -32.5\%}} \\
         GPT 3.5 Turbo 0613 + DFS & 54.5 & 53.2 \textcolor{red}{{\tiny 
         -2.4\%}}\\
         ToolLLaMA v2 + CoT & 25.0 & 15.0 \textcolor{red}{{\tiny 
         -40\%}} \\
         ToolLLaMA v2 + DFS & 57.0 & 34.0 \textcolor{red}{{\tiny 
         -40.4\%}}\\
         \bottomrule
    \end{tabular}
    \caption{Comparison of performance (Pass Rate) reported in the paper and reproduced by us of ChatGPT and ToolLLaMA v2 on the I1-Instruction group of ToolBench. }
    \label{tab:performance_comparison}
\end{table}

\section{Statistics of API change information}
Detailed statistics of API change categories and information are shown in \Cref{tab:api_change} and \Cref{tab:api_not_available}.

\begin{table}[h!]
    \centering
    \small
    \begin{tabular}{lcc}
     \toprule
    \textbf{Status Type} & \textbf{Number} & \textbf{Percentage} (\%) \\
    \midrule
    % Not Available & 7504 & 45.6 \\
    Not Available & 8095 & 49.2 \\
    % \quad-- Not Connectable & 2426 & 14.7 \\
    % \quad-- Not Found & 583 & 3.5 \\
    % \quad-- Parameter Issues & 591 & 3.6 \\
    % \quad-- Other issues & 4495 & 27.3 \\
    Not Authorised & 1058 & 6.4 \\
    % Parameter Change & 591 & 3.6 \\
    Success & 7311 & 44.4 \\
     \bottomrule
    \end{tabular}
    \caption{APIs changed in ToolBench.}
    \label{tab:api_change}
\end{table}



\begin{table}[h!]
    \centering
    \small
    \begin{tabular}{lcc}
     \toprule
    \textbf{Status Type} & \textbf{Number} & \textbf{Percentage} (\%) \\
    \midrule
    Not Connectable & 2426 & 30.0 \\
    Not Found & 583 & 7.2 \\
    Parameter Change & 591 & 7.3 \\
    Parsing Error & 4247 & 52.6 \\
    Other & 248 & 3.1 \\
    \midrule
    Total & 8095 & 100 \\
     \bottomrule
    \end{tabular}
    \caption{Categories of Not Availability in ToolBench.}
    \label{tab:api_not_available}
\end{table}


% \section{Stability Test Scores with Virtual API Systems}
% Detailed stability test scores are shown in \Cref{tab:real_api_stability_test}.

% \begin{table}[]
%     \centering
%     \small
%     \resizebox{\linewidth}{!}{
%     \begin{tabular}{lcccc}
%         \toprule
%         \multirow{2}{*}{\textbf{Method}} & \multicolumn{4}{c}{\textbf{Percentage of Failing Tools}}\\
%         % \cmidrule{2-5}
%           & 0\% & 10\% & 20\% & 50\% \\
%         \midrule
%          GPT 3.5 Turbo 0613 + CoT & 20.3{\tiny $\pm{0.8}$}  & 17.9{\tiny $\pm{1.2}$} & 16.3{\tiny $\pm{0.6}$} & 12.8{\tiny $\pm{1.7}$} \\
%          GPT 3.5 Turbo 0613 + DFS & 26.6{\tiny $\pm{0.3}$} & 23.9{\tiny $\pm{1.1}$} & 23.2{\tiny $\pm{1.0}$} & 16.3{\tiny $\pm{1.2}$}\\
%          GPT 4 0613 + CoT & 21.4{\tiny $\pm{0.5}$}  & 19.6{\tiny $\pm{0.9}$} & 15.5{\tiny $\pm{0.3}$} & 11.8{\tiny $\pm{0.8}$} \\
%          GPT 4 0613 + DFS &  24.2{\tiny $\pm{1.8}$}  & 24.0{\tiny $\pm{0.8}$} & 21.2{\tiny $\pm{1.8}$} & 16.9{\tiny $\pm{0.4}$}\\
%          \bottomrule
%     \end{tabular}
%     }
%     \caption{SoPR change when manually make APIs down on the I1 Instruction group.}
%     \label{tab:real_api_stability_test}
% \end{table}
\section{Stability Test Scores with Virtual API Systems}
\label{app:detailed_stability_test_virtual}
Detailed scores of stability tests of various models are shown in \Cref{tab:simulated_api_stability_test}. Note that in addition to GPT 3.5 Turbo 0613 and GPT 4 0613, we report the performance of newer versions, namely GPT 3.5 Turbo 1106 and GPT 4 Turbo Preview.
\begin{table}[]
    \centering
    \small
    \resizebox{\linewidth}{!}{
    \begin{tabular}{ccccc}
        \toprule
        \multirow{2}{*}{\textbf{Method}} & \multicolumn{4}{c}{\textbf{Real API Failure Rate}}\\
        \cmidrule{2-5}
          & 0\% & 10\% & 20\% & 50\% \\
        \midrule
         GPT 3.5 Turbo 0613 + CoT & 49.1{\tiny $\pm{1.0}$}  & 48.7{\tiny $\pm{0.9}$} & 51.2{\tiny $\pm{1.3}$} & 49.0{\tiny $\pm{0.7}$} \\
         GPT 3.5 Turbo 0613 + DFS & 68.1{\tiny $\pm{1.4}$}  & 70.9{\tiny $\pm{1.3}$} & 67.5{\tiny $\pm{1.8}$} & 67.3{\tiny $\pm{1.3}$}\\
         GPT 4 0613 + CoT & 55.4{\tiny $\pm{0.6}$}  & 55.5{\tiny $\pm{1.0}$} & 58.0{\tiny $\pm{0.5}$} & 55.2{\tiny $\pm{0.6}$} \\
         GPT 4 0613 + DFS & 69.7{\tiny $\pm{1.4}$}  & 71.4{\tiny $\pm{1.4}$} & 71.2{\tiny $\pm{0.9}$} & 69.9{\tiny $\pm{0.9}$}\\
         \midrule
         GPT 3.5 Turbo 1106 + CoT & 52.1{\tiny $\pm{0.7}$}  & 52.4{\tiny $\pm{0.8}$} & 53.9{\tiny $\pm{0.6}$} & 50.2{\tiny $\pm{0.6}$} \\
         GPT 3.5 Turbo 1106 + DFS & 69.9{\tiny $\pm{0.7}$}  & 71.7{\tiny $\pm{0.7}$} & 69.4{\tiny $\pm{0.8}$} & 71.6{\tiny $\pm{0.9}$}\\
         GPT 4 Turbo preview + CoT & 60.8{\tiny $\pm{0.7}$}  & 62.8{\tiny $\pm{0.5}$} & 64.2{\tiny $\pm{0.7}$} & 62.4{\tiny $\pm{0.5}$} \\
         GPT 4 Turbo preview + DFS& 73.2{\tiny $\pm{1.1}$}  & 76.7{\tiny $\pm{1.0}$} & 76.0{\tiny $\pm{0.8}$} & 74.2{\tiny $\pm{1.3}$}\\     
         \bottomrule
    \end{tabular}
    }
    \caption{Performance change when manually make APIs down with our virtual online API system. The results are averaged over all six groups. Solving rates are reported. We run each experiment one time and evaluate three times and take the average score.}
    \label{tab:simulated_api_stability_test}
\end{table}




\section{Call Error Identification and Cache Filtering Rule}\label{app:filter_rule}
We identify call errors and filter out invalid call to RapidAPI based on keyword occurences. In detail, we identify the following error:
\begin{itemize}
    \item Not Connected Error: when error information contains \texttt{HTTP} or the response infomation contains \texttt{HTTP error, connection, rate limit, time(d) out};
    \item Not Found Error: when the error information or response contains \texttt{not found, not available, API doesn't exists, Service Not Found, internal error} or 404 error message;
    \item Parameter Change: when the error information or response contains \texttt{parameter, parse, is not defined};
    \item Parsing Error: when the error information starts with \texttt{Function executing from};
    \item Not Authorised: when the error information or response contains \texttt{authoriz(s), unauthoriz(s), blocked user, unsubscribe, credential, disabled for your subscription, ACCESS\_DENIED} or 401, 403 error message;
    \item Other Errors: messages with non-empty error messages;
    \item Success: Other calls.
\end{itemize}
We consider all types of errors when identifying errors. However, when filtering the cache, we do not conside the``Other Errors''.

\begin{table*}[ht!]
    % \small
    \centering
    % \resizebox{\columnwidth}{!}{
    \begin{tabular}{p{0.1\textwidth}p{0.8\textwidth}}
    \toprule
    \rowcolor[gray]{0.95} 
    \multicolumn{2}{c}{\textbf{API Simulation Prompt}} \\
    \midrule
    System & \makecell[{{p{.8\textwidth}}}]{
    Imagine you are an API Server operating within a specialized tool, which contains a collection of distinct APIs. Your role is to deeply understand the function of each API based on their descriptions in the API documentation. As you receive specific inputs for individual API calls within this tool, analyze these inputs to determine their intended purpose. Your task is to craft a JSON formatted response that aligns with the expected output of the API, guided by the provided examples. \\
    Your responses must adhere to a specific JSON structure, which is as follows: \\
    \texttt{\{
        ``error'': ``'',
        ``response'': ``Your\_Response''
    \}}\\
The error field should remain empty, indicating no errors in processing. The response field should contain the content you formulate based on the API's functionality and the input provided. Ensure that your responses are meaningful, directly addressing the API's intended functionality. If the provided examples are mostly error messages or lack substantial content, use your judgment to create relevant and accurate responses. The key is to maintain the JSON format's integrity while ensuring that your response is an accurate reflection of the API's intended output within the tool.\\
Please note that your answer should not contain anything other than a json format object, which should be parsable directly to json. \\
Note that: \\
- your response should be around 100 to 200 words, containing rich information given the api input parameters. Keep Your answer short and simple.\\
- your response must be effective and have practical content.\\
- if the api response example if null or ineffective, ignore the example and give your independent response. \\
    } \\
    \hline
    User & \makecell[{{p{.85\linewidth}}}]{
    API Documentation:\\
    \texttt{Documentation JSON file}\\
    API Examples: \\
    \texttt{Example input 1: Example response 1}\\
    \texttt{Example input 2: Example response 2}\\
    \texttt{Example input 3: Example response 3}\\
    API input:\\
    \texttt{Argument JSON string, e.g:} \\
    \texttt{\{``category'':``Logistics'',}\texttt{``tool\_name'': ``SQUAKE'',}\\
    \texttt{``api\_name'': ``Checkhealth'',``tool\_input'': ``\{\}'',}\\
    \texttt{``strip'': ``filter''\}}
    } \\
    \bottomrule
    \end{tabular}
    \caption{Prompt used to simulate APIs.}
    \label{tab:prompt_simulate_api}
\end{table*}

\begin{table*}[ht!]
    % \small
    \centering
    % \resizebox{\columnwidth}{!}{
    \begin{tabular}{l}
    \toprule
    \rowcolor[gray]{0.95} 
    \textbf{Solvable Task Filtration Prompt} \\
    \midrule
    \makecell[l{p{\textwidth}}]{
    Please check whether the given task solvable with following rules:\\
    1. If the \texttt{query} provide invalid information (e.g. invalid email address or phone number), return \texttt{Unsolvable}\\
    2. If the \texttt{query} needs more information to solve (e.g. the target restaurant name in a navigation task), return \texttt{Unsolvable} \\
    3. If the current \texttt{available\_tools} are enough to solve the query, return \texttt{Solvable} \\
    4. Return only \texttt{Solvable} or \texttt{Unsolvable} \\
    \\
    Task:\{\texttt{task}\}
    \\
    Now please give your answer (only \texttt{Solvable} or \texttt{Unsolvable}):
}
    \\
    \bottomrule
    \end{tabular}
    \caption{Prompt used to filter solvable tasks.}
    \label{tab:task_solvability}
\end{table*}

\section{Configurations of API Diversity Analysis }\label{app:diversity_conf}
The configurations of diversity analysis are as follows:
\begin{itemize}
    \item Embedding model: \texttt{all-mpnet-base-v2};
    \item UMAP metric (distance metric): correlation;
    \item Num of neighbours: 15;
    \item Min distance: 0.5.
\end{itemize}


\section{Prompts of API simulation}
\label{app:prompt_simulation}


The prompt used to simulate API behaviours is shown in \Cref{tab:prompt_simulate_api}.


\section{Prompt to Filter Solvable Task}
\label{app:prompt_task_solvability}
The prompt used to filter solvable tasks is shown in \Cref{tab:task_solvability}.




\section{Prompt Used to Make API Calls}\label{app:prompt_make_call}
The prompt used to construct API calls to scan availables is shown in \Cref{tab:prompt_api_call}.

\begin{table*}[t!]
    % \small
    \centering
    % \resizebox{\columnwidth}{!}{
    \begin{tabular}{p{0.1\textwidth}p{0.8\textwidth}}
    \toprule
    \rowcolor[gray]{0.95} 
    \multicolumn{2}{c}{\textbf{API Call Writing Prompt}} \\
    \midrule
    System & \makecell[{{p{.8\textwidth}}}]{
Imagine you are an API requester, Your role is to deeply understand the function of each API based on their descriptions in the API documentation.  Your task is to craft a JSON formatted input that aligns with the expected input of the API, guided by the provided examples.\\
Your responses must adhere to a specific JSON structure, which is as follows:\\
Please note that your answer should not contain anything other than a json format object, which should be parsable directly to json. \\
Note that:\\
- your response should be around 100 to 500 words, containing rich information given the api input parameters.\\
- your response must be effective and have practical content.\\
- if the api response example if null or ineffective, ignore the example and give your independent response.\\
    } \\
    \hline
    User & \makecell[{{p{.85\linewidth}}}]{
    API Documentation:\\
    \texttt{Documentation JSON file}\\
    API Examples (if available): \\
    \texttt{Example input 1: Example response 1}\\
    \texttt{Example input 2: Example response 2}\\
    \texttt{Example input 3: Example response 3}\\
    one more API Input example:\\
    } \\
    \bottomrule
    \end{tabular}
    \caption{Prompt used to write API calls.}
    \label{tab:prompt_api_call}
\end{table*}




