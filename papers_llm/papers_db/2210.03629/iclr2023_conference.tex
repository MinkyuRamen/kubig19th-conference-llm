
\documentclass{article} %
\usepackage{iclr2023_conference,times}

\input{math_commands.tex}


\usepackage{hyperref}
\usepackage{url}
\usepackage[utf8]{inputenc} %
\usepackage[T1]{fontenc}    %
\usepackage{hyperref}       %
\usepackage{url}            %
\usepackage{booktabs}       %
\usepackage{amsfonts}       %
\usepackage{nicefrac}       %
\usepackage{microtype}      %
\usepackage{xcolor}         %
\usepackage{colortbl}
\usepackage{soul}
\usepackage{booktabs}       %
\usepackage{amsfonts}       %
\usepackage{nicefrac}       %
\usepackage{microtype}      %
\usepackage{xcolor}         %
\usepackage{paralist}       %
\usepackage{graphicx}
\usepackage{amsmath}
\usepackage[ruled,vlined]{algorithm2e}
\usepackage{caption}
\usepackage{subcaption}
\usepackage{diagbox}
\usepackage{bbding}
\usepackage{booktabs}
\usepackage{pifont}
\usepackage{wrapfig}
\usepackage{lipsum}
\usepackage{capt-of}    %
\usepackage{tabularx}   %
\usepackage{multirow}
\usepackage{makecell}
\usepackage{bbm}
\usepackage{pdfpages}
\usepackage{array, makecell} %
\usepackage{xcolor}
\usepackage{longtable}
\usepackage{float}
\usepackage{authblk}
\renewcommand\Authands{, } %


\title{\model: Synergizing Reasoning and Acting in Language Models}




\author[*,1]{{Shunyu Yao}\thanks{Work during Google internship. Projet page with code: \url{https://react-lm.github.io/}.
}}
\author[2]{{Jeffrey Zhao}}
\author[2]{{Dian Yu}}
\author[2]{{Nan Du}}
\author[2]{{Izhak Shafran}}
\author[1]{{Karthik Narasimhan}}
\author[2]{{Yuan Cao}}
\affil[1]{Department of Computer Science, Princeton University}
\affil[2]{Google Research, Brain team}
\affil[1]{\texttt{\{shunyuy,karthikn\}@princeton.edu}}
\affil[2]{\texttt {\{jeffreyzhao,dianyu,dunan,izhak,yuancao\}@google.com}}


\newcommand{\fix}{\marginpar{FIX}}
\newcommand{\new}{\marginpar{NEW}}
\newcommand{\model}{\texttt{ReAct}}
\newcommand{\reason}{\texttt{CoT}}
\newcommand{\reasons}{\texttt{CoT-SC}}
\newcommand{\act}{\texttt{Act}}
\newcommand{\modelim}{\texttt{ReAct-IM}}
\newcommand{\palm}{\texttt{Standard}}


\newcommand{\myparagraph}[1]{\paragraph{#1}}

\newcommand{\todo}[1]{\textcolor{red}{[TODO: #1]}} 
\newcommand{\sy}[1]{\textcolor{blue}{[Shunyu: #1]}}
\newcommand{\yc}[1]{\textcolor{teal}{[Yuan: #1]}}
\newcommand{\nd}[1]{\textcolor{violet}{[Nan: #1]}}
\newcommand{\kn}[1]{\textcolor{purple}{[Karthik: #1]}}
\newcommand{\dy}[1]{\textcolor{brown}{[Dian: #1]}}
\newcommand{\jz}[1]{\textcolor{orange}{[Jeff: #1]}}

\iclrfinalcopy %
\begin{document}


\maketitle

\begin{abstract}
Reinforcement Learning (RL) has achieved significant success in application domains such as robotics, games and health care. However, training RL agents is very time consuming. Current implementations exhibit poor performance due to challenges such as irregular memory accesses and thread-level synchronization overheads on CPU.
In this work, we propose a framework for generating scalable reinforcement learning implementations on multi-core systems. Replay Buffer is a key component of RL algorithms which facilitates storage of samples obtained from environmental interactions and data sampling for the learning process. We define a new data structure for Prioritized Replay Buffer based on $K$-ary sum tree that supports asynchronous parallel insertions, sampling, and priority updates. To address the challenge of irregular memory accesses, we propose a novel data layout to store the nodes of the sum tree that reduces the number of cache misses. Additionally, we propose \textit{lazy writing} mechanism to reduce thread-level synchronization overheads of the Replay Buffer operations. Our framework employs parallel actors to concurrently collect data via environmental interactions, and parallel learners to perform stochastic gradient descent using the collected data. Our framework supports a wide range of reinforcement learning algorithms including DQN, DDPG, etc. We demonstrate the effectiveness of our framework in accelerating RL algorithms by performing experiments on CPU + GPU platform using OpenAI benchmarks. 
Our results show that the performance of our $K$-ary sum tree based Prioritized Replay Buffer improves the baseline implementations by around 4x$\sim$100x. Our proposed synchronization optimizations improve the performance by around 2x$\sim$4.4x compared with using a global lock. 
By plugging our Replay Buffer implementation into existing open source reinforcement learning frameworks, we achieve 1.19x$\sim$1.75x speedup for various algorithms.

    % Reinforcement Learning (RL) has achieved significant success application domains such as robotics, games, health care and others. 
    % However, training RL agents is very time consuming, e.g it takes weeks to train AlphaZero on hundreds of GPUs.
    % Prior works focus on parallelizing reinforcement learning on clusters while the acceleration on multicore platforms is left unexplored.
    % Replay Buffer is a key component of off-policy RL algorithms which facilitates storage of samples obtained from environmental interactions and their sampling for the learning process. 
    % However, its current implementations exhibit poor performance due to challenges such as irregular memory accesses and synchronization overheads. 
    % In this work, we propose a framework for generating scalable reinforcement learning implementations on multicore systems.
    % % Compared with clusters, data communication is significantly reduced due to the benefit of shared memory.
    % We define a new data structure for prioritized replay buffer based on K-ary sum tree that supports asynchronous parallel insertion, sampling, and priority update.
    % To address the challenge of irregular memory accesses, we propose a novel data layout to store the nodes of the sum tree that reduces the number of cache misses.
    % Additionally, we propose \textit{lazy writing} mechanism to reduce synchronization overheads of the replay buffer: only lightweight bookkeeping is performed when acquiring the lock and heavyweight workloads such as data writing are postponed after releasing the lock.
    % Our framework employs parallel actors to collect the data in the environment concurrently and parallel learners to perform stochastic gradient descent. 
    % Given hardware configurations, our framework automatically allocates the number of threads for actors and learners such that the desired ratio between the throughput of the data collection and the throughput of the learning is satisfied.
    % Our framework supports a wide range of reinforcement learning algorithms including DQN, DDPG, TD3, SAC and so on.
    % We demonstrate the effectiveness of our framework in accelerating off-policy RL algorithms by performing experiments on xxx platform using Atari benchmarks.
    % Our results shows that the performance of our approach scales in linear with the number of cores. 
    % Compared with the baseline approaches, we reduce the convergence time by $x\%\sim x\%$. 
    % By plugging our replay buffer implementation into existing open source reinforcement learning frameworks, we achieve $x\%\sim x\%$ speedup.
\end{abstract}
\section{Introduction}
\label{sec:intro}

%------------------------------------------------------------------------------
\begin{figure*}
\centering
\tikzfading[
	name=fade out,
	inner color=transparent!90,
	outer color=transparent!10,
]

%------------------------------------------------------------------------------
\begin{tikzpicture}[
	scale=.3,
	font={\footnotesize},
	node distance=.5,
	ovr/.style={fill=white,fill opacity=.9},
	ten/.style={draw,ovr},
	ops/.style={op,ovr},
	rec/.style args={(#1/#2)}{draw,rectc,minimum width=#1cm,minimum height=#2cm,preaction={ovr}},
	att/.style={fill,path fading=fade out},
	atth/.style={att,fading transform={yscale=10}},
	dim/.style={text opacity=.5,inner sep=2pt,below=2pt of #1},
	sym/.style={above=3pt of #1},
	symt/.style={above=8pt of #1-northwest},
	key/.style={red,left,near end},
	back/.style={draw=#1!60,fill=#1!30,fill opacity=.5,inner sep=4},
	xback/.style={back=#1,inner xsep=8},
	yback/.style={back=#1,inner ysep=8},
]
\matrix[
	tight,
	row sep={38,between origins},column sep=12,
	cells={scale=.3,},
	nodes={node distance=.5},
] {
	\&[4]\&
% 	local channel
	\node[rec=(4/1),atth] (lc) {};
	\node[sym=lc] {$\vA_c^l$};
	\node[dim=lc] {$c \times 1 \times 1$};
	\&
	\node[ops] (lc1) {$\times$};
	\&
	\node[ops] (lc2) {$+$};
	\&
	\para[ten]{lcf}{0,-2.5,-2}{1,5,4};
	\node[symt=lcf] {$\vF_c^l$};
	\&\&
% 	local spatial
	\node[rec=(4/4),att] (ls) {};
	\node[sym=ls] {$\vA_s^l$};
	\node[dim=ls] {$1 \times h \times w$};
	\node[ops,right=of ls] (ls1) {$\times$};
	\&
	\node[ops] (ls2) {$+$};
	\&
	\para[ten]{lsf}{0,-2.5,-2}{1,5,4};
	\node[symt=lsf] {$\vF^l$};
	\&[5]
% 	local output
	\node[ops] (l1) {$\times$};
	\\
% 	backbone
	\para[ten]{if}{0,-2.5,-2}{1,5,4};
	\node[ovr,dim=if-south] {$c \times h \times w$};
	\node[symt=if] {$\vF$};
	\&
	\node[dot](s) at(0,0){};
	\&\&\&\&\&
	\node[dot] (c){};
	\&\&\&\&
% 	skip
	\node[ops] (f1) {$\times$};
	\&
	\node[ops] (f) {$+$};
	\&
% 	glam output
	\para[ten]{of}{0,-2.5,-2}{1,5,4};
	\node[dim=of-south] {$c \times h \times w$};
	\node[symt=of] {$\vF^{gl}$};
	\\
	\&\&
% 	global channel
	\node[rec=(4/4),att] (gc) {};
	\node[sym=gc] {$\vA_c^g$};
	\node[dim=gc] {$c \times c$};
	\&
	\node[ops] (gc1) {$\times$};
	\&\&
	\para[ten]{gcf}{0,-2.5,-2}{1,5,4};
	\node[symt=gcf] {$\vF_c^g$};
	\&\&
% 	global spatial
	\node[rec=(5/5),att] (gs) {};
	\node[sym=gs] {$\vA_s^g$};
	\node[dim=gs] {$hw \times hw$};
	\node[ops,right=of gs] (gs1) {$\times$};
	\&
	\node[ops] (gs2) {$+$};
	\&
	\para[ten]{gsf}{0,-2.5,-2}{1,5,4};
	\node[symt=gsf] {$\vF^g$};
	\&
% 	global output
	\node[ops] (g1) {$\times$};
	\\
};

% weights
\node[above=of l1] (wl) {$w_l$};
\node[above=of f1] (w)  {$w$};
\node[above=of g1] (wg) {$w_g$};

% input / output
\draw (if-east)--(s);
\draw[->] (s) |- (lc);
\draw[->] (s) |- (gc);
\draw[->] (l1) -| (f);
\draw[->] (g1) -| (f);

% local stream
\draw[->]
	(lc) edge (lc1)
	(lc1) edge (lc2)
	(lc2) edge (lcf-west)
	(ls) edge (ls1)
	(ls1) edge (ls2)
	(ls2) edge (lsf-west)
	(lsf-east)--(l1)
	;

% skip stream
\draw[->]
	(s)--(f1)
	(f1) edge (f)
	(f) edge (of-west)
	;

% center connections
\draw[->] (c) |- (ls);
\draw[->] (c) |- (gs);

% global stream
\draw[->]
	(gc) edge (gc1)
	(gc1) edge (gcf-west)
	(gs) edge (gs1)
	(gs1) edge (gs2)
	(gs2) edge (gsf-west)
	(gsf-east)--(g1)
	;

% local skips
\draw (lcf-north) -- +(0,.6) coordinate (lcf-n);
\draw[->] (lcf-n) -| (ls1);
\draw[->] (lcf-n) -| (ls2);

% global skips
\draw (gcf-south) -- +(0,-.6) coordinate (gcf-s);
\draw[->] (gcf-s) -| (gs1);
\draw[->] (gcf-s) -| (gs2);

% input skips
\path
	(s |- lcf-n) coordinate(s-n)
	(s |- gcf-s) coordinate(s-s)
	;
\draw[->] (s-n) -| (lc1);
\draw[->] (if-north) |- (s-n) -| (lc2);
\begin{pgfonlayer}{bg1}
	\draw[->] (if-south) |- (s-s) -| (gc1);
\end{pgfonlayer}

% weighting
\draw[->]
	(wl) edge (l1)
	(w) edge (f1)
	(wg) edge (g1)
	;

\begin{pgfonlayer}{bg2}
	\node[yback=blue,fit=(s-n) (s-s) (lc2)] (channel) {};
	\node[yback=red,fit=(lcf-n -| ls1) (gcf-s -| gs1) (c) (gs2)] (spatial) {};
	\node[back=black,fit=(l1) (g1) (f)] (fusion) {};
	\node[xback=yellow,fit=(s-n) (lcf-n) (lcf-south) (lsf-east)] (local) {};
	\node[xback=green,fit=(s-s) (gcf-north) (gcf-s) (gsf-east)] (global) {};
\end{pgfonlayer}

\node[blue,below=1pt of channel]{channel attention};
\node[red,below=1pt of spatial]{spatial attention};
\node[black,below=1pt of fusion]{fusion};
\node[yellow!60!red,above=1pt of local.north east]{local attention};
\node[green!60!black,below=1pt of global.south east]{global attention};

\end{tikzpicture}
%------------------------------------------------------------------------------

\caption{Our \emph{global-local attention module} (GLAM) involves both {\color{blue}channel} and {\color{red}spatial} attention, as well as both {\color{yellow!60!red}local} attention (channels/locations weighted independently, based on contextual information obtained by pooling) and {\color{green!60!black}global} attention (based on pairwise interaction between channels/locations). As a result, four attention maps are used: \emph{local channel} ($\vA_c^l$), \emph{local spatial} ($\vA_s^l$), \emph{global channel} ($\vA_c^g$) and \emph{global spatial} ($\vA_s^g$). The input feature map $\vF$ is weighted into local ($\vF^l$) and global ($\vF^g$) attention feature maps, which are fused with $\vF$ to yield the \emph{global-local attention feature map} $\vF^{gl}$. The diagram is abstract: The four attention modules are shown in more detail in Figures \ref{fig:fig4}, \ref{fig:fig3}, \ref{fig:fig6}, \ref{fig:fig5}.}
\label{fig:glam}
\end{figure*}
%------------------------------------------------------------------------------

Instance-level image retrieval is at the core of visual representation learning and is connected with many problems of visual recognition and machine learning, for instance \emph{metric learning}~\cite{oh2016deep,KKCK20}, \emph{few-shot learning}~\cite{SnellSZ17} and \emph{unsupervised learning}~\cite{chen2020simple}. Many large-scale open datasets~\cite{Babenko01, Radenovic01, Gordo01, Noh01, Weyand01}, and competitions\footnote{https://www.kaggle.com/c/landmark-retrieval-2020} have accelerated progress in instance-level image retrieval, which has been transformed by deep learning~\cite{Babenko01}.

Many studies on instance-level image retrieval focus on learning features from \emph{convolutional neural networks} (CNN), while others focus on \emph{re-ranking}, for instance by graph-based methods~\cite{Donoser01}. The former can be distinguished according to feature types: \emph{local descriptors}, reminiscent of SIFT~\cite{Lowe01}, where an image is mapped to a few hundred vectors; and \emph{global descriptors}, where an image is mapped to a single vector. In fact, deep learning has brought global descriptors with astounding performance, while allowing efficient search. Our study belongs to this type.

Studies on global descriptors have focused on \emph{spatial pooling}~\cite{Babenko03,Radenovic01}. The need for compact, discriminative representations that are resistant to clutter has naturally given rise to \emph{spatial attention} methods~\cite{Kalantidis01,Ng01}. Different kinds of attention have been studied in many areas of computer vision research. There is also \emph{channel attention}~\cite{Hu01,ChenKLYF18}; \emph{local attention}, applied independently to elements of the representation (feature map)~\cite{woo01,Kim01}; \emph{global attention}, based on interaction between elements~\cite{Wang02,ChenKLYF18}; and combinations thereof. Unfortunately, each study has been limited to one or two kinds of attention only; attention is not always learned; and applications vary.

It is the objective of our work to perform a comprehensive study of all forms of attention above, apply them to instance-level image retrieval and provide a detailed account of their interaction and impact on performance. As shown in \autoref{fig:glam}, we collect contextual information from images with both \emph{local} and \emph{global} attention, giving rise to two parallel network streams. Importantly, each operates on both \emph{spatial locations} and \emph{feature channels}. Local attention is about individual locations and channels; global is about interaction between locations and between channels. The extracted information is separately embedded in local and global attention feature maps, which are combined in a \emph{global-local attention feature map} before pooling.

Our contributions can be summarized as follows:
\begin{enumerate}[itemsep=2pt, parsep=0pt, topsep=0pt]
	 \item We propose a novel network that consists of both global and local attention for image retrieval. This is the first study that employs both mechanisms.
	 \item Each of the global and local attention mechanisms comprises both spatial and channel attention.
	 \item Focusing on global descriptors, we provide empirical evidence of the interaction of all forms of attention and improve the state of the art on standard benchmarks.
\end{enumerate}

\section{Global-local attention}
\label{sec:method}

We design a \emph{global-local attention module} (GLAM), which is attached at the end of a backbone network. \autoref{fig:glam} illustrates its main components. We are given a $c\times h \times w$ feature tensor $\vF$, where $c$ is the number of channels, and $h \times w$ is the spatial resolution. Local attention collects context from the image and applies pooling to obtain a $c \times 1 \times 1$ \emph{local channel attention map} $\vA_c^l$ and a $1 \times h \times w$ \emph{local spatial attention map} $\vA_s^l$. Global attention allows interaction between channels, resulting in a $c \times c$ \emph{global channel attention map} $\vA_c^g$, and between spatial locations, resulting in a $hw \times hw$ \emph{global spatial attention map} $\vA_s^g$. The feature maps produced by the two attention streams are combined with the original one by a learned fusion mechanism into the \emph{global-local attention feature map} $\vF^{gl}$ before being spatially pooled into a global image descriptor.

%------------------------------------------------------------------------------
\begin{figure}
\centering
%------------------------------------------------------------------------------
\begin{tikzpicture}[
	font={\footnotesize},
]
\matrix[
	row sep=15pt,column sep=10pt,cells={scale=1,},
]{
	\& \node[inp](i){feature map}; \\
	\& \node[down](d1){conv $1 \times 1$}; \\
	\node[fix](c3){conv $3 \times 3$}; \&
	\node[fix](c5){conv $5 \times 5$}; \&
	\node[fix](c7){conv $7 \times 7$}; \\
	\& \node[cat](ca){concat}; \\
	\& \node[down](d2){conv $1 \times 1$}; \\
	\& \node[outp](o){attention map}; \\
};
\coordinate(s) at($(d1.south)!.5!(c5.north)$); % split
\coordinate(m) at($(c5.south)!.5!(ca.north)$); % merge
\node(left)[left=5pt of c3.west]{};
\coordinate(l) at(left.center);
\draw[->]
	(i) edge node[dim,midway,right]{$c \times h \times w$} (d1)
	(d1) edge (c5)
	(c5) edge (ca)
	(ca) edge node[dim,midway,right]{$4c' \times h \times w$} (d2)
	(d2) edge node[dim,midway,right]{$1 \times h \times w$} (o)
	;
\draw[->] (s) -| (c3);
\draw[->] (s) -| node[dim,midway,above]{$c' \times h \times w$} (c7);
\draw (s) -| (l) (l) |- (m);
\draw (c3) |- (m) (c7) |- (m);
\node[dim,below right=2pt of c7.south](dil) {\emph{dilated} \\ \emph{conv}};
\node[left=1pt of i] {$\vF$};
\node[left=1pt of d1] {$\vF'$};
\node[left=1pt of o] {$\vA_s^l$};
\end{tikzpicture}
%------------------------------------------------------------------------------

\caption{Local spatial attention. Convolutional layers in blue implemented by dilated convolutions with kernel size $3 \times 3$ and dilation factors $1,3,5$.}
\label{fig:fig3}
\end{figure}
%------------------------------------------------------------------------------

\subsection{Local attention}
\label{sec:local}

We extract an 1D channel and a 2D spatial attention map to weigh the feature map in the corresponding dimensions.

\paragraph{Local channel attention}

Following ECA-Net~\cite{wang01}, this attention captures local channel information. As shown in \autoref{fig:fig4}, we are given a $c\times h\times w$ feature tensor $\vF$ from our backbone. We first reduce it to a $c \times 1 \times 1$ tensor by \emph{global average pooling} (GAP). Channel attention is then captured by a 1D convolution of kernel size $k$ along the channel dimension, where $k$ controls the extent of cross-channel interaction. This is followed by a sigmoid function, resulting in the $c\times 1\times 1$ \emph{local channel attention map} $\vA_c^l$.

%------------------------------------------------------------------------------

\paragraph{Local spatial attention}

Inspired by the inception module~\cite{Szegedy01} and similar to~\cite{Kim01}, this attention map captures local spatial information at different scales. As shown in \autoref{fig:fig3},
given the same $c\times h\times w$ feature tensor $\vF$ from our backbone, we obtain a new tensor $\vF'$ with channels reduced to $c'$, using a ${1 \times 1}$ convolution. We then extract local spatial contextual information using convolutional filters of kernel size ${3\times 3}$, ${5\times 5}$, and ${7\times 7}$, which are efficiently implemented by ${3\times 3}$ dilated convolutions~\cite{chen2017rethinking,Yu_2017_CVPR} with dilation parameter 1, 2, and 3 respectively. The resulting features, along with one obtained by ${1\times 1}$ convolution on $\vF'$, are concatenated into a $4c' \times h \times w$ tensor. Finally, we obtain the $1 \times h \times w$ \emph{local spatial attention map} $\vA_s^l$ by a ${1\times 1}$ convolution that reduces the channel dimension to $1$.

The middle column of \autoref{fig:fig7} shows heat maps of local spatial attention, localizing target objects in images.

%------------------------------------------------------------------------------

\paragraph{Local attention feature map}

We use the local channel attention map $\vA_c^l$ to weigh $\vF$ in the channel dimension
\begin{equation}
	\vF_c^l \defn \vF \odot \vA_c^l + \vF.
\label{eq:eq3}
\end{equation}
We then use local spatial attention map $\vA_s^l$ to weigh $\vF_c^l$ in the spatial dimensions, resulting in the $c \times h \times w$ \emph{local attention feature map}
\begin{equation}
	\vF^l = \vF_c^l \odot \vA_s^l + \vF_c^l.
\label{eq:eq3-1}
\end{equation}
Here, $\vA \odot \vB$ denotes an element-wise multiplication of tensors $\vA$ and $\vB$, with broadcasting when one tensor is smaller. We adopt the choice of applying channel followed by spatial attention from \emph{convolutional block attention module} CBAM~\cite{woo01}. However, apart from computing $\vA_s^l$ at different scales, both attention maps are obtained from the original tensor $\vF$ rather than sequentially. In addition, both~\eq{eq3} and~\eq{eq3-1} include residual connections, while CBAM includes a single residual connection over both steps.

%------------------------------------------------------------------------------
\begin{figure}
\centering
\begin{tikzpicture}[
	font={\footnotesize},
]
\matrix[
	row sep=15pt,column sep=10pt,cells={scale=1,},
]{
	\node[inp](i){feature map}; \&
	\node[gap](g){GAP}; \\
	\& \node[fix](c2){conv1d($k$)};
	\& \node[fix](c3){conv1d($k$)}; \\
	\& \node[fun](s1){$\sigmoid$};
	\& \node[fun](s2){$\sigmoid$}; \\
	\& \node[op](m1){$\times$}; \\
	\node[op](m2){$\times$}; \&
	\node[fun](sm){$\softmax$}; \\
	\node[outp](o){attention feature map}; \\
};
\coordinate(s) at($(g.south)!.5!(c2.north)$); % split
\draw[->]
	(i) edge (g)
	(g)  edge (c2)
	(c2) edge node[dim,midway,right]{$1 \times c$} (s1)
	(c3) edge node[dim,midway,right]{$1 \times c$} (s2)
	(s1) edge node[dim,midway,right]{$1 \times c$} node[key,midway,left]{$\vQ_c$} (m1)
	(m1) edge node[dim,midway,right]{$c \times c$} (sm)
	(i) edge node[dim,midway,right]{$hw \times c$} node[key,midway,left]{$\vV_c$} (m2)
	(sm) edge node[midway,above]{$\vA_c^g$} (m2)
	(m2) edge node[dim,midway,right]{$c \times h \times w$} (o)
	;
\draw[->] (s) -| node[dim,midway,above]{$1 \times c$} (c3);
\draw[->] (s2) |- node[dim,near start,right]{$1 \times c$} node[key,near start,left]{$\vK_c$} (m1);
\node[left=1pt of i]{$\vF$};
\node[left=1pt of o]{$\vG_c$};
\end{tikzpicture}

\caption{Global channel attention.}
\label{fig:fig6}
\end{figure}
%------------------------------------------------------------------------------

%------------------------------------------------------------------------------

\subsection{Global attention}
\label{sec:global}

We extract two matrices capturing global pairwise channel and spatial interaction to weigh the feature map.

%------------------------------------------------------------------------------

\paragraph{Global channel attention}

We introduce a \emph{global channel attention} mechanism that captures global channel interaction. This mechanism is based on the non-local neural network~\cite{Wang02}, but with the idea of 1D convolution from ECA-Net~\cite{wang01}. As shown in \autoref{fig:fig6}, we are given the $c\times h\times w$ feature tensor $\vF$ from our backbone. We apply GAP and squeeze spatial dimensions, followed by a 1D convolution of kernel size $k$ and a sigmoid function, to obtain $1 \times c$ \emph{query} $\vQ_c$ and \emph{key} $\vK_c$ tensors. The \emph{value} tensor $\vV_c$ is obtained by mere reshaping of $\vF$ to $hw \times c$, without GAP. Next, we form the outer product of $\vK_c$ and $\vQ_c$, followed by softmax over channels to obtain a $c \times c$ \emph{global channel attention map}
\begin{equation}
	\vA_c^g = \softmax({\vK_c}\tran \vQ_c).
\label{eq:eq6-1}
\end{equation}
Finally, this attention map is multiplied with $\vV_c$ and the matrix product $\vV_c \vA_c^g$ is reshaped back to $c \times h \times w$ to give the \emph{global channel attention feature map} $\vG_c$. In GSoP~\cite{Gao_2019_CVPR} and A$^2$-Net~\cite{ChenKLYF18}, a $c \times c$ global channel attention map is obtained by multiplication of $hw \times c$ matrices; \eq{eq6-1} is more efficient, using only an outer product of $1 \times c$ vectors.

%------------------------------------------------------------------------------
\begin{figure}
\centering
\begin{tikzpicture}[
	font={\footnotesize},
]
\matrix[
	row sep=15pt,column sep=10pt,cells={scale=1,},
]{
	\& \node[inp](i){feature map}; \\
	\node[down](c1){conv $1 \times 1$}; \&
	\node[down](c2){conv $1 \times 1$}; \&
	\node[down](c3){conv $1 \times 1$}; \\
	\& \node[op](m1){$\times$}; \\
	\node[op](m2){$\times$}; \&
	\node[fun](sm){$\softmax$}; \\
	\node[up](up){conv $1 \times 1$}; \\
	\node[outp](o){attention feature map}; \\
};
\coordinate(s) at($(i.south)!.5!(c2.north)$); % split
\draw[->]
	(i) edge (c2)
	(c2) edge node[dim,midway,right]{$c' \times hw$} node[key,midway,left]{$\vQ_s$} (m1)
	(m1) edge node[dim,midway,right]{$hw \times hw$} (sm)
	(up) edge node[dim,midway,right]{$c \times h \times w$} (o)
	(c1) edge node[dim,midway,right]{$c' \times hw$} node[key,midway,left]{$\vV_s$} (m2)
	(m2) edge node[dim,midway,right]{$c' \times h \times w$} (up)
	(sm) edge node[midway,above]{$\vA_s^g$} (m2)
	;
\draw[->] (s) -| (c1);
\draw[->] (s) -| node[dim,midway,above]{$c \times h \times w$} (c3);
\draw[->] (c3) |- node[dim,near start,right]{$c' \times hw$} node[key,near start,left]{$\vK_c$} (m1);
\node[left=1pt of i]{$\vF$};
\node[left=1pt of o]{$\vG_s$};
\end{tikzpicture}

\caption{Global spatial attention.}
\label{fig:fig5}
\end{figure}
%------------------------------------------------------------------------------

\paragraph{Global spatial attention}

Since ordinary convolution applies only a local neighborhood at a time, it cannot capture global contextual information. Thus, we apply \emph{non-local filtering}~\cite{Wang02}, which is a form of \emph{self-attention}~\cite{Vaswani01} in the spatial dimensions. As shown in \autoref{fig:fig5}, we are given the same $c\times h\times w$ feature tensor $\vF$ from our backbone. By using three $1\times 1$ convolutions, which reduce channels to $c'$, and flattening spatial dimensions to $hw$, we obtain $c' \times hw$ \emph{query} $\vQ_s$, \emph{key} $\vK_s$, and \emph{value} $\vV_s$ tensors, where each column is a feature vector corresponding to a particular spatial location. We capture pairwise similarities of these vectors by matrix multiplication of $\vK_s$ and $\vQ_s$, followed by softmax over locations to obtain a $hw \times hw$ \emph{global spatial attention map}:
\begin{equation}
	\vA_s^g = \softmax(\vK_s\tran \vQ_s).
\label{eq:eq4-1}
\end{equation}
This attention map is multiplied with $\vV_s$ and the matrix product $\vV_s \vA_s^g$ is reshaped back to $c' \times h \times w$ by expanding the spatial dimensions. Finally, using a ${1\times 1}$ convolution, which increases channels back to $c$, we obtain the $c \times h\times w$ \emph{global spatial attention feature map} $\vG_s$.

The right column of \autoref{fig:fig7} shows heat maps for global spatial attention, localizing target objects in images.

%------------------------------------------------------------------------------

\paragraph{Global attention feature map}

We use the global channel attention feature map $\vF_c$ to weigh $\vF$ element-wise
\begin{equation}
	\vF_c^g = \vF \odot \vG_c.
\label{eq:eq8}
\end{equation}
We then use global spatial attention feature map $\vG_s$ to weigh $\vF_c^g$ element-wise, resulting in the $c \times h \times w$ \emph{global attention feature map}
\begin{equation}
	\vF^g = \vF_c^g \odot \vG_s + \vF_c^g.
\label{eq:eq8-1}
\end{equation}
Similarly to $\mathbf{F}^{l}$ in~\eq{eq3} and~\eq{eq3-1}, we apply channel attention first, followed by spatial attention. However, unlike \eq{eq3}, there is no residual connection in~\eq{eq8}. This choice is supported by early experiments.

%------------------------------------------------------------------------------
\begin{figure}
\centering
\small
\setlength{\tabcolsep}{2pt}
\newcommand{\heat}[1]{%
	\fig[.28]{heatmap/#1/src.png} &
	\fig[.28]{heatmap/#1/l.png} &
	\fig[.28]{heatmap/#1/g.png} \\
}
\begin{tabular}{ccc}
	\heat{1}
	\heat{2}
	\heat{4}
	(a) input &
	(b) local &
	(c) global
\end{tabular}
\caption{\emph{Local and global spatial attention}. Left: input images. Middle: local spatial attention heat maps. Right: global spatial attention heat maps. Red (blue) means higher (lower) attention weight.}
\label{fig:fig7}
\end{figure}
%------------------------------------------------------------------------------

\subsection{Global-local attention}
\label{sec:embed}

\paragraph{Feature fusion}

As shown in \autoref{fig:glam}, we combine the local and global attention feature maps, $\vF^l$ and $\vF^g$, with the original feature $\vF$. While concatenation and summation are common operations for feature combination, we use a weighted average with weights $w_l$, $w_g$, $w$ respectively, obtained by softmax over three learnable scalar parameters, to obtain a $c \times h \times w$ \emph{global-local attention feature map}
\begin{equation}
	\vF^{gl} = w_l \vF^l + w_g \vF^l + w \vF.
\label{eq:eq10}
\end{equation}
EfficientDet~\cite{Tan01} has shown that this is the most effective, among a number of choices, for fusion of features across different scales.

%------------------------------------------------------------------------------

\paragraph{Pooling}

We apply GeM~\cite{Radenovic01}, a learnable spatial pooling mechanism, to feature map $\vF^{gl}$~\eq{eq10}, followed by a fully-connected (FC) layer with dropout and batch normalization. The final embedding is obtained by $\ell_2$-normalization.

\section{Knowledge-Intensive Reasoning Tasks}
\label{sec:knowledge}
We begin with knowledge-intensive reasoning tasks like multi-hop question answering and fact verification.
As shown in Figure~\ref{fig:teaser}(1d), by interacting with a Wikipedia API, \model{} is able to retrieve information to support reasoning, while also use reasoning to target what to retrieve next, demonstrating a synergy of reasoning and acting.




\subsection{Setup}



\myparagraph{Domains} We consider two datasets challenging knowledge retrieval and reasoning: (1) HotPotQA~\citep{yang2018hotpotqa}, a multi-hop question answering benchmark that requires reasoning over two or more Wikipedia passages, and
(2) FEVER~\citep{thorne2018fever}, a fact verification benchmark where each claim is annotated SUPPORTS, REFUTES, or NOT ENOUGH INFO, based on if there exists a Wikipedia passage to verify the claim.
In this work, we operate in a question-only setup for both tasks, where models only receive the question/claim as input without access to support paragraphs, and have to rely on their internal knowledge or retrieve knowledge via interacting with an external environment to support reasoning.

\myparagraph{Action Space} We design a simple Wikipedia web API with three types of actions to support interactive information retrieval: 
(1) \textbf{\texttt{search}}[\texttt{entity}], which returns the first 5 sentences from the corresponding \texttt{entity} wiki page if it exists, or else suggests top-5 similar entities from the Wikipedia search engine, 
(2) \textbf{\texttt{lookup}}[\texttt{string}], which would return the next sentence in the page containing \texttt{string}, simulating Ctrl+F functionality on the browser. 
(3) \textbf{\texttt{finish}}[\texttt{answer}], which would finish the current task with \texttt{answer}.
We note that this action space mostly can only retrieve a small part of a passage based on exact passage name, which is significantly weaker than state-of-the-art lexical or neural retrievers. The purpose is to simulate how humans would interact with Wikipedia, and force models to retrieve via explicit reasoning in language.


\subsection{Methods}\label{sec:methods}

\myparagraph{\model{} Prompting} For HotpotQA and Fever, we randomly select 6 and 3 cases\footnote{We find more examples do not improve performance.} from the training set and manually compose \model{}-format trajectories to use as few-shot exemplars in the prompts. Similar to Figure~\ref{fig:teaser}(d), each trajectory consists of multiple thought-action-observation steps (i.e.\,dense thought), where free-form thoughts are used for various purposes.  {Specifically, we use a combination of thoughts that decompose questions (``I need to search x, find y, then find z''), extract information from Wikipedia observations (``x was started in 1844'', ``The paragraph does not tell x''), perform commonsense (``x is not y, so z must instead be...'') or arithmetic reasoning (``1844 < 1989''), guide search reformulation (``maybe I can search/look up x instead''), and synthesize the final answer (``...so the answer is x''). See Appendix~\ref{sec:prompts} for more details.}

\myparagraph{Baselines} We systematically ablate \model{}  trajectories to build prompts for multiple baselines (with formats as Figure~\ref{fig:teaser}(1a-1c)): 
(a) \textbf{Standard prompting} (\palm{}), which removes all thoughts, actions, observations in \model{} trajectories. 
(b) \textbf{Chain-of-thought prompting} (\reason{})~\citep{wei2022chain}, which removes actions and observations and serve as a reasoning-only baseline. We also build a self-consistency baseline (\reasons{})~\citep{wang2022self-consistency,wang2022rationale} by sampling 21 \reason{} trajectories with decoding temperature 0.7 during inference and adopting the majority answer, which is found to consistently boost performance over \reason{}. 
(c) \textbf{Acting-only prompt} (\act{}), which removes thoughts in \model{} trajectories, loosely resembling how WebGPT~\citep{nakano2021webgpt} interacts with the Internet to answer questions, though it operates on a different task and action space, and uses imitation and reinforcement learning instead of prompting.

\myparagraph{Combining Internal and External Knowledge} As will be detail in Section~\ref{subsec:results}, we observe that the problem solving process demonstrated by \model{} is more factual and grounded, whereas \reason{} is more accurate in formulating reasoning structure but can easily suffer from hallucinated facts or thoughts. We therefore propose to incorporate \model{}  and \reasons{}, and let the model decide when to switch to the other method based on the following heuristics:
    A) 
    \textbf{\model{} $\to$ \reasons{}}: when \model{} fails to return an answer within given steps, back off to \reasons{}. We set 7 and 5 steps for HotpotQA and FEVER respectively as we find more steps will not improve \model{} performance\footnote{{Of all trajectories with correct final answers, those with 7 steps on HotpotQA and 5 steps on FEVER only take up 0.84\% and 1.33\% respectively.}}. 
    B) 
    \textbf{  \reasons{} $\to$  \model{}}: when the majority answer among $n$ \reasons{} samples occurs less than $n/2$ times (i.e.\,internal knowledge might not support the task confidently), back off to \model{}. 


\myparagraph{Finetuning} Due to the challenge of manually annotating reasoning traces and actions at scale, we consider a bootstraping approach similar to \citet{zelikman2022star}, using 3,000 trajectories with correct answers generated by \model{} (also for other baselines) to finetune smaller language models (PaLM-8/62B) to decode trajectories (all thoughts, actions, observations) conditioned on input questions/claims. More details are in Appendix~\ref{sec:hotpot_finetune}.






\subsection{Results and Observations} \label{subsec:results}


\myparagraph{\model{} outperforms \act{} consistently} Table~\ref{table:reasoning} shows HotpotQA and Fever results using PaLM-540B as the base model with different prompting methods. 
We note that \model{} is better than \act{} on both tasks, demonstrating the value of reasoning to guide acting, especially for synthesizing the final answer, as shown in Figure~\ref{fig:teaser} (1c-d). Fine-tuning results~\ref{fig:finetune} also confirm the benefit of reasoning traces for more informed acting.

\begin{table}[t]
\begin{minipage}{.42\linewidth}
    \centering


\resizebox{\columnwidth}{!}{%
\begin{tabular}{l|c|c}
\toprule
    \multirow{2}{*}{\textbf{Prompt Method\footnote{\tiny HotpotQA EM is 27.1, 28.9, 33.8 for \palm{}, \reason{}, \reasons{} in \cite{wang2022rationale}.}}} & \textbf{HotpotQA} & \textbf{Fever}  \\
    & (EM) & (Acc) \\ 
\midrule
    \palm{}  & 28.7 & 57.1 \\
    \reason{}{\scriptsize~\citep{wei2022chain}} & 29.4 & 56.3 \\ 
    \reasons{}{\scriptsize~\citep{wang2022self-consistency}} & 33.4 & 60.4 \\
    \midrule
    \act  & 25.7 & 58.9 \\ 
    \model   & 27.4 & {60.9} \\
    \reasons{} $\to$ \model   & 34.2 & \textbf{64.6} \\
    \model $\to$ \reasons{} & \textbf{35.1} & 62.0 \\ \midrule \midrule
    \textbf{Supervised SoTA\footnote{\tiny\citep{zhu2021adaptive,lewis2020retrieval}}} & 67.5 & 89.5 \\

    
\bottomrule
\end{tabular}%
}
\caption{
PaLM-540B prompting results on HotpotQA and Fever. 
}
\label{table:reasoning}


\end{minipage}%
\hspace{5pt}
\begin{minipage}{.57\linewidth}
    \centering
    
    \includegraphics[width=.49\textwidth]{iclr2023/figure/cots_scale.pdf}
    \includegraphics[width=.49\textwidth]{iclr2023/figure/fever_cots_scale.pdf}
    \captionof{figure}{PaLM-540B prompting results with respect to number of \reasons{} samples used.}
    \label{fig:cots_to_react}
    
    


\end{minipage}%
\vspace{-10pt}
\end{table}



\begin{table}[t]
\scriptsize
\begin{minipage}{1.0\linewidth}
    \centering
\begin{tabular}{l|clll}
\toprule
    & Type & Definition & \model & \reason \\
\midrule
\multirow{2}{*}{Success} & True positive & Correct reasoning trace and facts & 94\% & 86\% \\ & False positive & Hallucinated reasoning trace or facts & 6\% & 14\%\\
    \hline
    \multirow{4}{*}{Failure} & Reasoning error & Wrong reasoning trace (including failing to recover from repetitive steps) & 47\% & 16\% \\ & Search result error & Search return empty or does not contain useful information & 23\% & - \\ & Hallucination & Hallucinated reasoning trace or facts & 0\% & 56\% \\ & Label ambiguity & Right prediction but did not match the label precisely & 29\% & 28\%\\
\bottomrule
\end{tabular}
\caption{
Types of success and failure modes of \model{} and \reason{} on HotpotQA, as well as their percentages in randomly selected examples studied by human. %
}
\label{table:human_study_categories}
\end{minipage}%
\vspace{-10pt}
\end{table}



\myparagraph{\model{} vs. \reason{}} 
On the other hand, \model{} outperforms \reason{} on Fever (60.9 vs.\,56.3) and slightly lags behind \reason{} on HotpotQA (27.4 vs.\,29.4). 
Fever claims for SUPPORTS/REFUTES might only differ by a slight amount (see Appendix~\ref{sec:fever_trajs}), so acting to retrieve accurate and up-to-date knowledge is vital. 
To better understand the behavioral difference between \model{} and \reason{} on HotpotQA, we randomly sampled 50 trajectories with correct and incorrect answers (judged by EM) from \model{} and \reason{} respectively (thus 200 examples in total), and manually labeled their success and failure modes in Table~\ref{table:human_study_categories}. 
Some key observations are as follows:

\quad A) \textbf{Hallucination is a serious problem for \reason}, resulting in much higher false positive rate than \model{} (14\% vs. 6\%) in success mode, and make up its major failure mode (56\%).
In contrast, the problem solving trajectory of \model is more grounded, fact-driven, and trustworthy, thanks to the access of an external knowledge base.

\quad B) \textbf{While interleaving reasoning, action and observation steps improves \model's groundedness and trustworthiness, such a structural constraint also reduces its flexibility in formulating reasoning steps}, leading to more reasoning error rate than \reason. 
we note that there is one frequent error pattern specific to \model, in which the model repetitively generates the previous thoughts and actions, and we categorize it as part of ``reasoning error'' as the model fails to reason about what the proper next action to take and jump out of the loop\footnote{We suspect that this could be due to the sub-optimal greedy decoding procedure, and future work using better decoding (e.g.\,beam search) might help address this issue.}. 

\quad C) \textbf{For \model, successfully retrieving informative knowledge via search is critical.} Non-informative search, which counts for 23\% of the error cases, derails the model reasoning and gives it a hard time to recover and reformulate thoughts. This is perhaps an expected trade-off between factuality and flexibility, which motivates our proposed strategies of combining two methods.


We provide examples for each success and failure modes in Appendix \ref{sec:human_study_examples}. We also find some HotpotQA questions may contain outdated answer labels, see Figure~\ref{fig:date} for example.

\myparagraph{\model{} + \reason{}-SC perform best for prompting LLMs} Also shown in Table~\ref{table:reasoning}, the best prompting method on HotpotQA and Fever are \model{} $\to$ \reasons{} and \reasons{}  $\to$  \model{} respectively. Furthermore, Figure~\ref{fig:cots_to_react} shows how different methods perform with respect to the number of \reasons{} samples used. While two \model{} + \reasons{} methods are advantageous at one task each, they both significantly and consistently outperform \reasons{} across different number of samples, reaching \reasons{} performance with 21 samples using merely 3-5 samples. These results indicate the value of properly combining model internal knowledge and external knowledge for reasoning tasks. 

\begin{figure}[t]
    \centering
\includegraphics[width=.76\textwidth]{iclr2023/figure/hotpot_finetune.pdf}
\vspace{-5pt}
    \caption{Scaling results for prompting and finetuning on HotPotQA with \model{} (ours) and  baselines.}
    \label{fig:finetune}
    \vspace{-12pt}
\end{figure}



\myparagraph{\model{} performs best for fine-tuning} Figure~\ref{fig:finetune} shows the scaling effect of prompting/finetuning four methods (\palm{}, \reason{}, \act{}, \model{}) on HotpotQA. With PaLM-8/62B, prompting \model{} performs worst among four methods due to the difficulty to learn both reasoning and acting from in-context examples. However, when finetuned with just 3,000 examples, \model{} becomes the best method among the four, with PaLM-8B finetuned \model{} outperforming all PaLM-62B prompting methods, and PaLM-62B finetuned \model{} outperforming all 540B prompting methods. In contrast, finetuning \palm{} or \reason{} is significantly worse than finetuning \model{} or \act{} for both PaLM-8/62B, as the former essentially teaches models to memorize (potentially halluincated) knowledge facts, and the latter teaches models how to (reason and) act to access information from Wikipedia, a more generalizable skill for knowledge reasoning. 
As all prompting methods are still significantly far from domain-specific state-of-the-art approaches (Table~\ref{table:reasoning}), we believe finetuning with more human-written data might be a better way to unleash the power of \model{}. 






\section{Decision Making Tasks}
\label{decision_making_tasks}

We also test \model{} on two language-based interactive decision-making tasks, ALFWorld and WebShop,
both of which feature complex environments that require agents to act over long horizons with sparse rewards, warranting the need for reasoning to act and explore effectively.



\myparagraph{ALFWorld}
ALFWorld~\citep{shridhar2020alfworld} (Figure~\ref{fig:teaser}(2)) is a synthetic text-based game designed to align with the embodied ALFRED benchmark~\citep{shridhar2020alfred}. It includes 6 types of tasks in which an agent needs to achieve a high-level goal (e.g.\,examine paper under desklamp) by navigating and interacting with a simulated household via text actions (e.g.\,
go to coffeetable 1, take paper 2, use desklamp 1).
A task instance can have more than 50 locations and take an expert policy more than 50 steps to solve, thus challenging an agent to plan and track subgoals, as well as explore systematically (e.g.\,check all desks one by one for desklamp). In particular, one challenge built into ALFWorld is the need to determine likely locations for common household items (e.g.\,desklamps will likely be on desks, shelfs, or dressers), making this environment a good fit for LLMs to exploit their pretrained commonsense knowledge.
To prompt \model{}, we randomly annotate three trajectories from the training set for each task type, where each trajectory includes sparse thoughts that (1) decompose the goal, (2) track subgoal completion, (3) determine the next subgoal, and (4) reason via commonsense where to find an object and what to do with it.
We show prompts used for ALFWorld in Appendix~\ref{appendix:ALFWorld_prompts}.
Following ~\citet{shridhar2020alfworld}, we evaluate on 134 unseen evaluation games in a task-specific setup. For robustness, we construct 6 prompts for each task type through each permutation of 2 annotated trajectories from the 3 we annotate. \act{} prompts are constructed using the same trajectories, but without thoughts --- since task instances are randomly chosen from the training set, it favors neither \model{} nor \act{} and provides a fair and controlled comparison to test the importance of sparse thoughts. 
For baselines, we use BUTLER~\citep{shridhar2020alfworld}, an imitation learning agent trained on $10^5$ expert trajectories for each task type\footnote{\citet{micheli2021language} finetuned a GPT-2 model on 3553 task instances and achieved a much improved performance than BUTLER, but it is trained on all task types, thus not included as a baseline.}. 














\myparagraph{WebShop}
\label{sec:webshop}
Can \model{} also interact with noisy real-world language environments for practical applications? 
We investigate WebShop~\citep{yao2022webshop}, a recently proposed online shopping website environment with 1.18M real-world products and 12k human instructions. Unlike ALFWorld, Webshop contains a high variety of structured and unstructured texts (e.g.\,product titles, descriptions, and options crawled from Amazon), and requires an agent to purchase a product based on a user instruction (e.g.\,``I am looking for a nightstand
with drawers. It should have a nickel finish, and priced lower than \$140'') through web interactions (e.g.\,search ``nightstand drawers'', choose buttons such as ``color: modern-nickel-white'' or ``back to search''). 
This task is evaluated by average score (percentage of desired attributes covered by the chosen product averaged across all episodes) and success rate (percentage of episodes where the chosen product satisfies all requirements) on 500 test instructions. 
We formulate \act{} prompts with actions to search, choose product, choose options, and buy, with \model{} prompts additionally reasoning to determine what to explore, when to buy, and what products options are relevant to the instruction.
See Table~\ref{prompts:webshop} for an example prompt, and Table~\ref{trajectories:webshop} for model predictions in the Appendix.
We compare to an imitation learning (IL) method trained with 1,012 human annotated trajectories, and a imitation + reinforcement learning (IL + RL) method additionally trained with 10,587 training instructions.











\myparagraph{Results}






\begin{table}[t]
\begin{minipage}{.69\linewidth}
    \centering


\resizebox{\columnwidth}{!}{%
\begin{tabular}{l | cccccc | c}
\toprule
Method  & Pick & Clean & Heat & Cool & Look & Pick 2 & All \\ \midrule
\act{} {\tiny(best of 6)} & 88 & 42 & 74 & 67 & 72 & \textbf{41} & 45 \\
\model{} {\tiny(avg)} & 65 & 39 & 83 & 76 & 55 & 24 & 57 \\ 
\model{} {\tiny(best of 6)}  & \textbf{92} & 58 & \textbf{96} & 86 & \textbf{78} & \textbf{41} & \textbf{71} \\
\midrule
\modelim{}  {\tiny(avg)}       & 55 & 59 & 60 & 55 & 23 & 24 & 48 \\
\modelim{} {\tiny(best of 6)}  & 62 & \textbf{68} & 87 & 57 & 39 & 33 & 53 \\ 
\midrule
BUTLER$_g$ {\tiny(best of 8)}  & 33 & 26 & 70 & 76 & 17 & 12 & 22 \\
BUTLER {\tiny(best of 8)}  & 46 & 39 & 74 & \textbf{100} & 22 & 24 & 37 \\
\bottomrule
\end{tabular}%
}
\caption{
    AlfWorld task-specific success rates (\%). 
    BUTLER and BUTLER$_g$ results are from Table 4 of~\cite{shridhar2020alfworld}.
    All methods use greedy decoding, except that BUTLER uses beam search.
}
\label{table:alfworld}


\end{minipage}%
\hspace{5pt}
\begin{minipage}{.23\linewidth}
    \centering
\resizebox{\columnwidth}{!}{%
\begin{tabular}{c | cc}
\toprule
Method & Score & SR \\ \midrule
\act{} & 62.3 & 30.1 \\
\model{} & \textbf{66.6} & \textbf{40.0} \\ \midrule
IL & 59.9 & 29.1 \\ 
IL+RL & 62.4 & 28.7 \\ \midrule \midrule
Human & \multirow{2}{*}{82.1}  & \multirow{2}{*}{59.6} \\
Expert & & \\
\bottomrule
\end{tabular}%
}
\caption{
    Score and success rate (SR) on Webshop. IL/IL+RL taken from \cite{yao2022webshop}. 
}
\label{table:webshop}
\end{minipage}%
\vspace{-13pt}
\end{table}




\model{} outperforms \act{} on both ALFWorld (Table~\ref{table:alfworld}) and Webshop (Table \ref{table:webshop}). On ALFWorld, the best \model{} trial achieves an average success rate of 71\%, significantly outperforming the best \act{} (45\%) and BUTLER (37\%) trials. In fact, even the worse \model{} trial (48\%) beats the best trial of both methods. Moreover, the advantage of \model{} over \act{} is consistent across six controlled trials, with relative performance gain ranging from 33\% to 90\% and averaging 62\%. Qualitatively, we saw that, without any thoughts at all, \act{} fails to correctly decompose goals into smaller subgoals, or loses track of the current state of the environment. Example trajectories comparing \model{} and \act{} can be found in Appendix~\ref{appendix:react_ALFWorld_trajectory} and Appendix~\ref{appendix:act_ALFWorld_trajectory}. %

On Webshop, one-shot \act{} prompting already performs on par with IL and IL+RL methods. With additional sparse reasoning, \model{} achieves significantly better performance, with an absolute 10\% improvement over the  previous best success rate. 
By checking examples, we find that \model{} is more likely to identify instruction-relevant products and options by reasoning to bridge the gap between noisy observations and actions (e.g.\,``For `space-saving ottoman bench for living room', the item has options `39x18x18inch' and `blue' and seems good to buy.'').
However, existing methods are still far from the performance of expert humans (Table~\ref{table:webshop}), who perform significantly more product explorations and query re-formulations that are still challenging for prompting-based methods.

\myparagraph{On the value of internal reasoning vs. external feedback}

To our knowledge, \model{} is the first demonstration of combined reasoning and action using an LLM applied to an interactive environment within a closed-loop system. Perhaps the closest prior work is Inner Monologue (IM), from \cite{huang2022inner}, in which actions from an embodied agent are motivated by an eponymous ``inner monologue''. \textbf{However, IM's ``inner monologue'' 
is limited to observations of the environment state and what needs to be completed by the agent for the goal to be satisfied.}
In contrast, the reasoning traces in \model{} for decision making is flexible and sparse, allowing diverse reasoning types (see Section~\ref{sec:react}) to be induced for different tasks.

To demonstrate the differences between \model{} and IM, and to highlight the importance of internal reasoning vs. simple reactions to external feedback, we ran an ablation experiment using a thought pattern composed of IM-like dense external feedback. As can be seen in Table~\ref{table:alfworld}, \model{} substantially outperforms IM-style prompting (\modelim{}) (71 vs.\,53 overall success rate), with consistent advantages on five out of six tasks. 
Qualitatively, we observed that \modelim{} often made mistakes in identifying when subgoals were finished, or what the next subgoal should be, due to a lack of high-level goal decomposition. Additionally, many \modelim{} trajectories struggled to determine where an item would likely be within the ALFWorld environment, due to a lack of commonsense reasoning. Both shortcomings can be addressed in the \model{} paradigm. 
More details about \modelim{} is in Appendix~\ref{sec:alfworld_im}. An example prompt for \modelim{} can be found in Appendix~\ref{appendix:ALFWorld_prompts}, and an example trajectory in Appendix~\ref{appendix:reactim_ALFWorld_trajectory}.
%auto-ignore

\section{Related Work}
There is a long history of pre-training general language representations, and we briefly review the most widely-used approaches in this section.


\subsection{Unsupervised Feature-based Approaches}
Learning widely applicable representations of words has been an active area of research for decades, including non-neural~\cite{brown-etal:1992:_class, ando-zhang:2005, blitzer-mcdonald-pereira:2006:_domain} and neural~\cite{mikolov-etal:2013, pennington-socher-manning:2014:_glove} methods. Pre-trained word embeddings are an integral part of modern NLP systems, offering significant improvements over embeddings learned from scratch~\cite{turian-ratinov-bengio:2010:_word_repres}. To pre-train word embedding vectors, left-to-right language modeling objectives have been used~\cite{minh09}, as well as objectives to  discriminate correct from incorrect words in left and right context~\cite{mikolov-etal:2013}.

These approaches have been generalized to coarser granularities, such as sentence embeddings~\cite{kiros-etal:2015:_skip, logeswaran2018an} or paragraph embeddings~\cite{le-mikolov:2014:_distr}. To train sentence representations, prior work has used objectives to rank candidate next sentences  \cite{DBLP:journals/corr/JerniteBS17, logeswaran2018an},  left-to-right generation of next sentence words given a representation of the previous sentence~\cite{kiros-etal:2015:_skip}, or denoising auto-encoder derived objectives~\cite{hill16}.

%
\begin{figure*}[t!]
%\small
\begin{center}
\includegraphics[width=1\textwidth]{BERT_Overall.pdf}
\end{center}
\caption{Overall pre-training and fine-tuning procedures for BERT. Apart from output layers, the same architectures are used in both pre-training and fine-tuning. The same pre-trained model parameters are used to initialize models for different down-stream tasks.  During fine-tuning, all parameters are fine-tuned. {\tt [CLS]} is a special symbol added
in front of every input example, and {\tt [SEP]} is a special separator token (e.g. separating  questions/answers).}
\label{fig:bert_overall}
\end{figure*}
%



ELMo and its predecessor~\cite{peters-etal:2017:_semi, peters-etal:2018:_deep} generalize traditional word embedding research along a different dimension. They extract \emph{context-sensitive} features from a left-to-right and a right-to-left language model. The contextual representation of each token is the concatenation of the left-to-right and right-to-left representations. When integrating contextual word embeddings with existing task-specific architectures, ELMo advances the state of the art for several major NLP benchmarks~\cite{peters-etal:2018:_deep} including question answering~\cite{rajpurkar-etal:2016:_squad}, sentiment analysis~\cite{socher-etal:2013:_recur}, and named entity recognition~\cite{tjong-de:2003}.
\citet{melamud2016context2vec} proposed learning contextual representations through a task to predict a single word from both left and right context
using LSTMs. Similar to ELMo, their model is feature-based and not deeply bidirectional. 
\citet{fedus2018maskgan} shows that the cloze task can be used to improve the robustness of text generation models. 




\subsection{Unsupervised Fine-tuning Approaches}

As with the feature-based approaches, the first works in this direction only pre-trained word embedding parameters from unlabeled text ~\cite{collobert-weston:2008}.  

More recently, sentence or document encoders which produce contextual token representations have been pre-trained from unlabeled text and fine-tuned for a supervised downstream task~\cite{dai-le:2015:_semi, howard-ruder:2018, radford-etal:2018}.  The advantage of these approaches is that few parameters need to be learned from scratch. At least partly due to this advantage, OpenAI GPT~\cite{radford-etal:2018} achieved previously state-of-the-art results on many sentence-level tasks from the GLUE benchmark~\cite{wang-etal:2018:_glue}.  Left-to-right language modeling and auto-encoder objectives have been used for pre-training such models~\cite{howard-ruder:2018, radford-etal:2018,dai-le:2015:_semi}.





\subsection{Transfer Learning from Supervised Data}

There has also been work showing effective transfer from supervised tasks with large datasets, such as natural language inference~\cite{conneau-EtAl:2017:EMNLP2017} and machine translation~\cite{mccann-etal:2017:_learn_trans}. 
Computer vision research has also demonstrated the importance of transfer learning from large pre-trained models, where an effective recipe is to fine-tune models pre-trained with ImageNet~\cite{imagenet_cvpr09, yosinski2014transferable}. 




\section{Conclusion}
We have proposed \model{} -- a simple yet effective method for synergizing reasoning and acting in large language models. Through a diverse set of experiments on multi-hop question-answering, fact checking, and interactive decision-making tasks, we show that \model{} leads to superior performance with interpretable decision traces. Despite the simplicity of our method, complex tasks with large action spaces require more demonstrations to learn well, which unfortunately can easily go beyond the input length limit of in-context learning. We explore the fine-tuning approach on HotpotQA with initial promising results, but learning from more high-quality human annotations will be the desiderata to further improve the performance. Scaling up \model{} with multi-task training and combining it with complementary paradigms like reinforcement learning could result in stronger agents that further unlock the potential of LLMs for more applications.




\subsubsection*{Acknowledgments}
We thank the support and feedback of many people from Google Brain team and Princeton NLP Group.
This work was supported in part by the National Science Foundation under Grant No. 2107048. Any opinions, findings, and conclusions or recommendations expressed in this material are those of the author(s) and do not necessarily reflect the views of the National Science Foundation.

{
\subsubsection*{Reproducibility Statement}
Our main experiments are done on PaLM~\citep{chowdhery2022palm}, which is not an openly accessible model yet. To increase reproducibility, we have included all used prompts in Appendix~\ref{sec:prompts}, additional experiments using GPT-3~\citep{brown2020language} in Appendix~\ref{sec:gpt3}, and associated GPT-3 \model{} prompting code at \url{https://anonymous.4open.science/r/ReAct-2268/}. 
\subsubsection*{Ethics Statement}
\model{} prompts large language models to generate more human interpretable, diagnosable, and controllable task-solving trajectories than previous methods. 
However, hooking up a large language model with an action space to interact with external environments (e.g.\,the web, physical environments) has potential dangers, e.g.\, looking up inappropriate or private information, or taking harmful actions in an environment. 
Our experiments minimize such risks by limiting the interactions to specific websites (Wikipedia or WebShop) that are free of private information, without any dangerous actions in the action space design
(i.e.\,models cannot really buy products on WebShop the research benchmark, or edit Wikipedia).
We believe researchers should be aware of such risks before designing more extensive experiments in the future.
}




\bibliography{iclr2023_conference}
\bibliographystyle{iclr2023_conference}

\newpage

\appendix

\section{Comparison of Reported and Reproduced Performance}
Detailed comparison scores of reported and reproduced performance are shown in \Cref{tab:performance_comparison}.

\begin{table}[ht!]
    \centering
    \small
    \begin{tabular}{lcc}
        \toprule
        {\textbf{Method}} & Reported & Reproduced \\
        \midrule
         GPT 3.5 Turbo 0613 + CoT & 41.5 & 35.2 \textcolor{red}{{\tiny 
         -32.5\%}} \\
         GPT 3.5 Turbo 0613 + DFS & 54.5 & 53.2 \textcolor{red}{{\tiny 
         -2.4\%}}\\
         ToolLLaMA v2 + CoT & 25.0 & 15.0 \textcolor{red}{{\tiny 
         -40\%}} \\
         ToolLLaMA v2 + DFS & 57.0 & 34.0 \textcolor{red}{{\tiny 
         -40.4\%}}\\
         \bottomrule
    \end{tabular}
    \caption{Comparison of performance (Pass Rate) reported in the paper and reproduced by us of ChatGPT and ToolLLaMA v2 on the I1-Instruction group of ToolBench. }
    \label{tab:performance_comparison}
\end{table}

\section{Statistics of API change information}
Detailed statistics of API change categories and information are shown in \Cref{tab:api_change} and \Cref{tab:api_not_available}.

\begin{table}[h!]
    \centering
    \small
    \begin{tabular}{lcc}
     \toprule
    \textbf{Status Type} & \textbf{Number} & \textbf{Percentage} (\%) \\
    \midrule
    % Not Available & 7504 & 45.6 \\
    Not Available & 8095 & 49.2 \\
    % \quad-- Not Connectable & 2426 & 14.7 \\
    % \quad-- Not Found & 583 & 3.5 \\
    % \quad-- Parameter Issues & 591 & 3.6 \\
    % \quad-- Other issues & 4495 & 27.3 \\
    Not Authorised & 1058 & 6.4 \\
    % Parameter Change & 591 & 3.6 \\
    Success & 7311 & 44.4 \\
     \bottomrule
    \end{tabular}
    \caption{APIs changed in ToolBench.}
    \label{tab:api_change}
\end{table}



\begin{table}[h!]
    \centering
    \small
    \begin{tabular}{lcc}
     \toprule
    \textbf{Status Type} & \textbf{Number} & \textbf{Percentage} (\%) \\
    \midrule
    Not Connectable & 2426 & 30.0 \\
    Not Found & 583 & 7.2 \\
    Parameter Change & 591 & 7.3 \\
    Parsing Error & 4247 & 52.6 \\
    Other & 248 & 3.1 \\
    \midrule
    Total & 8095 & 100 \\
     \bottomrule
    \end{tabular}
    \caption{Categories of Not Availability in ToolBench.}
    \label{tab:api_not_available}
\end{table}


% \section{Stability Test Scores with Virtual API Systems}
% Detailed stability test scores are shown in \Cref{tab:real_api_stability_test}.

% \begin{table}[]
%     \centering
%     \small
%     \resizebox{\linewidth}{!}{
%     \begin{tabular}{lcccc}
%         \toprule
%         \multirow{2}{*}{\textbf{Method}} & \multicolumn{4}{c}{\textbf{Percentage of Failing Tools}}\\
%         % \cmidrule{2-5}
%           & 0\% & 10\% & 20\% & 50\% \\
%         \midrule
%          GPT 3.5 Turbo 0613 + CoT & 20.3{\tiny $\pm{0.8}$}  & 17.9{\tiny $\pm{1.2}$} & 16.3{\tiny $\pm{0.6}$} & 12.8{\tiny $\pm{1.7}$} \\
%          GPT 3.5 Turbo 0613 + DFS & 26.6{\tiny $\pm{0.3}$} & 23.9{\tiny $\pm{1.1}$} & 23.2{\tiny $\pm{1.0}$} & 16.3{\tiny $\pm{1.2}$}\\
%          GPT 4 0613 + CoT & 21.4{\tiny $\pm{0.5}$}  & 19.6{\tiny $\pm{0.9}$} & 15.5{\tiny $\pm{0.3}$} & 11.8{\tiny $\pm{0.8}$} \\
%          GPT 4 0613 + DFS &  24.2{\tiny $\pm{1.8}$}  & 24.0{\tiny $\pm{0.8}$} & 21.2{\tiny $\pm{1.8}$} & 16.9{\tiny $\pm{0.4}$}\\
%          \bottomrule
%     \end{tabular}
%     }
%     \caption{SoPR change when manually make APIs down on the I1 Instruction group.}
%     \label{tab:real_api_stability_test}
% \end{table}
\section{Stability Test Scores with Virtual API Systems}
\label{app:detailed_stability_test_virtual}
Detailed scores of stability tests of various models are shown in \Cref{tab:simulated_api_stability_test}. Note that in addition to GPT 3.5 Turbo 0613 and GPT 4 0613, we report the performance of newer versions, namely GPT 3.5 Turbo 1106 and GPT 4 Turbo Preview.
\begin{table}[]
    \centering
    \small
    \resizebox{\linewidth}{!}{
    \begin{tabular}{ccccc}
        \toprule
        \multirow{2}{*}{\textbf{Method}} & \multicolumn{4}{c}{\textbf{Real API Failure Rate}}\\
        \cmidrule{2-5}
          & 0\% & 10\% & 20\% & 50\% \\
        \midrule
         GPT 3.5 Turbo 0613 + CoT & 49.1{\tiny $\pm{1.0}$}  & 48.7{\tiny $\pm{0.9}$} & 51.2{\tiny $\pm{1.3}$} & 49.0{\tiny $\pm{0.7}$} \\
         GPT 3.5 Turbo 0613 + DFS & 68.1{\tiny $\pm{1.4}$}  & 70.9{\tiny $\pm{1.3}$} & 67.5{\tiny $\pm{1.8}$} & 67.3{\tiny $\pm{1.3}$}\\
         GPT 4 0613 + CoT & 55.4{\tiny $\pm{0.6}$}  & 55.5{\tiny $\pm{1.0}$} & 58.0{\tiny $\pm{0.5}$} & 55.2{\tiny $\pm{0.6}$} \\
         GPT 4 0613 + DFS & 69.7{\tiny $\pm{1.4}$}  & 71.4{\tiny $\pm{1.4}$} & 71.2{\tiny $\pm{0.9}$} & 69.9{\tiny $\pm{0.9}$}\\
         \midrule
         GPT 3.5 Turbo 1106 + CoT & 52.1{\tiny $\pm{0.7}$}  & 52.4{\tiny $\pm{0.8}$} & 53.9{\tiny $\pm{0.6}$} & 50.2{\tiny $\pm{0.6}$} \\
         GPT 3.5 Turbo 1106 + DFS & 69.9{\tiny $\pm{0.7}$}  & 71.7{\tiny $\pm{0.7}$} & 69.4{\tiny $\pm{0.8}$} & 71.6{\tiny $\pm{0.9}$}\\
         GPT 4 Turbo preview + CoT & 60.8{\tiny $\pm{0.7}$}  & 62.8{\tiny $\pm{0.5}$} & 64.2{\tiny $\pm{0.7}$} & 62.4{\tiny $\pm{0.5}$} \\
         GPT 4 Turbo preview + DFS& 73.2{\tiny $\pm{1.1}$}  & 76.7{\tiny $\pm{1.0}$} & 76.0{\tiny $\pm{0.8}$} & 74.2{\tiny $\pm{1.3}$}\\     
         \bottomrule
    \end{tabular}
    }
    \caption{Performance change when manually make APIs down with our virtual online API system. The results are averaged over all six groups. Solving rates are reported. We run each experiment one time and evaluate three times and take the average score.}
    \label{tab:simulated_api_stability_test}
\end{table}




\section{Call Error Identification and Cache Filtering Rule}\label{app:filter_rule}
We identify call errors and filter out invalid call to RapidAPI based on keyword occurences. In detail, we identify the following error:
\begin{itemize}
    \item Not Connected Error: when error information contains \texttt{HTTP} or the response infomation contains \texttt{HTTP error, connection, rate limit, time(d) out};
    \item Not Found Error: when the error information or response contains \texttt{not found, not available, API doesn't exists, Service Not Found, internal error} or 404 error message;
    \item Parameter Change: when the error information or response contains \texttt{parameter, parse, is not defined};
    \item Parsing Error: when the error information starts with \texttt{Function executing from};
    \item Not Authorised: when the error information or response contains \texttt{authoriz(s), unauthoriz(s), blocked user, unsubscribe, credential, disabled for your subscription, ACCESS\_DENIED} or 401, 403 error message;
    \item Other Errors: messages with non-empty error messages;
    \item Success: Other calls.
\end{itemize}
We consider all types of errors when identifying errors. However, when filtering the cache, we do not conside the``Other Errors''.

\begin{table*}[ht!]
    % \small
    \centering
    % \resizebox{\columnwidth}{!}{
    \begin{tabular}{p{0.1\textwidth}p{0.8\textwidth}}
    \toprule
    \rowcolor[gray]{0.95} 
    \multicolumn{2}{c}{\textbf{API Simulation Prompt}} \\
    \midrule
    System & \makecell[{{p{.8\textwidth}}}]{
    Imagine you are an API Server operating within a specialized tool, which contains a collection of distinct APIs. Your role is to deeply understand the function of each API based on their descriptions in the API documentation. As you receive specific inputs for individual API calls within this tool, analyze these inputs to determine their intended purpose. Your task is to craft a JSON formatted response that aligns with the expected output of the API, guided by the provided examples. \\
    Your responses must adhere to a specific JSON structure, which is as follows: \\
    \texttt{\{
        ``error'': ``'',
        ``response'': ``Your\_Response''
    \}}\\
The error field should remain empty, indicating no errors in processing. The response field should contain the content you formulate based on the API's functionality and the input provided. Ensure that your responses are meaningful, directly addressing the API's intended functionality. If the provided examples are mostly error messages or lack substantial content, use your judgment to create relevant and accurate responses. The key is to maintain the JSON format's integrity while ensuring that your response is an accurate reflection of the API's intended output within the tool.\\
Please note that your answer should not contain anything other than a json format object, which should be parsable directly to json. \\
Note that: \\
- your response should be around 100 to 200 words, containing rich information given the api input parameters. Keep Your answer short and simple.\\
- your response must be effective and have practical content.\\
- if the api response example if null or ineffective, ignore the example and give your independent response. \\
    } \\
    \hline
    User & \makecell[{{p{.85\linewidth}}}]{
    API Documentation:\\
    \texttt{Documentation JSON file}\\
    API Examples: \\
    \texttt{Example input 1: Example response 1}\\
    \texttt{Example input 2: Example response 2}\\
    \texttt{Example input 3: Example response 3}\\
    API input:\\
    \texttt{Argument JSON string, e.g:} \\
    \texttt{\{``category'':``Logistics'',}\texttt{``tool\_name'': ``SQUAKE'',}\\
    \texttt{``api\_name'': ``Checkhealth'',``tool\_input'': ``\{\}'',}\\
    \texttt{``strip'': ``filter''\}}
    } \\
    \bottomrule
    \end{tabular}
    \caption{Prompt used to simulate APIs.}
    \label{tab:prompt_simulate_api}
\end{table*}

\begin{table*}[ht!]
    % \small
    \centering
    % \resizebox{\columnwidth}{!}{
    \begin{tabular}{l}
    \toprule
    \rowcolor[gray]{0.95} 
    \textbf{Solvable Task Filtration Prompt} \\
    \midrule
    \makecell[l{p{\textwidth}}]{
    Please check whether the given task solvable with following rules:\\
    1. If the \texttt{query} provide invalid information (e.g. invalid email address or phone number), return \texttt{Unsolvable}\\
    2. If the \texttt{query} needs more information to solve (e.g. the target restaurant name in a navigation task), return \texttt{Unsolvable} \\
    3. If the current \texttt{available\_tools} are enough to solve the query, return \texttt{Solvable} \\
    4. Return only \texttt{Solvable} or \texttt{Unsolvable} \\
    \\
    Task:\{\texttt{task}\}
    \\
    Now please give your answer (only \texttt{Solvable} or \texttt{Unsolvable}):
}
    \\
    \bottomrule
    \end{tabular}
    \caption{Prompt used to filter solvable tasks.}
    \label{tab:task_solvability}
\end{table*}

\section{Configurations of API Diversity Analysis }\label{app:diversity_conf}
The configurations of diversity analysis are as follows:
\begin{itemize}
    \item Embedding model: \texttt{all-mpnet-base-v2};
    \item UMAP metric (distance metric): correlation;
    \item Num of neighbours: 15;
    \item Min distance: 0.5.
\end{itemize}


\section{Prompts of API simulation}
\label{app:prompt_simulation}


The prompt used to simulate API behaviours is shown in \Cref{tab:prompt_simulate_api}.


\section{Prompt to Filter Solvable Task}
\label{app:prompt_task_solvability}
The prompt used to filter solvable tasks is shown in \Cref{tab:task_solvability}.




\section{Prompt Used to Make API Calls}\label{app:prompt_make_call}
The prompt used to construct API calls to scan availables is shown in \Cref{tab:prompt_api_call}.

\begin{table*}[t!]
    % \small
    \centering
    % \resizebox{\columnwidth}{!}{
    \begin{tabular}{p{0.1\textwidth}p{0.8\textwidth}}
    \toprule
    \rowcolor[gray]{0.95} 
    \multicolumn{2}{c}{\textbf{API Call Writing Prompt}} \\
    \midrule
    System & \makecell[{{p{.8\textwidth}}}]{
Imagine you are an API requester, Your role is to deeply understand the function of each API based on their descriptions in the API documentation.  Your task is to craft a JSON formatted input that aligns with the expected input of the API, guided by the provided examples.\\
Your responses must adhere to a specific JSON structure, which is as follows:\\
Please note that your answer should not contain anything other than a json format object, which should be parsable directly to json. \\
Note that:\\
- your response should be around 100 to 500 words, containing rich information given the api input parameters.\\
- your response must be effective and have practical content.\\
- if the api response example if null or ineffective, ignore the example and give your independent response.\\
    } \\
    \hline
    User & \makecell[{{p{.85\linewidth}}}]{
    API Documentation:\\
    \texttt{Documentation JSON file}\\
    API Examples (if available): \\
    \texttt{Example input 1: Example response 1}\\
    \texttt{Example input 2: Example response 2}\\
    \texttt{Example input 3: Example response 3}\\
    one more API Input example:\\
    } \\
    \bottomrule
    \end{tabular}
    \caption{Prompt used to write API calls.}
    \label{tab:prompt_api_call}
\end{table*}






\end{document}
