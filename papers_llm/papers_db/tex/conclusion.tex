\section{Conclusion}
\label{sec:conclusion}

We have introduced a novel approach that extracts global and local contextual information using attention mechanisms for instance-level image retrieval. It is manifested as a network architecture consisting of global and local attention components, each operating on both spatial and channel dimensions. This constitutes a comprehensive study and empirical evaluation of all four forms of attention that have previously been studied only in isolation. Our findings indicate that the gain (or loss) brought by one form of attention alone strongly depends on the presence of the others, with the maximum gain appearing when all forms are present. The output is a modified feature tensor that can be used in any way, for instance with local feature detection instead of spatial pooling for image retrieval.

With the advent of \emph{vision transformers}~\cite{dosovitskiy2020image,2101.11986} and their recent application to image retrieval~\cite{2102.05644}, attention is expected to play a more and more significant role in vision. According to our classification, transformers perform global spatial attention alone. It is of great interest to investigate the role of the other forms of attention, where our approach may yield a basic building block of such architectures. One may even envision an extension to language models, where transformers originate from~\cite{VSP+17}.
