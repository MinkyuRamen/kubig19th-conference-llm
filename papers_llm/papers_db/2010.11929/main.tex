
\documentclass{article} %
\usepackage{iclr2021_conference,times}

\input{math_commands.tex}

\usepackage{hyperref}
\usepackage{url}

\usepackage[pdftex]{graphicx} 
\usepackage{booktabs}
\usepackage{xspace}
\usepackage{xcolor}
\usepackage{wrapfig}
\usepackage{xfrac}
\usepackage{tabularx}
\usepackage{tikz}

\definecolor{natural}{rgb}{0.7137,0.3333,0.3333}
\definecolor{specialized}{rgb}{0.4118,0.6431,0.4314}
\definecolor{structured}{rgb}{0.3254,0.4431,0.6666}
\definecolor{all}{rgb}{0.7529,0.4902,0.6471}

\definecolor{alexey}{rgb}{0.8, 0.0, 0.8}
\newcommand{\alexey}[1]{\textcolor{alexey}{Alexey: #1}}
\definecolor{matthias}{rgb}{0.0, 0.8, 0.8}
\newcommand{\matthias}[1]{\textcolor{matthias}{Matthias: #1}}
\definecolor{sylvain}{rgb}{0.8, 0.8, 0.0}
\newcommand{\sylvain}[1]{\textcolor{sylvain}{Sylvain: #1}}
\newcommand{\TODO}[1]{{\color{blue}[\textbf{TODO:}~#1]}}
\newcommand{\oursabbrv}{ViT\xspace}
\newcommand{\oursfull}{Vision Transformer\xspace}
\newcommand{\imagenet}{ImageNet\xspace}
\newcommand{\valstd}[2]{$#1 {\scriptstyle \,\pm\, #2}$}
\newcommand{\valstdb}[2]{$\mbf{#1} {\scriptstyle \,\pm\, #2}$}


\title{An Image is Worth 16x16 Words:\\ Transformers for Image Recognition at Scale}


\author{
\centerline{Alexey Dosovitskiy$^{*,\dagger}$, Lucas Beyer$^{*}$, Alexander Kolesnikov$^{*}$, Dirk Weissenborn$^{*}$,} \vspace{1mm}\\
\centerline{\textbf{Xiaohua Zhai$^{*}$, Thomas Unterthiner, Mostafa Dehghani, Matthias Minderer,}} \vspace{0.5mm} \\
\centerline{\textbf{Georg Heigold, Sylvain Gelly, Jakob Uszkoreit, Neil Houlsby$^{*,\dagger}$}} \vspace{0.7mm} \\
\centerline{$^*$equal technical contribution, $^\dagger$equal advising}  \vspace{0.7mm} \\
\centerline{Google Research, Brain Team}  \vspace{0.8mm} \\
\centerline{\texttt{\{adosovitskiy, neilhoulsby\}@google.com}} \\
}


\newcommand{\fix}{\marginpar{FIX}}
\newcommand{\new}{\marginpar{NEW}}

\iclrfinalcopy %
\begin{document}


\maketitle

\begin{abstract}
While the Transformer architecture has become the de-facto standard for natural language processing tasks, its applications to computer vision remain limited.
In vision, attention is either applied in conjunction with convolutional networks, or used to replace certain components of convolutional networks while keeping their overall structure in place.
We show that this reliance on CNNs is not necessary and a pure transformer applied directly to sequences of image patches can perform very well on image classification tasks.
When pre-trained on large amounts of data and transferred to multiple mid-sized or small image recognition benchmarks (\imagenet, CIFAR-100, VTAB, etc.), \oursfull{} (\oursabbrv) attains excellent results compared to state-of-the-art convolutional networks while requiring substantially fewer computational resources to train.\footnote{Fine-tuning code and pre-trained models are available at \url{https://github.com/google-research/vision_transformer}}


\end{abstract}

\section{Introduction}


Self-attention-based architectures, in particular Transformers~\citep{vaswani2017}, have become the model of choice in natural language processing (NLP).
The dominant approach is to pre-train on a large text corpus and then fine-tune on a smaller task-specific dataset~\citep{devlin19-bert}.
Thanks to Transformers' computational efficiency and scalability, it has become possible to train models of unprecedented size, with over 100B parameters~\citep{brown2020-gpt3,lepikhin2020gshard}.
With the models and datasets growing, there is still no sign of saturating performance.


In computer vision, however, convolutional architectures remain dominant~\citep{LeCun1989BackpropagationAT,KrizhevskyNIPS12,he2016deep}.
Inspired by NLP successes, multiple works try combining CNN-like architectures with self-attention~\citep{wang2018-nonlocalnn,carion20-detr}, some replacing the convolutions entirely~\citep{ramachandran19-sasa,wang2020-axialdeeplab}.
The latter models, while theoretically efficient, have not yet been scaled effectively on modern hardware accelerators due to the use of specialized attention patterns.
Therefore, in large-scale image recognition, classic ResNet-like architectures are still state of the art~\citep{mahajan2018,xie2020-noisystudent,kolesnikov2020-bit}.

Inspired by the Transformer scaling successes in NLP, we experiment with applying a standard Transformer directly to images, with the fewest possible modifications.
To do so, we split an image into patches and provide the sequence of linear embeddings of these patches as an input to a Transformer.
Image patches are treated the same way as tokens (words) in an NLP application.
We train the model on image classification in supervised fashion.

When trained on mid-sized datasets such as \imagenet without strong regularization, these models yield modest accuracies of a few percentage points below ResNets of comparable size.
This seemingly discouraging outcome may be expected: Transformers lack some of the inductive biases inherent to CNNs, such as translation equivariance and locality, and therefore do not generalize well when trained on insufficient amounts of data.

However, the picture changes if the models are trained on larger datasets (14M-300M images). 
We find that large scale training trumps inductive bias.
Our \oursfull{} (\oursabbrv{}) attains excellent results when pre-trained at sufficient scale and transferred to tasks with fewer datapoints.
When pre-trained on the public ImageNet-21k dataset or the in-house JFT-300M dataset, \oursabbrv{} approaches or beats state of the art on multiple image recognition benchmarks.
In particular, the best model reaches the accuracy of $88.55\%$ on \imagenet, $90.72\%$ on \imagenet-ReaL, $94.55\%$ on CIFAR-100, and $77.63\%$ on the VTAB suite of 19 tasks.
















\section{Related Work}

Transformers were proposed by \citet{vaswani2017} for machine translation, and have since become the state of the art method in many NLP tasks. Large Transformer-based models are often pre-trained on large corpora and then fine-tuned for the task at hand: BERT~\citep{devlin19-bert} uses a denoising self-supervised pre-training task, while the GPT line of work uses language modeling as its pre-training task~\citep{radford2018-gpt,radford2019-gpt2,brown2020-gpt3}.

Naive application of self-attention to images would require that each pixel attends to every other pixel. With quadratic cost in the number of pixels, this does not scale to realistic input sizes.
Thus, to apply Transformers in the context of image processing, several approximations have been tried in the past.
\citet{parmar18-imagetransformer} applied the self-attention only in local neighborhoods for each query pixel instead of globally.
Such local multi-head dot-product self attention blocks can completely replace convolutions~\citep{hu2019local,ramachandran19-sasa,zhao2020-san}.
In a different line of work, Sparse Transformers~\citep{child2019-sparsetransformers} employ scalable approximations to global self-attention in order to be applicable to images.
An alternative way to scale attention is to apply it in blocks of varying sizes \citep{weissenborn2019-savm}, in the extreme case only along individual axes~\citep{ho2019-axialattention,wang2020-axialdeeplab}.
Many of these specialized attention architectures demonstrate promising results on computer vision tasks, but require complex engineering to be implemented efficiently on hardware accelerators.

Most related to ours is the model of~\citet{cordonnier2020-sacnn}, which extracts patches of size $2 \times 2$ from the input image and applies full self-attention on top. This model is very similar to ViT, but our work goes further to demonstrate that large scale pre-training makes vanilla transformers competitive with (or even better than) state-of-the-art CNNs. Moreover, \citet{cordonnier2020-sacnn} use a small patch size of $2 \times 2$ pixels, which makes the model applicable only to small-resolution images, while we handle medium-resolution images as well.

There has also been a lot of interest in combining convolutional neural networks (CNNs) with forms of self-attention, e.g. by augmenting feature maps for image classification~\citep{bello2019-attentionaugmentedcnn} or by further processing the output of a CNN using self-attention, e.g. for object detection~\citep{hu2018-relationnetworks, carion20-detr}, video processing~\citep{wang2018-nonlocalnn,sun2019-videobert}, image classification~\citep{wu2020-visualtransformer}, unsupervised object discovery~\citep{locatello2020-slotattention}, or unified text-vision tasks~\citep{chenuniter,vilbert,visualbert}.

Another recent related model is image GPT (iGPT)~\citep{chen20-igpt}, which applies Transformers to image pixels after reducing image resolution and color space. The model is trained in an unsupervised fashion as a generative model, and the resulting representation can then be fine-tuned or probed linearly for classification performance, achieving a maximal accuracy of 72\% on \imagenet.


Our work adds to the increasing collection of papers that explore image recognition at larger scales than the standard \imagenet dataset.
The use of additional data sources allows to achieve state-of-the-art results on standard benchmarks~\citep{mahajan2018, touvron2019, xie2020-noisystudent}.
Moreover, \citet{sun2017-jft} study how CNN performance scales with dataset size, and \citet{kolesnikov2020-bit, djolonga2020-robustness} perform an empirical exploration of CNN transfer learning from large scale datasets such as ImageNet-21k and JFT-300M.
We focus on these two latter datasets as well, but train Transformers instead of ResNet-based models used in prior works. 



\newcommand{\op}[1]{\operatorname{#1}}
\newcommand{\mbf}[1]{\mathbf{#1}}

\section{Method}

\begin{figure}[h]
\begin{center}
\begin{tabular}{c}
\includegraphics[width=.83\textwidth]{images/model_scheme.pdf}
\end{tabular}
\end{center}
\caption{Model overview. We split an image into fixed-size patches, linearly embed each of them, add position embeddings, and feed the resulting sequence of vectors to a standard Transformer encoder. In order to perform classification, we use the standard approach of adding an extra learnable ``classification token'' to the sequence. The illustration of the Transformer encoder was inspired by \citet{vaswani2017}.}
\label{fig:model}
\end{figure}

In model design we follow the original Transformer \citep{vaswani2017} as closely as possible. 
An advantage of this intentionally simple setup is that scalable NLP Transformer architectures -- and their efficient implementations -- can be used almost out of the box.

\subsection{\oursfull{} (\oursabbrv{})}\label{sec:patch_transformer}

An overview of the model is depicted in Figure~\ref{fig:model}.
The standard Transformer receives as input a 1D sequence of token embeddings.
To handle 2D images, we reshape the image $\mbf{x} \in \mathbb{R}^{H \times W \times C}$ into a sequence of flattened 2D patches $\mbf{x}_p \in \mathbb{R}^{N \times (P^2 \cdot C)}$, where $(H, W)$ is the resolution of the original image, $C$ is the number of channels, $(P,P)$ is the resolution of each image patch, and $N=HW/P^2$ is the resulting number of patches, which also serves as the effective input sequence length for the Transformer.
The Transformer uses constant latent vector size $D$ through all of its layers, so we flatten the patches and map to $D$ dimensions with a trainable linear projection (Eq.~\ref{eq:embedding}).
We refer to the output of this projection as the patch embeddings.

Similar to BERT's \verb|[class]| token, we prepend a learnable embedding to the sequence of embedded patches ($\mbf{z}_0^0=\mbf{x}_\text{class}$), whose state at the output of the Transformer encoder ($\mbf{z}^0_L$) serves as the image representation $\mbf{y}$ (Eq.~\ref{eq:final_rep}). 
Both during pre-training and fine-tuning, a classification head is attached to $\mbf{z}^0_L$.
The classification head is implemented by a MLP with one hidden layer at pre-training time and by a single linear layer at fine-tuning time.

Position embeddings are added to the patch embeddings to retain positional information.
We use standard learnable 1D position embeddings, since we have not observed significant performance gains from using more advanced 2D-aware position embeddings  (Appendix~\ref{app:pos_emb}).
The resulting sequence of embedding vectors serves as input to the encoder.

The Transformer encoder \citep{vaswani2017} consists of alternating layers of multiheaded self-attention (MSA, see Appendix~\ref{sec:self_attention}) and MLP blocks (Eq.~\ref{eq:msa_apply}, \ref{eq:mlp_apply}).
Layernorm (LN) is applied before every block, and residual connections after every block \citep{wang2019-preLN,Baevski2019Adaptive}.
The MLP contains two layers with a GELU non-linearity.
\begin{align}
    \mbf{z}_0 &= [ \mbf{x}_\text{class}; \, \mbf{x}^1_p \mbf{E}; \, \mbf{x}^2_p \mbf{E}; \cdots; \, \mbf{x}^{N}_p \mbf{E} ] + \mbf{E}_{pos},
    && \mbf{E} \in \mathbb{R}^{(P^2 \cdot C) \times D},\, \mbf{E}_{pos}  \in \mathbb{R}^{(N + 1) \times D} \label{eq:embedding} \\
    \mbf{z^\prime}_\ell &= \op{MSA}(\op{LN}(\mbf{z}_{\ell-1})) + \mbf{z}_{\ell-1}, && \ell=1\ldots L \label{eq:msa_apply} \\
    \mbf{z}_\ell &= \op{MLP}(\op{LN}(\mbf{z^\prime}_{\ell})) + \mbf{z^\prime}_{\ell}, && \ell=1\ldots L  \label{eq:mlp_apply} \\
    \mbf{y} &= \op{LN}(\mbf{z}_L^0) \label{eq:final_rep}
\end{align}

\paragraph{Inductive bias.} 
We note that Vision Transformer has much less image-specific inductive bias than CNNs. 
In CNNs, locality, two-dimensional neighborhood structure, and translation equivariance are baked into each layer throughout the whole model. 
In ViT, only MLP layers are local and translationally equivariant, while the self-attention layers are global. 
The two-dimensional neighborhood structure is used very sparingly: in the beginning of the model by cutting the image into patches and at fine-tuning time for adjusting the position embeddings for images of different resolution (as described below).
Other than that, the position embeddings at initialization time carry no information about the 2D positions of the patches and all spatial relations between the patches have to be learned from scratch.

\paragraph{Hybrid Architecture.}
As an alternative to raw image patches, the input sequence can be formed from feature maps of a CNN~\citep{LeCun1989BackpropagationAT}.
In this hybrid model, the patch embedding projection $\mbf{E}$  (Eq.~\ref{eq:embedding}) is applied to patches extracted from a CNN feature map.
As a special case, the patches can have spatial size 1x1, which means that the input sequence is obtained by simply flattening the spatial dimensions of the feature map and projecting to the Transformer dimension.  
The classification input embedding and position embeddings are added as described above.

\subsection{Fine-tuning and Higher Resolution}

Typically, we pre-train ViT on large datasets, and fine-tune to (smaller) downstream  tasks.
For this, we remove the pre-trained prediction head and attach a zero-initialized $D\times K$ feedforward layer, where $K$ is the number of downstream classes.
It is often beneficial to fine-tune at higher resolution than pre-training~\citep{touvron2019,kolesnikov2020-bit}.
When feeding images of higher resolution, we keep the patch size the same, which results in a larger effective sequence length.
The \oursfull{} can handle arbitrary sequence lengths (up to memory constraints), however, the pre-trained position embeddings may no longer be meaningful.
We therefore perform 2D interpolation of the pre-trained position embeddings, according to their location in the original image.
Note that this resolution adjustment and patch extraction are the only points at which an inductive bias about the 2D structure of the images is manually injected into the \oursfull{}.


\section{Experiments}
We evaluate the representation learning capabilities of ResNet, \oursfull{} (\oursabbrv{}), and the hybrid.
To understand the data requirements of each model, we pre-train on datasets of varying size and evaluate many benchmark tasks.
When considering the computational cost of pre-training the model, \oursabbrv{} performs very favourably, attaining state of the art on most recognition benchmarks at a lower pre-training cost.
Lastly, we perform a small experiment using self-supervision, and show that self-supervised \oursabbrv{} holds promise for the future.

\subsection{Setup}

\textbf{Datasets.}
To explore model scalability, we use the ILSVRC-2012 ImageNet dataset with 1k classes and 1.3M images (we refer to it as \imagenet in what follows),
its superset ImageNet-21k with 21k classes and 14M images~\citep{deng2009-imagenet},
and JFT~\citep{sun2017-jft} with  18k classes and 303M high-resolution images.
We de-duplicate the pre-training datasets w.r.t. the test sets of the downstream tasks following~\citet{kolesnikov2020-bit}.
We transfer the models trained on these dataset to several benchmark tasks:
\imagenet on the original validation labels and the cleaned-up ReaL labels~\citep{beyer2020-imagenet},
CIFAR-10/100~\citep{Krizhevsky2009-cifar}, 
Oxford-IIIT Pets~\citep{parkhi2012-pets}, 
and  Oxford Flowers-102~\citep{Nilsback2008-flowers}.
For these datasets, pre-processing follows \citet{kolesnikov2020-bit}.

We also evaluate on the 19-task VTAB classification suite~\citep{vtab}.
VTAB evaluates low-data transfer to diverse tasks, using 1\,000 training examples per task.
The tasks are divided into three groups: 
\textit{Natural} -- tasks like the above, Pets, CIFAR, etc.
\textit{Specialized} -- medical and satellite imagery, and
\textit{Structured} -- tasks that require geometric understanding like localization.

\textbf{Model Variants.}
We base \oursabbrv{} configurations on those used for BERT~\citep{devlin19-bert}, as summarized in Table~\ref{tbl:models}. 
The ``Base'' and ``Large'' models are directly adopted from BERT and we add the larger ``Huge'' model.
In what follows we use brief notation to indicate the model size and the input patch size: for instance, \oursabbrv{}-L/16 means the ``Large'' variant with $16\times 16$ input patch size.
Note that the Transformer's sequence length is inversely proportional to the square of the patch size, thus models with smaller patch size are computationally more expensive.

For the baseline CNNs, we use ResNet~\citep{he2016deep}, but replace the Batch Normalization layers~\citep{ioffe2015batch} with Group Normalization~\citep{wu2018group}, and used standardized convolutions~\citep{qiao2019ws}.
These modifications improve transfer~\citep{kolesnikov2020-bit}, and we denote the modified model ``ResNet (BiT)''.
For the hybrids, we feed the intermediate feature maps into \oursabbrv{} with patch size of one ``pixel''.
To experiment with different sequence lengths, we either
(i) take the output of stage 4 of a regular ResNet50 or
(ii) remove stage 4, place the same number of layers in stage 3 (keeping the total number of layers), and take the output of this extended stage 3.
Option (ii) results in a 4x longer sequence length, and a more expensive \oursabbrv{} model.

\begin{table}[t]
\centering
\small
\begin{tabular}{l c c c c c}
\toprule
Model            & Layers & Hidden size $D$ & MLP size &  Heads  & Params \\
\midrule 
\oursabbrv-Base   &   12   &        768      &   3072   &   12    &   86M  \\
\oursabbrv-Large  &   24   &       1024      &   4096   &   16    &  307M  \\
\oursabbrv-Huge   &   32   &       1280      &   5120   &   16    &  632M  \\
\bottomrule
\end{tabular}
\caption{Details of \oursfull model variants.}
\label{tbl:models}
\end{table}

\textbf{Training \& Fine-tuning.}
We train all models, including ResNets, using Adam~\citep{kingma2015adam} with $\beta_1=0.9$, $\beta_2=0.999$, a batch size of 4096 and apply a high weight decay of $0.1$, which we found to be useful for transfer of all models (Appendix~\ref{sec:sgd_vs_adam} shows that, in contrast to common practices, Adam works slightly better than SGD for ResNets in our setting).
We use a linear learning rate warmup and decay, see Appendix~\ref{sec:training} for details.
For fine-tuning we use SGD with momentum, batch size 512, for all models, see Appendix~\ref{sec:finetuning}.
For \imagenet results in Table~\ref{tbl:best_results}, we fine-tuned at higher resolution: $512$ for \oursabbrv-L/16 and $518$ for \oursabbrv-H/14, and also used \citet{polyak} averaging with a factor of $0.9999$~\citep{ramachandran19-sasa,wang2020axial}.


\textbf{Metrics.}
We report results on downstream datasets either through few-shot or fine-tuning accuracy.
Fine-tuning accuracies capture the performance of each model after fine-tuning it on the respective dataset.
Few-shot accuracies are obtained by solving a regularized least-squares regression problem that maps the (frozen) representation of a subset of training images to $\{-1,1\}^K$ target vectors.
This formulation allows us to recover the exact solution in closed form.
Though we mainly focus on fine-tuning performance, we sometimes use linear few-shot accuracies for fast on-the-fly evaluation where fine-tuning would be too costly.

\subsection{Comparison to State of the Art}

We first compare our largest models~-- \oursabbrv-H/14 and \oursabbrv-L/16 ~-- to state-of-the-art CNNs from the literature.
The first comparison point is Big Transfer (BiT)~\citep{kolesnikov2020-bit}, which performs supervised transfer learning with large ResNets.
The second is Noisy Student~\citep{xie2020-noisystudent},
which is a large EfficientNet trained using semi-supervised learning on \imagenet and JFT-300M with the labels removed.
Currently, Noisy Student is the state of the art on \imagenet and BiT-L on the other datasets reported here.
All models were trained on TPUv3 hardware, and we report the number of TPUv3-core-days taken to pre-train each of them, that is, the number of TPU v3 cores (2 per chip) used for training multiplied by the training time in days.

\begin{table}[t]
\centering
\resizebox{\textwidth}{!}{
\begin{tabular}{l c c c c c}
\toprule
                   & Ours-JFT                & Ours-JFT              & Ours-I21k             & BiT-L                 & Noisy Student    \\
                   & (\oursabbrv-H/14)       & (\oursabbrv-L/16)     & (\oursabbrv-L/16)     & (ResNet152x4)         & (EfficientNet-L2)    \\
\midrule 
\imagenet          & \valstdb{88.55}{0.04}   & \valstd{87.76}{0.03}  & \valstd{85.30}{0.02}  & \valstd{87.54}{0.02}  &  $88.4/88.5^*$           \\
\imagenet ReaL     & \valstdb{90.72}{0.05}   & \valstd{90.54}{0.03}  & \valstd{88.62}{0.05}  & $90.54$               &  $90.55$          \\
CIFAR-10           & \valstdb{99.50}{0.06}   & \valstd{99.42}{0.03}  & \valstd{99.15}{0.03}  & \valstd{99.37}{0.06}  &  $-$              \\
CIFAR-100          & \valstdb{94.55}{0.04}   & \valstd{93.90}{0.05}  & \valstd{93.25}{0.05}  & \valstd{93.51}{0.08}  &  $-$              \\
Oxford-IIIT Pets   & \valstdb{97.56}{0.03}   & \valstd{97.32}{0.11}  & \valstd{94.67}{0.15}  & \valstd{96.62}{0.23}  &  $-$              \\
Oxford Flowers-102 & \valstd{99.68}{0.02}    & \valstdb{99.74}{0.00} & \valstd{99.61}{0.02}  & \valstd{99.63}{0.03}  &  $-$              \\
VTAB (19 tasks)    & \valstdb{77.63}{0.23}   & \valstd{76.28}{0.46}  & \valstd{72.72}{0.21}    & \valstd{76.29}{1.70}  &  $-$              \\
\midrule 
TPUv3-core-days    & $2.5$k                  & $0.68$k               & $0.23$k               & $9.9$k                &  $12.3$k           \\
\bottomrule
\end{tabular}
}
\caption{
Comparison with state of the art on popular image classification benchmarks.
We report mean and standard deviation of the accuracies, averaged over three fine-tuning runs.
\oursfull models pre-trained on the JFT-300M dataset outperform ResNet-based baselines on all datasets, while taking substantially less computational resources to pre-train.
\oursabbrv pre-trained on the smaller public ImageNet-21k dataset performs well too.
$^*$Slightly improved $88.5\%$ result reported in~\citet{touvron2020}.}
\label{tbl:best_results}
\end{table}

\begin{figure}[]
\begin{center}
\includegraphics[width=0.98\textwidth]{images/vit-vtab.pdf}
\end{center}
\caption{Breakdown of VTAB performance in \textit{Natural}, \textit{Specialized}, and \textit{Structured} task groups. 
}
\label{fig:vtab}
\vspace{-3mm}
\end{figure}


Table~\ref{tbl:best_results} shows the results.
The smaller \oursabbrv-L/16 model pre-trained on JFT-300M outperforms BiT-L (which is pre-trained on the same dataset) on all tasks, while requiring substantially less computational resources to train.
The larger model, \oursabbrv-H/14, further improves the performance, especially on the more challenging datasets~-- \imagenet, CIFAR-100, and the VTAB suite.
Interestingly, this model still took substantially less compute to pre-train than prior state of the art. However, we note that pre-training efficiency may be affected not only by the architecture choice, but also other parameters, such as training schedule, optimizer, weight decay, etc.
We provide a controlled study of performance vs. compute for different architectures in Section~\ref{sec:scaling_architectures}.
Finally, the \oursabbrv-L/16 model pre-trained on the public ImageNet-21k dataset performs well on most datasets too, while taking fewer resources to pre-train: it could be trained using a standard cloud TPUv3 with 8 cores in approximately 30 days.

Figure~\ref{fig:vtab} decomposes the VTAB tasks into their respective groups, and compares to previous SOTA methods on this benchmark:
BiT,
VIVI -- a ResNet co-trained on \imagenet and Youtube~\citep{vivi},
and S4L -- supervised plus semi-supervised learning on \imagenet~\citep{zhai2019s4l}.
\oursabbrv{}-H/14 outperforms BiT-R152x4, and other methods, on the \textit{Natural} and \textit{Structured} tasks.
On the \textit{Specialized} the performance of the top two models is similar.

\subsection{Pre-training Data Requirements}
\label{sec:data_efficiency}

\begin{figure}
    \begin{minipage}[t]{0.47\textwidth}
        \centering
        \includegraphics[width=1.0\textwidth]{images/dataset_analysis/transvolution-i1k-scaling}
        \caption{Transfer to \imagenet. While large \oursabbrv{} models perform worse than BiT ResNets (shaded area) when pre-trained on small datasets, they shine when pre-trained on larger datasets. Similarly, larger \oursabbrv{}  variants overtake smaller ones as the dataset grows.}
        \label{fig:imagenet_imagenet21k_jft}
    \end{minipage}\;\;\;\;
    \begin{minipage}[t]{0.47\textwidth}
        \centering
        \includegraphics[width=1.0\textwidth]{images/dataset_analysis/imagenet_5shot}
        \caption{Linear few-shot evaluation on ImageNet versus pre-training size. 
        ResNets perform better with smaller pre-training datasets but plateau sooner than \oursabbrv{}, which performs better with larger pre-training. \oursabbrv-b is \oursabbrv-B with all hidden dimensions halved.}
        \label{fig:jft_amount_of_data}
    \end{minipage}
    \vspace{-3mm}
\end{figure}

The \oursfull performs well when pre-trained on a large JFT-300M dataset. 
With fewer inductive biases for vision than ResNets, how crucial is the dataset size?
We perform two series of experiments.

First, we pre-train \oursabbrv{} models on datasets of increasing size: \imagenet, ImageNet-21k, and JFT-300M.
To boost the performance on the smaller datasets, we optimize three basic regularization parameters~-- weight decay, dropout, and label smoothing.
Figure~\ref{fig:imagenet_imagenet21k_jft} shows the results after fine-tuning to \imagenet (results on other datasets are shown in Table~\ref{tbl:imagenet_imagenet21k_jft})\footnote{Note that the \imagenet pre-trained models are also fine-tuned, but again on \imagenet. This is because the resolution increase during fine-tuning improves the performance.}.
When pre-trained on  the smallest dataset, \imagenet, \oursabbrv{}-Large models underperform compared to \oursabbrv{}-Base models, despite (moderate) regularization.
With ImageNet-21k pre-training, their  performances are similar.
Only with JFT-300M, do we see the full benefit of larger models.
Figure~\ref{fig:imagenet_imagenet21k_jft} also shows the performance region spanned by BiT models of different sizes.
The BiT CNNs outperform \oursabbrv{} on \imagenet, but with the larger datasets, \oursabbrv{} overtakes.

Second, we train our models on random subsets of 9M, 30M, and 90M as well as the full JFT-300M dataset.
We do not perform additional regularization on the smaller subsets and use the same hyper-parameters for all settings.
This way, we assess the intrinsic model properties, and not the effect of regularization.
We do, however, use early-stopping, and report the best validation accuracy achieved during training.
To save compute, we report few-shot linear accuracy instead of full fine-tuning accuracy.
Figure~\ref{fig:jft_amount_of_data} contains the results.
\oursfull{}s overfit more than ResNets with comparable computational cost on smaller datasets. 
For example, \oursabbrv-B/32 is slightly faster than ResNet50; it performs much worse on the 9M subset, but better on 90M+ subsets.
The same is true for ResNet152x2 and \oursabbrv-L/16.
This result reinforces the intuition that the convolutional inductive bias is useful for smaller datasets, but for larger ones, learning the relevant patterns directly from data is sufficient, even beneficial.

Overall, the few-shot results on ImageNet (Figure~\ref{fig:jft_amount_of_data}), as well as the low-data results on VTAB (Table~\ref{tbl:best_results}) seem promising for very low-data transfer. Further analysis of few-shot properties of \oursabbrv{} is an exciting direction of future work.






\begin{figure}
\begin{center}
\includegraphics[width=0.92\textwidth]{images/finetune_vs_compute2}
\vspace{-3mm}
\end{center}
\caption{Performance versus pre-training compute for different architectures: \oursfull{}s, ResNets, and hybrids. \oursfull{}s generally outperform ResNets with the same computational budget. Hybrids improve upon pure Transformers for smaller model sizes, but the gap vanishes for larger models.}
\vspace{-3mm}
\label{fig:scaling_architectures}
\end{figure}

\subsection{Scaling Study}
\label{sec:scaling_architectures}

We perform a controlled scaling study of different models by evaluating transfer performance from JFT-300M.
In this setting data size does not bottleneck the models' performances, and we assess performance versus pre-training cost of each model.
The model set includes: 
7 ResNets, R50x1, R50x2 R101x1, R152x1, R152x2, pre-trained for 7 epochs, plus R152x2 and R200x3 pre-trained for 14 epochs;
6 \oursfull{}s, \oursabbrv-B/32, B/16, L/32, L/16, pre-trained for 7 epochs, plus L/16 and H/14 pre-trained for 14 epochs;
and 5 hybrids, R50+\oursabbrv-B/32, B/16, L/32, L/16 pre-trained for 7 epochs, plus R50+\oursabbrv-L/16 pre-trained for 14 epochs (for hybrids, the number at the end of the model name stands not for the patch size, but for the total dowsampling ratio in the ResNet backbone).

Figure~\ref{fig:scaling_architectures} contains the transfer performance versus  total pre-training compute (see  Appendix~\ref{sec:empirical_computation} for details on computational costs).
Detailed results per model are provided in Table~\ref{tbl:scaling_architectures} in the Appendix.
A few patterns can be observed.
First, \oursfull{}s dominate ResNets on the performance/compute trade-off.
\oursabbrv{} uses approximately $2-4\times$ less compute to attain the same performance (average over 5 datasets).
Second, hybrids slightly outperform \oursabbrv{} at small computational budgets, but the difference vanishes for larger models.
This result is somewhat surprising, since one might expect convolutional local feature processing to assist \oursabbrv{} at any size.
Third, \oursfull{}s appear not to saturate within the range tried, motivating future scaling efforts.



\subsection{Inspecting \oursfull{}}
\begin{wrapfigure}{r}{0.3\textwidth}
\centering
\vspace{-2.7mm}
\includegraphics[height=2.5in,trim={0in 0in 0.15in 0in},clip]{images/visualizations/20201002_selected_attention_examples.pdf}
\caption{Representative examples of attention from the output token to the input space. See Appendix~\ref{sec:appendix_attention_distance} for details.}
\label{fig:selected_attention_examples}
\vspace{-4mm}
\end{wrapfigure}
To begin to understand how the \oursfull{} processes image data, we analyze its internal representations.
The first layer of the \oursfull{} linearly projects the flattened patches into a lower-dimensional space (Eq.~\ref{eq:embedding}). 
Figure~\ref{fig:transformer_visualization} (left) shows the top principal components of the the learned embedding filters. 
The components resemble plausible basis functions for a low-dimensional representation of the fine structure within each patch.

After the projection, a learned position embedding is added to the  patch representations. 
Figure~\ref{fig:transformer_visualization} (center) shows that the model learns to encode distance within the image in the similarity of position embeddings, i.e. closer patches tend to have more similar position embeddings. 
Further, the row-column structure appears; patches in the same row/column have similar embeddings.
Finally, a sinusoidal structure is sometimes apparent for larger grids (Appendix~\ref{sec:additional_analyses}). 
That the position embeddings learn to represent 2D image topology explains why hand-crafted 2D-aware embedding variants do not yield improvements (Appendix~\ref{app:pos_emb}).

\begin{figure}[t]
\vspace{3mm}
    \includegraphics[height=1.5in,trim={0.15in 0in 0in 0in},clip]{images/visualizations/20201002_rgb_filter_pca.pdf}
    \hfill
    \raisebox{-0.15in}{\includegraphics[height=1.5in]{images/visualizations/20201002_position_embeddings_17085772_1.pdf}}
    \hfill
    \raisebox{-0.15in}{\includegraphics[height=1.5in,trim={0in 0in 0in 0in},clip]{images/visualizations/20201002_attention_distance_by_depth_main.pdf}}
    \caption{
        \textbf{Left:} Filters of the initial linear embedding of RGB values of \oursabbrv-L/32.
        \textbf{Center:} Similarity of position embeddings of \oursabbrv-L/32. Tiles show the cosine similarity between the position embedding of the patch with the indicated row and column and the position embeddings of all other patches.
        \textbf{Right:} Size of attended area by head and network depth. Each dot shows the mean attention distance across images for one of 16 heads at one layer. See Appendix~\ref{sec:appendix_attention_distance} for details.}
    \label{fig:transformer_visualization}
    \vspace{5mm}
\end{figure}

Self-attention allows \oursabbrv to integrate information across the entire image even in the lowest layers. We investigate to what degree the network makes use of this capability. Specifically, we compute the average distance in image space across which information is integrated, based on the attention weights (Figure~\ref{fig:transformer_visualization}, right). This ``attention distance'' is analogous to receptive field size in CNNs. We find that some heads attend to most of the image already in the lowest layers, showing that the ability to integrate information globally is indeed used by the model. Other attention heads have consistently small attention distances in the low layers. This highly localized attention is less pronounced in hybrid models that apply a ResNet before the Transformer (Figure~\ref{fig:transformer_visualization}, right), suggesting that it may serve a similar function as early convolutional layers in CNNs. 
Further, the attention distance increases with network depth.
Globally, we find that the model attends to image regions that are semantically relevant for classification (Figure~\ref{fig:selected_attention_examples}).


\subsection{Self-supervision}
Transformers show impressive performance on NLP tasks. However, much of their success stems not only from their excellent scalability but also from large scale self-supervised pre-training \citep{devlin19-bert,radford2018-gpt}. We also perform a preliminary exploration on \textit{masked patch prediction} for self-supervision, mimicking the masked language modeling task used in BERT. With self-supervised pre-training, our smaller \oursabbrv-B/16 model achieves 79.9\% accuracy on \imagenet{}, a significant improvement of 2\% to training from scratch, but still 4\% behind supervised pre-training. Appendix~\ref{sec:self_supervision} contains further details.
We leave exploration of contrastive pre-training \citep{Chen2020simclr,he2020moco,bachman2019amdim,henaff2020cpc} to future work.





\section{Conclusion}

We have explored the direct application of Transformers to image recognition.
Unlike prior works using self-attention in computer vision, we do not introduce image-specific inductive biases into the architecture apart from the initial patch extraction step.
Instead, we interpret an image as a sequence of patches and process it by a standard Transformer encoder as used in NLP.
This simple, yet scalable, strategy works surprisingly well when coupled with pre-training on large datasets.
Thus, \oursfull matches or exceeds the state of the art on many image classification datasets, whilst being relatively cheap to pre-train.

While these initial results are encouraging, many challenges remain.
One is to apply \oursabbrv to other computer vision tasks, such as detection and segmentation.
Our results, coupled with those in \citet{carion20-detr}, indicate the promise of this approach.
Another challenge is to continue exploring self-supervised pre-training methods.
Our initial experiments show improvement from self-supervised pre-training, but there is still large gap between self-supervised and large-scale supervised pre-training.
Finally, further scaling of \oursabbrv{} would likely lead to improved performance.

\section*{Acknowledgements}
The work was performed in Berlin, Z\"urich, and Amsterdam. We thank many colleagues at Google for their help, in particular Andreas Steiner for crucial help with the infrastructure and the open-source release of the code; Joan Puigcerver and Maxim Neumann for help with the large-scale training infrastructure; Dmitry Lepikhin, Aravindh Mahendran, Daniel Keysers, Mario Lu\v{c}i\'{c}, Noam Shazeer, Ashish Vaswani, and Colin Raffel for useful discussions.


\bibliography{transvolution}
\bibliographystyle{iclr2021_conference}

\newpage
\input{06_appendix}

\end{document}
