
\documentclass{article} % For LaTeX2e
\usepackage{iclr2023_conference,times}

% Optional math commands from https://github.com/goodfeli/dlbook_notation.
\input{math_commands.tex}

\usepackage{hyperref}
\usepackage{url}
\usepackage{dcolumn}

% added by Yujia
\usepackage{colortbl}
\usepackage{url}
\usepackage{xspace,mfirstuc,tabulary}
\usepackage{booktabs}
\usepackage{amssymb}
\usepackage{amsmath}
\usepackage{multirow,booktabs, hhline}
\usepackage[ruled,noend]{algorithm2e}
\usepackage{amsmath, bm}
\usepackage{graphicx}
\usepackage{color}
\usepackage{subfigure}
\usepackage{enumitem}
\newenvironment{itemize*}%
 {\leftmargini=20pt\begin{itemize}%
  \setlength{\itemsep}{3pt}%
  \setlength{\parskip}{0pt}%
  }%
 {\end{itemize}}
\newenvironment{enumerate*}%
 {\begin{enumerate}%
  \setlength{\itemsep}{0pt}%
  \setlength{\parskip}{0pt}}%
 {\end{enumerate}}

\usepackage{makecell}
\usepackage{pifont}

\usepackage{multicol}
% \usepackage{algorithm}  
% \usepackage{algorithmicx}  
% \usepackage{algpseudocode}


\usepackage{microtype}
\usepackage{xspace,mfirstuc,tabulary}
\usepackage{booktabs}
\usepackage{amssymb}
\usepackage{pifont}
\usepackage{amsmath}
\usepackage{multirow,booktabs, hhline}
\usepackage[ruled,noend]{algorithm2e}
\usepackage{arydshln}
\usepackage{amsmath, bm}
\usepackage{color}
\usepackage{bbm}
\usepackage{bbding}
\usepackage{subfigure}
\usepackage{makecell}
\usepackage{CJKutf8}
\usepackage{cleveref}
\usepackage{listings}
\usepackage{titlesec}
\crefname{section}{§}{§§}
\Crefname{section}{§}{§§}
\usepackage{wrapfig}
\usepackage{lipsum}
% \usepackage{xcolor}
\usepackage{hyperref}


\hypersetup{
    % colorlinks=true,
    % linkcolor=blue,          % color of internal links
    % citecolor=blue,          % color of links to bibliography
    % filecolor=magenta,      % color of file links
    % urlcolor=blue            % color of external links
}


\definecolor{lightergray}{RGB}{230,230,230}
\definecolor{DarkGreen}{RGB}{30,130,30}
\newcommand{\cmark}{\textcolor{DarkGreen}{\ding{51}}}
\newcommand{\xmark}{\textcolor{red}{\ding{55}}}%
\newcommand\icon{\raisebox{-3.7pt}{\includegraphics[width=1.5em]{figs/logo.png}}}

\newcommand\ourdata{ToolBench\xspace}
\newcommand\ourmodel{ToolLLaMA\xspace}
\newcommand\dfs{DFSDT\xspace}
\newcommand\turbo{ChatGPT\xspace}

\title{\icon ToolLLM: Facilitating Large Language Models to Master 16000+ Real-world APIs}

\iclrfinalcopy
\author{Yujia Qin$^{1}\thanks{\ \ Indicates equal contribution.}$\hspace{0.5em}, Shihao Liang$^{1*}$, Yining Ye$^1$, Kunlun Zhu$^{1}$, Lan Yan$^{1}$, Yaxi Lu$^1$, Yankai Lin$^3\thanks{\ \  Corresponding author.}$\hspace{0.5em}, \\ \textbf{Xin Cong$^1$, Xiangru Tang$^4$, Bill Qian$^4$, Sihan Zhao$^1$, Lauren Hong$^1$, Runchu Tian$^1$,} \\
\textbf{Ruobing Xie$^5$, Jie Zhou$^5$, Mark Gerstein$^4$, Dahai Li$^{2,6}$, Zhiyuan Liu$^{1\dag}$, Maosong Sun$^{1\dag}$} \\
$^1$Tsinghua University $^2$ModelBest Inc. $^3$Renmin University of China\\
$^4$Yale University
$^5$WeChat AI, Tencent Inc. 
$^6$Zhihu Inc.
\\
\texttt{yujiaqin16@gmail.com} \\
}

% The \author macro works with any number of authors. There are two commands
% used to separate the names and addresses of multiple authors: \And and \AND.
%
% Using \And between authors leaves it to \LaTeX{} to determine where to break
% the lines. Using \AND forces a linebreak at that point. So, if \LaTeX{}
% puts 3 of 4 authors names on the first line, and the last on the second
% line, try using \AND instead of \And before the third author name.

\newcommand{\fix}{\marginpar{FIX}}
\newcommand{\new}{\marginpar{NEW}}

\begin{document}


\maketitle

\begin{abstract}
Despite the advancements of open-source large language models (LLMs), e.g., LLaMA, they remain significantly limited in tool-use capabilities, i.e., using external tools (APIs) to fulfill human instructions. The reason is that current instruction tuning largely focuses on basic language tasks but ignores the tool-use domain.
This is in contrast to the excellent tool-use capabilities of state-of-the-art (SOTA) closed-source LLMs, e.g., ChatGPT.
To bridge this gap, we introduce ToolLLM, a general tool-use framework encompassing data construction, model training, and evaluation.
We first present \ourdata, an instruction-tuning dataset for tool use, which is constructed automatically using ChatGPT. Specifically, the construction can be divided into three stages: (i) API collection: we collect $16,464$ real-world RESTful APIs spanning $49$ categories from RapidAPI Hub; (ii) instruction generation: we prompt \turbo to generate diverse instructions involving these APIs, covering both single-tool and multi-tool scenarios; (iii) solution path annotation: we use \turbo to search for a valid solution path (chain of API calls) for each instruction.
To enhance the reasoning capabilities of LLMs, we develop a novel depth-first search-based decision tree algorithm. It enables LLMs to evaluate multiple reasoning traces and expand the search space.
Moreover, to evaluate the tool-use capabilities of LLMs, we develop an automatic evaluator: ToolEval.
Based on \ourdata, we fine-tune LLaMA to obtain an LLM \ourmodel, and equip it with a neural API retriever to recommend appropriate APIs for each instruction. Experiments show that \ourmodel demonstrates a remarkable ability to execute complex instructions and generalize to unseen APIs, and exhibits comparable performance to ChatGPT. 
Our \ourmodel also demonstrates strong zero-shot generalization ability in an out-of-distribution tool-use dataset: APIBench.
The codes, trained models, and demo are publicly available at \url{https://github.com/OpenBMB/ToolBench}.

\end{abstract}

\section{Introduction}
\label{sec:introduction}
Tool learning~\citep{qin2023tool} aims to unleash the power of large language models (LLMs) to effectively interact with various tools (APIs) to accomplish complex tasks. By integrating LLMs with APIs, we can greatly expand their utility and empower them to serve as efficient intermediaries between users and the vast ecosystem of applications. Although open-source LLMs, e.g., LLaMA~\citep{touvron2023llama}, have achieved versatile capabilities through instruction tuning~\citep{alpaca,vicuna2023}, they still lack the sophistication in performing higher-level tasks, such as appropriately interacting with tools (APIs) to fulfill complex human instruction.
This deficiency is because current instruction tuning largely focuses on basic language tasks, with a relative neglect of the tool-use domain.
On the other hand, current state-of-the-art (SOTA) LLMs (e.g., ChatGPT~\citep{openaichatgptblog} and GPT-4~\citep{openai2023gpt4}), which have demonstrated impressive competencies in utilizing tools~\citep{bubeck2023sparks}, are closed-source with their inner mechanisms opaque. This limits the democratization of AI technologies and the scope of community-driven innovation and development.
In this regard, we deem it urgent to \textit{empower open-source LLMs to skillfully master diverse APIs.}
% Besides, such data can be constructed efficiently using self-instruct~\citep{wang2022self}, in a manner similar to performing knowledge distillation from SOTA LLMs.
% show that the capability gap between open-source LLMs and SOTA LLMs can be narrowed by supervised fine-tuning (SFT) on high-quality
% 跟真实场景很接近,不是局限于某一类问题,可落地/可行方案

% After supervised fine-tuning on \ourdata, LLaMA achieves comparable performance compared with \turbo.

\begin{figure*}[!t]
    \centering
    \subfigure{\includegraphics[width=\textwidth]{figs/overview.pdf}}
    \caption{
    \small{Three phases of constructing \ourdata and how we train our API retriever and \ourmodel. During inference of an instruction, the API retriever recommends relevant APIs to \ourmodel, which performs multiple rounds of API calls to derive the final answer. The whole reasoning process is evaluated by ToolEval.
    }
    }
    \label{fig:overview}
\end{figure*}

Although prior works have explored building instruction tuning data for tool use~\citep{li2023api,patil2023gorilla,tang2023toolalpaca,xu2023tool}, they fail to fully stimulate the tool-use capabilities within LLMs and have inherent limitations: (1) \textbf{limited APIs}: they either fail to involve real-world APIs (e.g., RESTAPI)~\citep{patil2023gorilla,tang2023toolalpaca} or consider only a small scope of APIs with poor diversity~\citep{patil2023gorilla,xu2023tool,li2023api};
\begin{wrapfigure}{r}{0.4\textwidth}
  \begin{center}
    \includegraphics[width=\linewidth]{figs/toolllama_scatterplot.pdf}
  \end{center}
  \caption{
    \small{Pass rate ($\uparrow$) and win rate ($\uparrow$) of different methods in tool-use evaluation. For win rate, we compare each method with ChatGPT-ReACT. \dfs is our improved reasoning strategy over ReACT. \ourmodel surpasses Text-Davinci-003, Claude-2, and almost performs on par with ChatGPT.}
    }
    \label{fig:show_figure}
\end{wrapfigure}
(2) \textbf{constrained scenario}: existing works are confined to instructions that only involve one single tool. In contrast, real-world scenarios may require that multiple tools are interleaved together for multi-round tool execution to solve a complex task. Besides, they often assume that users manually specify the ideal API set for a given instruction in advance, which is infeasible with a large collection of real-world APIs; (3) \textbf{inferior planning and reasoning}: existing works adopted either CoT~\citep{wei2023chainofthought} or ReACT~\citep{yao2022react} for model reasoning, which cannot fully elicit the capabilities stored in LLMs and thus fail to handle complex instructions. 
In addition, some works do not even execute APIs to obtain real responses~\citep{patil2023gorilla,tang2023toolalpaca}, which serve as important information for subsequent model planning.
% This issue is particularly severe for open-source LLMs, which exhibit markedly inferior reasoning ability compared with their SOTA counterparts.

To facilitate tool-use capabilities within open-source LLMs, we introduce \textbf{ToolLLM}, a general tool-use framework including data construction, model training, and evaluation.
As illustrated in Figure~\ref{fig:overview}, we collect a high-quality instruction-tuning dataset \textbf{\ourdata}. It is constructed automatically using ChatGPT (\textit{gpt-3.5-turbo-16k}), which has been upgraded with function call (\textcolor{blue}{\href{https://openai.com/blog/function-calling-and-other-api-updates}{link}}) capabilities. The comparison between \ourdata and prior works is listed in Table~\ref{tab:dataset_comparison}.
Specifically, the construction of \ourdata entails three phases: 


\begin{itemize} [topsep=1pt, partopsep=1pt, leftmargin=12pt, itemsep=-1pt]
    \item \textbf{API Collection}: we gather $\textbf{16,464}$ representational state transfer (REST) APIs from RapidAPI (\textcolor{blue}{\href{https://rapidapi.com/hub}{link}}), a platform that hosts massive real-world APIs provided by developers. 
    These APIs span $\mathbf{49}$ diverse categories such as social media, e-commerce, and weather. For each API, we crawl detailed API documents from RapidAPI, including the functionality descriptions, required parameters, code snippets for API calls, etc. By comprehending these documents to learn to execute APIs, LLMs can generalize to new APIs unseen during training;

    \item \textbf{Instruction Generation}: we first sample APIs from the whole set and then prompt ChatGPT to generate diverse instructions for these APIs. To cover practical scenarios, we curate instructions that involve both \textbf{single-tool} and \textbf{multi-tool} scenarios. This ensures that our model learns not only how to interact with individual tools but also how to combine them to accomplish complex tasks;

    \item \textbf{Solution Path Annotation}: %we annotate high-quality solution paths to these instructions. 
    each solution path may contain multiple rounds of model reasoning and real-time API calls to derive the final response.
    However, even the most sophisticated LLM, i.e., GPT-4, achieves a low pass rate for complex human instructions, making annotation inefficient. To this end, we develop a novel \textbf{depth-first search-based decision tree} (\dfs) to bolster the planning and reasoning ability of LLMs. Compared with conventional ReACT, \dfs enables LLMs to evaluate a multitude of reasoning paths and make deliberate decisions to either retract steps or proceed along a promising path. In experiments, \dfs significantly improves the annotation efficiency and successfully completes those complex instructions that cannot be fulfilled using ReACT.
\end{itemize}

\begin{table*}[!t]
    \centering
    \small
    \resizebox{0.95\linewidth}{!}{
    \begin{tabular}{lccccc}
        \toprule
        \textbf{Resource} & \makecell{\textbf{\ourdata}\\(this work)}& \makecell{ \textbf{APIBench} \\\citep{patil2023gorilla} } & 
        \makecell{ \textbf{API-Bank}\\\citep{li2023api} } &  \makecell{ \textbf{ToolAlpaca} \\\citep{tang2023toolalpaca} } & \makecell{ \textbf{ToolBench}\\\citep{xu2023tool} } 
        \\
         \cmidrule(lr){1-1}  \cmidrule(lr){2-2}  \cmidrule(lr){3-3}  \cmidrule(lr){4-4}  \cmidrule(lr){5-5} \cmidrule(lr){6-6}
          Real-world API?   & \cmark & \xmark & \cmark & \xmark & \cmark  \\ 
          Real API Call\&Response? & \cmark & \xmark & \cmark & \xmark & \cmark  \\ 
          Multi-tool Scenario?   & \cmark & \xmark & \xmark & \xmark & \xmark  \\ 
          API Retrieval?   & \cmark & \cmark & \xmark & \xmark & \cmark  \\ 
          Multi-step Reasoning?   & \cmark & \xmark & \cmark & \cmark & \cmark  \\ 
          \hdashline
          Number of tools   & $\mathbf{3451}$ & $3$ & $53$ & $400$ & $8$ \\ 
          Number of APIs   & $\mathbf{16464}$ & $1645$ & $53$ & $400$ & $232$ \\ 
          Number of Instances   & $\mathbf{126486}$ & $17002$ & $274$ & $3938$ & $2746$ \\ 
          Number of Real API Calls   & $\mathbf{469585}$ & $0$ & $568$ & $0$ & $3926$ \\ 
          Avg. Reasoning Traces  & $4.0$ & $1.0$ & $2.1$ & $1.0$ & $\mathbf{5.9}$ \\ 
         \bottomrule
    \end{tabular}
    }
    \caption{
    \small{A comparison of our \ourdata to notable instruction tuning dataset for tool learning.
    }
    }
    \label{tab:dataset_comparison}
\end{table*}

To assess the tool-use capabilities of LLMs, we develop an automatic evaluator, \textbf{ToolEval}, backed up by \turbo. It comprises two key metrics: (1) \textit{pass rate}, which measures LLM's ability to successfully execute an instruction within limited budgets, and (2) \textit{win rate}, which compares the quality and usefulness of two solution paths. We demonstrate that ToolEval achieves a high correlation with human evaluation and provides a robust, scalable, and reliable assessment for machine tool use.

By fine-tuning LLaMA on \ourdata, we obtain \textbf{\ourmodel}. After evaluation based on our ToolEval, we derive the following findings:

\begin{itemize} [topsep=1pt, partopsep=1pt, leftmargin=12pt, itemsep=-1pt]
    \item \ourmodel demonstrates a compelling capability to handle both single-tool and complex multi-tool instructions. As depicted in Figure~\ref{fig:show_figure}, \ourmodel outperforms Text-Davinci-003 and Claude-2, achieves comparable performance to the ``teacher model'' \turbo, and is only slightly inferior to GPT4.
    Besides, \ourmodel exhibits \textbf{robust generalization to previously unseen APIs}, requiring only the API documentation to adapt to new APIs effectively. This flexibility allows users to incorporate novel APIs seamlessly, thus enhancing the model's practical utility.

    \item We show that our \dfs serves as a general decision-making strategy to enhance the reasoning capabilities of LLMs. \dfs broadens the search space by considering multiple reasoning traces and achieves significantly better performance than ReACT.

    \item We train a neural \textbf{API retriever}, which alleviates the need for manual selection from the large API pool in practice. As shown in Figure~\ref{fig:overview}, given an instruction, the API retriever recommends a set of relevant APIs, which are sent to \ourmodel for multi-round decision making to derive the final answer.
    Despite sifting through a large pool of APIs, the retriever exhibits remarkable retrieval precision, returning APIs closely aligned with the ground truth.

    \item \ourmodel exhibits strong \textbf{generalization} performance on an \textbf{out-of-distribution} (OOD) dataset APIBench~\citep{patil2023gorilla}. Despite not training on any of the APIs or instructions on APIBench, \ourmodel performs on par with Gorilla, a pipeline specifically designed for APIBench.
\end{itemize}

% In summary, this work targets empowering open-source LLMs to execute complex instructions involving diverse APIs in practical scenarios. We hope this work will inspire further research in the intersection of instruction tuning and tool use.
\section{Dataset Construction}
We introduce the three-stage construction process of \ourdata: API collection (\cref{sec:api_collection}), instruction generation (\cref{sec:instruction_generation}), and solution path annotation (\cref{sec:answer_annotation}). All procedures are based on \turbo (\textit{gpt-3.5-turbo-16k}), requiring minimal human supervision and can be easily extended to new APIs. 

\subsection{API Collection}
\label{sec:api_collection}
We start by introducing RapidAPI and its hierarchy, followed by how we crawl and filter APIs.

\begin{figure*}[!t]
    \centering
    \subfigure{\includegraphics[width=\textwidth]{figs/phase12.pdf}}
    \caption{
    \small{The hierarchy of RapidAPI (left) and the process of instruction generation (right).}
    }
    \label{fig:phase12}
\end{figure*}

\textbf{RapidAPI Hub} \quad
RapidAPI is a leading API marketplace that connects developers with thousands of real-world APIs, streamlining the process of integrating diverse services into applications. Developers can test and connect with various APIs by registering only a RapidAPI key. All APIs in RapidAPI can be classified into $49$ \textit{coarse-grained} \textbf{categories} (\textcolor{blue}{\href{https://rapidapi.com/categories}{link}}), such as sports, finance, and weather. 
The categories associate an API with the most relevant topic.
Additionally, the hub also provides $500+$ \textit{fine-grained} categorization called \textbf{collections} (\textcolor{blue}{\href{https://rapidapi.com/collections}{link}}), e.g., Chinese APIs and database APIs. APIs in the same collection share a common characteristic and often have similar functionalities or goals.

\textbf{Hierarchy of RapidAPI} \quad
As shown in Figure~\ref{fig:phase12}, each tool may be composed of multiple APIs. For each tool, we crawl the following information: the name and description of the tool, the URL of the host, and all the available APIs belonging to the tool; for each API, we record its name, description, HTTP method, required parameters, optional parameters, request body, executable code snippets for API call, and an example API call response. This rich and detailed metadata serves as a valuable resource for LLMs to understand and effectively use the APIs, even in a zero-shot manner.

\textbf{API Filtering} \quad
Initially, we gathered $10,853$ tools ($53,190$ APIs) from RapidAPI. However, the quality and reliability of these APIs can vary significantly. In particular, some APIs may not be well-maintained, such as returning 404 errors or other internal errors.
To this end, we perform a rigorous filtering process (details in \cref{sec:detail_filtering_api}) to ensure that the ultimate tool set of \ourdata is reliable and functional.
Finally, we only retain $3,451$ high-quality tools ($16,464$ APIs).

\subsection{Instruction Generation}
\label{sec:instruction_generation}

Different from prior works, we specifically focus on two crucial aspects for instruction generation: (1) \textbf{diversity}: to train LLMs to handle a wide range of API usage scenarios, thereby boosting their generalizability and robustness; and (2) \textbf{multi-tool usage}: to mirror real-world situations that often demand the interplay of multiple tools, improving the practical applicability and flexibility of LLMs. 
To this end, instead of brainstorming instructions from scratch and then searching for relevant APIs, we sample different combinations of APIs and craft various instructions that involve them.
% In practice, this strategy ensures the coverage for all collected APIs.
% Furthermore, we use the RapidAPI hierarchy to identify tool relationships, which aids in the generation of multi-tool instructions.

\textbf{Generating Instructions for APIs} \quad
Define the total API set as $\sS_\text{API}$, at each time, we sample a few APIs: $\sS_\text{N}^{\text{sub}} \!=\! \{\text{API}_1, \cdots, \text{API}_\text{N}\}$ from $\sS_\text{API}$. We prompt \turbo to understand the functionalities of these APIs and then generate (1) possible instructions ($\text{Inst}_*$) that involve APIs in $\sS_\text{N}^{\text{sub}}$, and (2) relevant APIs ($\sS_*^{\text{rel}} \!\subset\!\sS_\text{N}^{\text{sub}}$) for each instruction ($\text{Inst}_*$), i.e., $\{[\sS_1^{\text{rel}}, \text{Inst}_1], \cdots, [\sS_{\text{N}'}^{\text{rel}}, \text{Inst}_{\text{N}'}]\}$, where $\text{N}'$ denotes the number of generated instances.
These (instruction, relevant API) pairs will be used for training the API retriever in \cref{sec:prelim_exp}.
We use different sampling strategies (introduced later) to cover all APIs and most of their combinations, thus ensuring the diversity of our instructions.

The prompt for \turbo is composed of (1) a general description of the intended instruction generation task, (2) comprehensive documentation of each API in $\sS_\text{N}^{\text{sub}}$, which helps \turbo understand their functionality and interplay, and (3) three in-context seed examples $\{\text{seed}_1, \text{seed}_2, \text{seed}_3\}$. Each seed example is an ideal instruction generation written by human experts. These seed examples are leveraged to better regulate \turbo's behavior through in-context learning. In total, we wrote $12$ / $36$ diverse seed examples ($\sS_\text{seed}$) for the single-tool / multi-tool setting, and randomly sampled three examples at each time. Detailed prompts for instruction generation are described in \cref{sec:inst_prompt}. Overall, the generation process can be formulated as follows:
\begin{equation}
    \small
    \begin{aligned}
    \underset{\{\text{API}_1, \cdots, \text{API}_\text{N}\} \in \sS_{\text{API}}, \{\text{seed}_1, \cdots, \text{seed}_3\} \in \sS_\text{seed}}{\text{\turbo}}(\{[\sS_1^{\text{rel}}, \text{Inst}_1], \cdots, [\sS_{\text{N'}}^{\text{rel}}, \text{Inst}_{\text{N}'}]\} | \text{API}_1, \cdots, \text{API}_\text{N}, \text{seed}_1, \cdots, \text{seed}_3).
        \nonumber
    \end{aligned}
    \label{eq:instruction_generation}
\end{equation}

% \underset{\substack{\{\text{API}_1, \cdots, \text{API}_\text{N}\} \in \sS_{\text{API}}, \\ \{\text{seed}_1, \cdots, \text{seed}_3\} \in \sS_\text{seed}}}{\text{\turbo}}(\{[\sS_1^{\text{rel}}, \text{Inst}_1], \cdots, [\sS_{\text{N'}}^{\text{rel}}, \text{Inst}_{\text{N}'}]\} | \text{API}_1, \cdots, \text{API}_\text{N}, \text{seed}_1, \cdots, \text{seed}_3).


\textbf{Sampling Strategies for Different Scenarios} \quad
As shown in Figure~\ref{fig:phase12}, for the \textbf{single-tool instructions (I1)}, we iterate over each tool and generate instructions for its APIs. However, for the \textbf{multi-tool setting}, since the interconnections among different tools in RapidAPI are sparse, random sampling tool combinations from the whole tool set often leads to a series of irrelevant tools that cannot be covered by a single instruction in a natural way. To address the sparsity issue, we leverage the RapidAPI hierarchy information.
Since tools belonging to the same RapidAPI \textit{category} or \textit{collection} are generally related to each other in the functionality and goals, we randomly select $2$-$5$ tools from the same category / collection and sample at most $3$ APIs from each tool to generate the instructions. We denote the generated instructions as \textbf{intra-category multi-tool instructions (I2)} and \textbf{intra-collection multi-tool instructions (I3)}, respectively. Through rigorous human evaluation, we find that instructions generated in this way already have a high diversity that covers various practical scenarios. We also provide visualization for instructions using Atlas (\textcolor{blue}{\href{https://atlas.nomic.ai/map/58aca169-c29a-447a-8f01-0d418fc4d341/030ddad7-5305-461c-ba86-27e1ca79d899}{link}}) to support our claim.

\begin{figure*}[!t]
    \centering
    \subfigure{\includegraphics[width=\textwidth]{figs/answer_anno.pdf}}
    \caption{
    \small{A comparison of our \dfs and conventional CoT or ReACT during model reasoning (left). We show part of the solution path annotation process using \turbo (right).}
    }
    \label{fig:phase3}
\end{figure*}

After generating the initial set of instructions, we further filter those with the hallucinated relevant APIs by assessing whether they exist in $\sS_\text{N}^{\text{sub}}$.
Finally, we collect nearly $200$k qualified (instruction, relevant API) pairs, including $87413$, $84815$, and $25251$ instances for I1, I2, and I3, respectively.

\subsection{Solution Path Annotation}
\label{sec:answer_annotation}
As shown in Figure~\ref{fig:phase3}, given an instruction $\text{Inst}_*$, we prompt \turbo to search for a valid action sequence: $\{a_1, \cdots, a_\text{N}\}$. Such a multi-step decision-making process is cast as a multi-round conversation for \turbo. At each round $t$, the model generates an action $a_t$ based on previous interactions, i.e., $\text{\turbo}(a_t|\{a_1, r_1, \cdots, a_{t-1}, r_{t-1}\}, \text{Inst}_*)$, where $r_*$ denotes the real API response.
For each $a_t$, \turbo should specify its ``thought'',  which API to use, and the specific parameters for this API, i.e., $a_t$ has the following format: ``\texttt{Thought}: $\cdots$, \texttt{API Name}: $\cdots$, \texttt{Parameters}: $\cdots$''.
% Hence the decision space is the Cartesian product of the thought, available APIs, and possible parameters, which is infinite by nature.

To leverage the \textbf{function call} feature of \turbo, we treat each API as a special function and feed its API documentation into \turbo's function field. In this way, the model understands how to call the API. For each instruction $\text{Inst}_*$, we feed all the sampled APIs $\sS_\text{N}^{\text{sub}}$ to \turbo's as available functions.
To let \turbo finish an action sequence, we define two additional functions, i.e., ``Finish with Final Answer'' and ``Finish by Giving Up''. The former function has a parameter that corresponds to a detailed final answer to the original instruction; while the latter function is designed for cases where the provided APIs cannot complete the original instruction after multiple API call attempts.
% instead of only its relevant APIs $\sS_*^{\text{rel}}$. In this way, the model gains access to a broader scope of APIs and expands the action space.

\textbf{Depth First Search-based Decision Tree} \quad
In our pilot studies, we find that CoT~\citep{wei2023chainofthought} or ReACT~\citep{yao2022react} has inherent limitations: (1) \textbf{error propagation}: a mistaken action may propagate the errors further and cause the model to be trapped in a faulty loop, such as continually calling an API in a wrong way or hallucinating APIs; (2) \textbf{limited exploration}: CoT or ReACT only explores one possible direction, leading to limited exploration of the whole action space. Hence even GPT-4 often fails to find a valid solution path, making annotation difficult.

To this end, we propose to construct a decision tree to expand the search space and increase the possibility of finding a valid path. As depicted in Figure~\ref{fig:phase3}, our \dfs allows the model to assess different reasoning paths and choose to either (1) proceed along a promising path or (2) abandon an existing node by calling the ``Finish by Giving Up'' function and expand a new node. During node expansion, to diversify the child nodes and expand the search space, we prompt \turbo with the information of the previously generated nodes and explicitly encourage the model to generate a distinct node. For the searching process, we prefer depth-first search (DFS) instead of breadth-first search (BFS) because the annotation can be finished as long as one valid path is found. Using BFS will cost excessive OpenAI API calls. More details are described in \cref{sec:answer_prompt}.
% before reaching a terminal node (the node of ``Finish with Final Answer'' or ``Finish by Giving Up'').
We perform \dfs for all the generated instructions and only retain those passed solution paths. Ultimately, we generate $126,486$ (instruction, solution path) pairs, which are used to train \ourmodel in \cref{sec:main_exp}. % Although it is possible to construct more training instances, we find that $12,657$ instances already bring satisfying generalization performance.
\section{Experiments}
\label{sec:exp_toolllama}
In this section, we investigate the performance of ToolLLM framework. We first introduce the evaluation metric and evaluate the efficacy of API retriever and \dfs in \cref{sec:prelim_exp}. Then we present the main experiments in \cref{sec:main_exp}, followed by a generalization experiment in \cref{sec:ood_exp}.

\subsection{Preliminary Experiments}
\label{sec:prelim_exp}

\textbf{ToolEval} \quad
Considering the API's temporal variability on RapidAPI and the infinite potential solution paths for an instruction, it is infeasible to annotate a fixed ground-truth solution path for each test instruction. Moreover, when comparing different models, it is crucial to ensure they employ the same version of APIs during evaluation.
Considering that human evaluation can be time-consuming, we follow AlpacaEval~\citep{alpaca_eval} to develop an efficient evaluator \textbf{ToolEval} based on \turbo, which incorporates two evaluation metrics (details in \cref{sec:details_tooleval}): (1) \textbf{Pass Rate}: it calculates the proportion of successfully completing an instruction within limited budgets. The metric measures the executability of instructions for an LLM and can be seen as a basic requirement for ideal tool use; and (2) \textbf{Win Rate}: we provide an instruction and two solution paths to \turbo evaluator and obtain its preference (i.e., which one is better). We pre-define a set of criteria for both metrics and these criteria are organized as prompts for our \turbo evaluator. We evaluate multiple times based on \turbo to improve the reliability. Then we calculate the average results from the evaluator.

Through rigorous testing (details in \cref{sec:details_tooleval}), we find that ToolEval demonstrates a high agreement of $87.1\%$ in pass rate and $80.3\%$ in win rate with human annotators. This shows that ToolEval can reflect and represent human evaluation to a large extent.
% We also obtain the agreement (see Appendix \ref{sec:details_of_metrics_for_evaluator}) among different human annotators ($\textbf{83.54\%}$), and the agreement between humans and our evaluator ($\textbf{80.21\%}$).

\textbf{Efficacy of API Retriever} \quad
The API retriever aims to retrieve relevant APIs to an instruction. We employ Sentence-BERT~\citep{reimers2019sentence} to train a dense retriever based on BERT-BASE~\citep{devlin2018bert}. The API retriever encodes the instruction and API document into two embeddings, and 
calculates their relevance with embedding similarity. For training, we regard the relevant APIs of each instruction generated in \cref{sec:instruction_generation} as positive examples and sample a few other APIs as negative examples for contrastive learning.
% We chose bert-base as our base model for the dense passage retrieval method. We set the sequence length as 256, batch size as 32, and trained 5 epochs. 
For baselines, we choose BM25~\citep{robertson2009probabilistic} and OpenAI's \textit{text-embedding-ada-002} (\textcolor{blue}{\href{https://openai.com/blog/new-and-improved-embedding-model}{link}}). We evaluate the retrieval performance using NDCG~\citep{jarvelin2002cumulated}.
We train and evaluate our model on single-tool instructions (I1), intra-category multi-tool instructions (I2), and intra-collection multi-tool instructions (I3).



% \begin{table}[!t]
% \centering
% \resizebox{\textwidth}{!}{%
% \begin{tabular}{c|ccc|ccc|ccc}
% \hline
% \multicolumn{1}{c|}{\multirow{2}{*}{Instruction}} & \multicolumn{3}{c|}{\underline{API Retriever (ours)}} & \multicolumn{3}{c|}{\underline{BM25}} & \multicolumn{3}{c}{\underline{Ada Embedding}} \\
% \multicolumn{1}{c|}{} & NDCG1 & NDCG3 & NDCG5 & NDCG1 & NDCG3 & NDCG5 & NDCG1 & NDCG3 & NDCG5 \\
% \hline
% Single-tool (I1) & $88.83$ & $91.54$ & $92.31$ & $24.59$ & $20.66$ & $19.86$ & $64.49$ & $60.74$ & $60.49$ \\
% Category (I2) & $78.32$ & $86.31$ & $85.84$ & $15.46$ & $11.63$ & $10.20$ & $46.42$ & $34.98$ & $30.71$ \\
% Collection (I3) & $69.72$ & $80.68$ & $81.72$ & $26.38$ & $20.34$ & $17.08$ & $59.17$ & $49.72$ & $44.22$ \\
% All & $82.42$ & $87.06$ & $87.18$ & $19.78$ & $15.96$ & $14.83$ & $53.34$ & $46.72$ & $44.30$ \\
% \hline
% \end{tabular}%
% }
% \caption{
% \small{We compare our API retriever with two baselines (BM25 and Ada Embeddings) on single-tool, intra-category multi-tool, intra-collection multi-tool instructions, and the whole data,respectively.}
% }
% \label{tab:IR}
% \end{table}


\begin{table}[!t]
\centering
% Start the first minipage
\begin{minipage}[t]{0.57\linewidth}  % Adjust the width of this minipage as required
\centering
\resizebox{\textwidth}{!}{%
\begin{tabular}{ccccccccc}
\toprule
\multicolumn{1}{c}{\multirow{2}{*}{\textbf{Method}}} & \multicolumn{2}{c}{\underline{\textbf{I1}}} & \multicolumn{2}{c}{\underline{\textbf{I2}}} & \multicolumn{2}{c}{\underline{\textbf{I3}}} & \multicolumn{2}{c}{\underline{\textbf{Average}}} \\
\multicolumn{1}{c}{} & \multicolumn{2}{c}{\textbf{NDCG}} & \multicolumn{2}{c}{\textbf{NDCG}} & \multicolumn{2}{c}{\textbf{NDCG}} & \multicolumn{2}{c}{\textbf{NDCG}} \\
% NDCG5 & NDCG1 & NDCG5 & NDCG1 & NDCG5 
\multicolumn{1}{c}{} & \textbf{@1} & \textbf{@5} & \textbf{@1} & \textbf{@5} & \textbf{@1} & \textbf{@5} & \textbf{@1} & \textbf{@5} \\
\toprule
BM25 & $18.4$ & $19.7$ & $12.0$ & $11.0$ & $25.2$ & $20.4$ & $18.5$ & $17.0$ \\
Ada & $\underline{57.5}$ & $\underline{58.8}$ & $\underline{36.8}$ & $\underline{30.7}$ & $\underline{54.6}$  & $\underline{46.8}$ & $\underline{49.6}$ & $\underline{45.4}$ \\
Ours & $\textbf{84.2}$ & $\textbf{89.7}$ & $\textbf{68.2}$ & $\textbf{77.9}$ & $\textbf{81.7}$ & $\textbf{87.1}$ & $\textbf{78.0}$ &  $\textbf{84.9}$\\
\bottomrule
\end{tabular}%
}
\caption{
\small{Our API retriever v.s. two baselines for three types of instructions (I1, I2, I3). We report NDCG@1 and NDCG@5.}
}
\label{tab:IR}
\end{minipage}
\hfill  % Creates a space between the two minipages
% Start the second minipage
\begin{minipage}[t]{0.39\linewidth}  % Adjust the width of this minipage as required
\centering
\resizebox{\textwidth}{!}{%
\begin{tabular}{crrrr}
\toprule
Method                  & \underline{\textbf{I1}}    & \underline{\textbf{I2}}    & \underline{\textbf{I3}}    & \underline{\textbf{Average}}   \\ \toprule
ReACT                     & $37.8$ & $40.6$ & $27.6$ & $35.3$ \\
ReACT@N & $\underline{49.4}$ & $\underline{49.4}$ & $\underline{34.6}$ & $\underline{44.5}$ \\
\dfs           & $\textbf{58.0}$ & $\textbf{70.6}$ & $\textbf{62.8}$ & $\textbf{63.8}$ \\ \bottomrule
\end{tabular}
}
\caption{
\small{Pass rate of different reasoning strategies for three types of instructions (I1, I2, I3) based on \turbo. 
% ReACT@N and \dfs consume nearly the same budgets.
}
}
\label{tab:dfsdt_vs_react}
\end{minipage}
\end{table}

% \begin{table}[!t]
% \centering
% \resizebox{\textwidth}{!}{%
% \begin{tabular}{c|ccc|ccc|ccc}
% \hline
% \multicolumn{1}{c|}{\multirow{2}{*}{Instruction}} & \multicolumn{3}{c|}{\underline{API Retriever (ours)}} & \multicolumn{3}{c|}{\underline{BM25}} & \multicolumn{3}{c}{\underline{Ada Embedding}} \\
% \multicolumn{1}{c|}{} & NDCG1 & NDCG3 & NDCG5 & NDCG1 & NDCG3 & NDCG5 & NDCG1 & NDCG3 & NDCG5 \\
% \hline
% Single-tool (I1) & $84.20$ & $89.59$ & $89.65$ & $18.37$ & $17.97$ & $19.65$ & $57.52$ & $54.90$ & $58.83$ \\
% Category (I2) & $68.24$ & $77.43$ & $77.90$ & $11.97$ & $9.85$ & $10.95$ & $36.82$ & $28.83$ & $30.68$ \\
% Collection (I3) & $81.65$ & $87.24$ & $87.13$ & $25.23$ & $18.95$ & $20.37$ & $54.59$ & $42.55$ & $46.83$ \\
% % All & $75.73$ & $83.19$ & $83.06$ & $15.84$ & $13.98$ & $15.63$ & $46.59$ & $41.06$ & $43.95$ \\
% \hline
% \end{tabular}%
% }
% \caption{
% \small{We compare our API retriever with two baselines (BM25 and Ada Embeddings) on single-tool, intra-category multi-tool, and intra-collection multi-tool instructions, respectively.}
% }
% \label{tab:IR}
% \end{table}

% \begin{table}[!t]
% \centering
% \small
% \begin{tabular}{lcccc}
% \hline
% Method                  & Single-tool (I1)    & Category (I2)    & Collection (I3)    & Average   \\ \hline
% ReACT                     & $43.98$ & $23.62$ & $20.42$ & $29.34$ \\
% ReACT@N & $50.80$ & $36.14$ & $32.87$ & $39.94$ \\
% \dfs           & $\textbf{54.10}$ & $\textbf{47.35}$ & $\textbf{44.80}$ & $\textbf{48.75}$ \\ \hline
% \end{tabular}
% \caption{
% \small{Pass rate of different reasoning strategies for three types of instructions (I1, I2, I3) based on \turbo. ReACT@N and \dfs consume nearly the same OpenAI API calls per instruction.}
% }
% \label{tab:dfsdt_vs_react}
% \end{table}

As shown in Table~\ref{tab:IR}, our API retriever consistently outperforms baselines across all settings, indicating its feasibility in real-world scenarios with massive APIs. Also, the NDCG score of I1 is generally higher than I2 and I3, which means single-tool instruction retrieval is simpler than multi-tool setting.

\textbf{Superiority of \dfs over ReACT} \quad
Before solution path annotation, we validate the efficacy of \dfs.
Based on \turbo, we compare \dfs and ReACT using the pass rate metric. Since \dfs consumes more OpenAI API calls than ReACT, for a fairer comparison, we also establish a ``ReACT@N'' baseline, which conducts multiple times of ReACT until the total costs reach the same level of \dfs. 
Once a valid solution is found by ReACT@N, we deem it a pass.

From Table~\ref{tab:dfsdt_vs_react}, it can be observed that \dfs significantly outperforms the two baselines in all scenarios. Since we only retain those passed annotations as the training data, given the same budgets, using \dfs could annotate more instructions. This makes \dfs a more efficient way that saves the total annotation cost.
We also find that the performance improvement of \dfs is more evident for harder instructions (i.e., I2 and I3) than those simpler instructions (I1). This means that by expanding the search space, \dfs can better solve those difficult, complex instructions that are unanswerable by the vanilla ReACT no matter how many times it is performed. Involving such ``hard examples'' in our dataset can fully elicit the tool-use capabilities for those complex scenarios.

\subsection{Main Experiments}
\label{sec:main_exp}

\textbf{ToolLLaMA} \quad
We fine-tune LLaMA-2 7B model~\citep{touvron2023llama2} using the instruction-solution pairs. The original LLaMA-2 model has a sequence length of $4096$, which is not enough under our setting since the API response can be very long. To this end, we use positional interpolation~\citep{chen2023extending} to extend the context length to $8192$ (training details in \cref{details_training_toolllama}).

\textbf{Settings} \quad
Ideally, by scaling the number and diversity of instructions and unique tools in the training data, \ourmodel is expected to generalize to new instructions and APIs unseen during training. This is meaningful since users can define customized APIs and expect \ourmodel to adapt according to the documentation.
To this end, we strive to evaluate the \textbf{generalization ability} of \ourmodel at three levels: (1) \textbf{Inst.}: \textbf{unseen instructions} for the same set of tools in the training data, (2) \textbf{Tool}: \textbf{unseen tools} that belong to the \textbf{same (seen) category} of the tools in the training data, and (3) \textbf{Cat.}: \textbf{unseen tools} that belong to a \textbf{different (unseen) category} of tools in the training data.

We perform experiments on three scenarios: single-tool instructions (I1), intra-category multi-tool instructions (I2), and intra-collection multi-tool instructions (I3). For I1, we conduct the evaluation for the aforementioned three levels (I1-Inst., I1-Tool, and I1-Cat.); for I2, since the training instructions already involve different tools of the same category, we only perform level 1 and level 3 for the generalization evaluation (I2-Inst. and I2-Cat.); similarly, we only perform level 1 generalization for I3 (I3-Inst.) since it already covers instructions that involve various combinations of tools from different categories (the tools in a RapidAPI collection may come from different RapidAPI categories).
For each test instruction, we feed the ground-truth (oracle) APIs $\sS_\text{N}^{\text{sub}}$ to each model. This simulates the scenario where the user specifies the API set they prefer.
% After that, we incorporate the documentation of each API into the input prompt of different models. The models would perform multiple rounds of reasoning and API call and ultimately derive the final response.
% For I1 and I2, we randomly select $6$ categories as the testing categories, leaving the remaining $43$ categories for training.

% Specifically, in the I1 scenario, where only one tool is used in one instance, we adopt an 8:1:1 ratio to divide the tools within each training category into training, validation, and testing tools. 
% Subsequently, the instances in each training tool is further split in a 1:1:1 ratio to form the training, validation, and testing instances. 
% The \textit{I1-training set} is composed by gathering the training instances from all training tools across all training categories. 
% To test the instruction generalization ability of Tool-LLaMA, we construct \textit{I1-instruction-generalization} test set, which is formed by sampling 100 testing instances from the training tools within the training categories.
% To test the model's generalization abilities on unseen tools from seen categories, we construct the \textit{I1-tool-generalization} test set which consists of 100 instances from the testing tools within the training categories. 
% And to test on unseen tools from unseen categories, we construct \textit{I1-category-generalization} test set, which comprises 100 testing instances from the testing categories.

% In the I2 scenario, where several tools within the same category are used to construct an instance, we opt not to distinguish between training, validation, and testing tools in the training categories.
% Instead, we merge all data from the training categories and split it in a 30:1 ratio to create the training and validation sets. 
% For the test set in this scenario, we sample 100 instances from the testing categories, referred to as the \textit{I2-category-generalization} test set.
% In I3, since each instance includes APIs from different tools and categories, we do not designate specific training and testing categories. Instead, we directly split all the instances approximately in an 8:1:1 ratio to construct the training, validation, and test set. We consider the test set as \textit{I3-instruction-generalization} test set
\textbf{Baselines} \quad
We choose two LLaMA variants that have been fine-tuned for general-purpose dialogue, i.e., Vicuna~\citep{vicuna2023} and Alpaca~\citep{alpaca}. 
We also choose the ``teacher model'' \turbo, Text-Davinci-003, GPT-4, and Claude-2 as baselines, and apply both \dfs and ReACT to them. When calculating the win rate, each model is compared with \turbo-ReACT.
% We conduct sophisticated prompt engineering for both models to elicit the best of their tool-use abilities.
% Since the original LLaMA checkpoint is not fine-tuned toward any downstream task, it cannot be leveraged to use tools directly. 
% Both models have shown strong instruction-following capabilities.

\definecolor{themegreen}{HTML}{C5E0B4}
\definecolor{themeblue}{HTML}{CFE2F3}
\definecolor{themeyellow}{HTML}{FFE699}
\definecolor{themedarkyellow}{HTML}{FFBF87}

\newcommand{\femph}[1]{\cellcolor[HTML]{C5E0B4}{#1}}
\newcommand{\semph}[1]{\cellcolor[HTML]{CFE2F3}{#1}}

\newcommand{\crect}[1]{{\begin{tikzpicture}
\node[rectangle,
    draw = themeyellow,
    fill = themedarkyellow,
    inner sep=0pt,
    line width = 0.03cm,
    minimum width = #1 cm, 
    minimum height = 0.25 cm] at (0,0) {};
\end{tikzpicture}} 
}

\begin{table}[!t]
    \centering
    \resizebox{\textwidth}{!}{%
    \begin{tabular}{cc|rr|rr|rr|rr|rr|rr|rr}
    % {c@{~~~}c@{~~~}r@{~~~}r@{~~~}r@{~~~}r@{~~~}r@{~~~}r@{~~~}r@{~~~}r@{~~~}r@{~~~}r@{~~~}r@{~~~}r@{~~~}r@{~~~}r}
    \toprule
    \multicolumn{1}{c}{\multirow{2}{*}{\textbf{Model}}} & \multicolumn{1}{c|}{\multirow{2}{*}{\textbf{Method}}} &\multicolumn{2}{c|}{\underline{\textbf{I1-Inst.}}} & \multicolumn{2}{c|}{\underline{\textbf{I1-Tool}}} & \multicolumn{2}{c|}{\underline{\textbf{I1-Cat.}}} & \multicolumn{2}{c|}{\underline{\textbf{I2-Inst.}}} & \multicolumn{2}{c|}{\underline{\textbf{I2-Cat.}}} & \multicolumn{2}{c|}{\underline{\textbf{I3-Inst.}}} & \multicolumn{2}{c}{\underline{\textbf{Average}}} \\
    \multicolumn{1}{c}{} & \multicolumn{1}{c|}{} & Pass & Win & Pass & Win & Pass & Win & Pass & Win & Pass & Win & Pass & Win & Pass & Win \\
    \toprule
    % \ourmodel & $75.0$ & $\textbf{50.0}$ & $68.0$ & $45.0$ & $80.0$ & $45.0$ & $56.0$ & $\textbf{59.0}$ & $47.0$ & $48.0$ & $40.0$ & $46.0$ & $61.0$ & $48.8$ \\
    % \midrule
    ChatGPT & ReACT & $41.5$ & - & $44.0$ & - & $44.5$ & - & $42.5$ & - & $46.5$ & - & $22.0$ & - & $40.2$ & -  \\
    & DFSDT & $54.5$ & $60.5$ & $\underline{65.0}$ & $\underline{62.0}$ & $60.5$ & $57.3$ & $75.0$ & $\underline{72.0}$ & $71.5$ & $\textbf{64.8}$ & $62.0$ & $69.0$ & $64.8$ & $64.3$  \\
    % \hline
    Claude-2 & ReACT & $5.5$ & $31.0$ & $3.5$ & $27.8$ & $5.5$ & $33.8$ & $6.0$ & $35.0$ & $6.0$ & $31.5$ & $14.0$ & $47.5$ & $6.8$ & $34.4$  \\
    & DFSDT & $20.5$ & $38.0$ & $31.0$ & $44.3$ & $18.5$ & $43.3$ & $17.0$ & $36.8$ & $20.5$ & $33.5$ & $28.0$ & $65.0$ & $22.6$ & $43.5$  \\
    % \hline
    Text-Davinci-003 & ReACT & $12.0$ & $28.5$ & $20.0$ & $35.3$ & $20.0$ & $31.0$ & $8.5$ & $29.8$ & $14.5$ & $29.8$ & $24.0$ & $45.0$ & $16.5$ & $33.2$  \\
    & DFSDT & $43.5$ & $40.3$ & $44.0$ & $43.8$ & $46.0$ & $46.8$ & $37.0$ & $40.5$ & $42.0$ & $43.3$ & $46.0$ & $63.0$ & $43.1$ & $46.3$  \\
    % \hline
    GPT4 & ReACT & $53.5$ & $60.0$ & $50.0$ & $58.8$ & $53.5$ & $\underline{63.5}$ & $67.0$ & $65.8$ & $72.0$ & $60.3$ & $47.0$ & $\underline{78.0}$ & $57.2$ & $\underline{64.4}$  \\
    & DFSDT & $\underline{60.0}$ & $\textbf{67.5}$ & $\textbf{71.5}$ & $\textbf{67.8}$ & $\textbf{67.0}$ & $\textbf{66.5}$ & $\underline{79.5}$ & $\textbf{73.3}$ & $\textbf{77.5}$ & $\underline{63.3}$ & $\textbf{71.0}$ & $\textbf{84.0}$ & $\textbf{71.1}$ & $\textbf{70.4}$  \\
    \midrule
    Vicuna & ReACT \& DFSDT & $0.0$ & 0.0 & $0.0$ & 0.0 & $0.0$ & 0.0 & $0.0$ & 0.0 & $0.0$ & 0.0 & $0.0$ & 0.0 & $0.0$ & 0.0  \\
    % \hline
    Alpaca & ReACT \& DFSDT & $0.0$ & 0.0 & $0.0$ & 0.0 & $0.0$ & 0.0 & $0.0$ & 0.0 & $0.0$ & 0.0 & $0.0$ & 0.0 & $0.0$ & 0.0  \\
    % \midrule
     & ReACT & $25.0$ & $45.0$ & $29.0$ & $42.0$ & $33.0$ & $47.5$ & $30.5$ & $50.8$ & $31.5$ & $41.8$ & $25.0$ & $55.0$ & $29.0$ & $47.0$\\
    ToolLLaMA & DFSDT & $57.0$ & $55.0$ & $61.0$ & $55.3$ & $\underline{62.0}$ & $54.5$ & $77.0$ & $68.5$ & $\underline{77.0}$ & $58.0$ & $\underline{66.0}$ & $69.0$ & $66.7$ & $60.0$ \\
    & DFSDT-Retriever  & $\textbf{64.0}$ & $\underline{62.3}$ & $64.0$ & $59.0$ & $60.5$ & $55.0$ & $\textbf{81.5}$ & $68.5$ & $68.5$ & $60.8$ & $65.0$ & $73.0$ & $\underline{67.3}$ & $63.1$ \\
    
    \bottomrule
    \end{tabular}%
    }
    \caption{
    \small{Main experiments of \ourdata. Win rate is calculated by comparing each model with ChatGPT-ReACT. A win rate higher than $50\%$ means the model performs better than ChatGPT-ReACT. Apart from ToolLLaMA-DFSDT-Retriever, all methods use the oracle API retriever (i.e., ground truth API).
    }
    }
    \label{tab:main_exp}
    \end{table}

% , except in the case of ToolLLaMA-DFSDT-Retriever and ToolLLaMA-ReACT, which are compared with ToolLLaMA-DFSDT

\textbf{Main Results} \quad
The results are placed in Table~\ref{tab:main_exp}, from which we derive that: 
\begin{enumerate}[topsep=1pt, partopsep=1pt, leftmargin=12pt, itemsep=-1pt]
    \item Although we conduct prompt engineering extensively, both Vicuna and Alpaca fail to pass any instruction (pass rate \& win rate = 0), which means their instruction-following abilities do not cover the tool-use domain. This underscores \textbf{the deficiency of current instruction tuning attempts}, which largely focus on language skills;
    \item For all LLMs, using \dfs significantly outperforms ReACT in both pass rate and win rate. Notably, \turbo+\dfs surpasses GPT-4+ReACT in pass rate and performs comparably in win rate. This underscores \textbf{the superiority of \dfs over ReACT} in decision-making;
    \item When using DFSDT, \ourmodel performs much better than Text-Dainci-003 and Claude-2, and achieves a result almost on par with \turbo (the teacher model). In general, despite generalizing to unseen instructions and tools, \ourmodel+\dfs demonstrates \textbf{competitive generalization performance} in all scenarios, achieving a pass rate second to GPT4+\dfs.
\end{enumerate}
Overall, these results demonstrate that \ourdata can sufficiently elicit the tool-use capabilities within LLMs and empower them to skillfully master even unseen APIs for various instructions.

% From Table~\ref{tab:main_exp}, it can be derived that \dfs achieves a significantly higher pass rate and is more preferred across all the scenarios.
% Besides, compared with the results in Table~\ref{tab:dfsdt_vs_react}, we find that the improvements brought by \dfs over ReACT are more evident for \ourmodel than \turbo, which demonstrates that expanding the search space is more important for LLMs with inferior reasoning capabilities. This finding reveals the potential of applying \dfs to small-scale models in practice.





% For the win rate, \ourmodel generally matches ChatGPT+\dfs's capability and even surpasses the latter in the \textbf{I2-Cat.} setting. 

% \begin{table}[!t]
% \centering
% \resizebox{\textwidth}{!}{%
% \begin{tabular}{l|cc|cc|cc|cc|cc|cc|cc}
% \toprule
% \multicolumn{1}{c}{\multirow{2}{*}{Model}} & \multicolumn{2}{c|}{\textbf{I1-Inst.}} & \multicolumn{2}{c|}{\textbf{I1-Tool}} & \multicolumn{2}{c|}{\textbf{I1-Cat.}} & \multicolumn{2}{c|}{\textbf{I2-Inst.}} & \multicolumn{2}{c|}{\textbf{I2-Cat.}} & \multicolumn{2}{c|}{\textbf{I3-Inst.}} & \multicolumn{2}{c}{\textbf{Average}} \\
% \multicolumn{1}{c|}{} & Pass & Win & Pass & Win & Pass & Win & Pass & Win & Pass & Win & Pass & Win & Pass & Win \\
% \toprule
% ToolLLaMA & $\underline{57.0}$ & 0.0 & $\underline{61.0}$ & 0.0 & $\textbf{62.0}$ & 0.0 & $\underline{77.0}$ & 0.0 & $\textbf{77.0}$ & 0.0 & $\textbf{66.0}$ & 0.0 & $\underline{66.7}$ & - \\
% \midrule
% $\rightarrow$API Retriever  & $\textbf{64.0}$ & $47.0$ & $\textbf{64.0}$ & $42.5$ & $\underline{60.5}$ & $45.0$ & $\textbf{81.5}$ & $43.0$ & $\underline{68.5}$ & $45.0$ & $\underline{65.0}$ & $45.0$ & $\textbf{67.3}$ & $44.6$ \\
% $\rightarrow$ReACT & $25.0$ & $33.0$ & $29.0$ & $32.5$ & $33.0$ & $31.5$ & $30.5$ & $25.5$ & $31.5$ & $26.5$ & $25.0$ & $29.0$ & $29.0$ & $29.7$\\
% % $\rightarrow$LoRA & $51.0$ & $\underline{34.0}$ & $\underline{63.0}$ & $\textbf{44.0}$ & $61.0$ & $\underline{39.0}$ & $38.0$ & $\underline{38.0}$ & $42.0$ & $\underline{42.0}$ & $\underline{45.0}$ & $\textbf{54.0}$ & $50.0$ & $\underline{41.8}$\\

% \bottomrule
% \end{tabular}%
% }
% \caption{
% \small{Additional analyses of \ourmodel: (1) replacing the ground truth APIs with those recommended by our API retriever and (2) degrading the reasoning method from \dfs to ReACT. We compare each variant with the default \ourmodel for win rate.}
% }
% \label{tab:sub_exp}
% \end{table}

\textbf{Integrating API Retriever with \ourmodel} \quad
In real-world scenarios, asking users to manually recommend APIs from a large pool may not be practical.
%users may not be able to manually recommend APIs from a large pool. 
To emulate this practical setting and test the efficiency of our API retriever, we feed the top $5$ APIs (instead of the ground truth APIs $\sS_\text{N}^{\text{sub}}$) recommended by our API retriever to \ourmodel. As shown in Table~\ref{tab:main_exp}, using retrieved APIs even improves the performance (both pass rate and win rate) compared to the ground truth API set.
This is because many APIs in the ground truth API set can be replaced by other similar APIs with better functionalities, which our API retriever can successfully identify. In other words, \textbf{our retriever expands the search space of relevant APIs and finds more appropriate ones for the current instruction}.
It provides robust evidence of the excellent ability of our API retriever to retrieve relevant APIs, especially considering the vast pool ($16,000$+) of APIs from which our API retriever selects. 


\subsection{Out-of-Distribution (OOD) Generalization to APIBench~\citep{patil2023gorilla}}
\label{sec:ood_exp}

\textbf{Settings} \quad
We further extend \ourmodel to an OOD dataset APIBench to validate its generalization ability. To assess the generalization ability of ToolLLaMA in these new domains, we equip \ourmodel with two retrievers: our trained API retriever and the oracle retriever. We evaluate three domains of APIBench, i.e., TorchHub, TensorHub, and HuggingFace. We compare \ourmodel with Gorilla, a LLaMA-7B model fine-tuned using the training data of APIBench. Following the original paper, we adopt two official settings for Gorilla: zero-shot setting (ZS) and retrieval-aware setting (RS). The latter means (RS) the retrieved APIs are sent to the model as part of the prompts; while the former (ZS) does not incorporate the APIs in the prompts when training the model. We adopt the official evaluation metric and report the AST accuracy along with the hallucination rates.

\textbf{Results} \quad
The results are shown in Table~\ref{gorilla-results}. In general, \ourmodel achieves \textbf{remarkable OOD generalization performance} on all three datasets, despite being trained on a completely different API domain and instruction domain. Specifically, ToolLLaMA+our API retriever outperforms Gorilla+BM25 from both training settings (ZS / RS) in terms of AST accuracy on HuggingFace and TorchHub. With the same oracle retriever, ToolLLaMA is consistently superior when compared to Gorilla-ZS. It should be noted that Gorilla model cannot be generalized to our ToolBench dataset due to our more complex settings, such as the multi-tool use and multi-step reasoning.

% also achieves lower hallucination rates than Gorilla-ZS on all three datasets. When the oracle retriever is employed by both models, ToolLLaMA 

\begin{table}[!t]
\centering
\small
\begin{tabular}{c@{~~~}r@{~~~}r@{~~~}r@{~~~}r@{~~~}r@{~~~}r}
\midrule
\multicolumn{1}{c}{\multirow{2}{*}{Method}} & \multicolumn{2}{c} {\underline{HuggingFace}} & \multicolumn{2}{c}{\underline{TorchHub}} & \multicolumn{2}{c}{\underline{TensorHub}} \\
\multicolumn{1}{c}{} & Hallu. ($\downarrow$) & AST ($\uparrow$) & Hallu. ($\downarrow$) & AST ($\uparrow$) & Hallu. ($\downarrow$) & AST ($\uparrow$) \\
\midrule 
ToolLLaMA + Our Retriever & \underline{10.60} &	\textbf{16.77} &	\underline{15.70} &	\textbf{51.16} &	\underline{6.48} &	\underline{40.59} \\
Gorilla-ZS + BM25 &	46.90 &	10.51 &	17.20 &	44.62 &	20.58 &	34.31 \\
Gorilla-RS + BM25 & \textbf{6.42} & \underline{15.71} & \textbf{5.91} & \underline{50.00} & \textbf{2.77} & \textbf{41.90} \\
\midrule
ToolLLaMA + Oracle & \underline{8.66} &	\underline{88.80} &	\underline{14.12} &	\underline{85.88} &	\underline{7.44} &	\underline{88.62} \\
Gorilla-ZS + Oracle &	52.88 &	44.36 &	39.25 &	59.14 &	12.99 &	83.21 \\
Gorilla-RS + Oracle &	\textbf{6.97} &	\textbf{89.27} &	\textbf{6.99} &	\textbf{93.01} &	\textbf{2.04} &	\textbf{94.16} \\
% ChatGPT-DFSDT + Oracle &	\textbf{4.20} &	\textbf{93.25} &	\textbf{0.00} &	\textbf{100.00} &	\textbf{1.61} &	\textbf{94.45} \\
\hline
\end{tabular}
\caption{
\small{OOD generalization experiments on APIBench. For the Gorilla entries, ZS / RS means that Gorilla was trained in a zero-shot / retrieval-aware setting on APIBench. We report hallucination rate and AST accuracy.}}
\label{gorilla-results}
\end{table}
% We follow the original paper to include BM25 here.
\section{Related work}
\label{related}
%https://arxiv.org/pdf/2304.08465.pdf
%https://arxiv.org/pdf/2305.16807.pdf
% https://arxiv.org/pdf/2212.04489.pdf
%(1) try adding two z and correlation, (2) try mu, mu_hat with 30 images

% One of the most fundamental challenges in generative models is to work on real images. The efforts are focused mainly on finding the random images that create the generated image as mentioned in Section~\ref{sec:intro} as well as on edit while maintaining the input content/identity. In this section, we present how other methods tackle these problems problem.

% We divide this section into two: applications that use the DDPM sampling and the ones that use DDIM sampling. 

\subsection{Inversion of diffusion models}
Editing a real image using diffusion models requires extracting the noise vectors that would generate that image when used within the generative process. %The DDPM sampling is stochastic, therefore, it is not trivial to invert this model.
The vast majority of diffusion-based editing works use the DDIM scheme, which is a deterministic mapping from a single noise map to a generated image~\cite{Hertz22,Mokady22,Narek22,Guillaume22,Bram22,Parmar23,Adham22}. The original DDIM paper~\cite{Song21} suggested an efficient approximate inversion for that scheme. This method incurs a small error at every diffusion timestep, and these errors often accumulate into meaningful deviations when using classifier-free guidance~\cite{Ho21}. Mokady \etal~\cite{Mokady22} improve the reconstruction quality by fixing each timestep drifting. Their two-step approach first uses DDIM inversion to compute a sequence of noise vectors, and then uses this sequence to optimize the input null-text embedding at every timestep. Miyake \etal~\cite{miyake23}
achieve similar reconstruction accuracy through forward propagation without optimization, thereby enabling much faster editing processes. An improvement in the reconstruction quality was suggested by Han \etal~\cite{Han23} that integrate a regularization term into the null-text embedding optimization. % Proximal guidance was introduced by  
EDICT~\cite{Bram22} enables mathematically exact DDIM-inversion of real images by maintaining two coupled noise vectors which are used to invert each other in an alternating fashion. This method doubles the computation time of the diffusion process. %and cannot produce multiple results.
CycleDiffusion~\cite{Wu22} presents a DDPM-inversion method by recovering a sequence of noise vectors that perfectly reconstruct the image within the DDPM sampling process. As opposed to our method, their extracted noise maps are distributed like the native noise space of DDPM, which results in limited editing capabilities (see Figs.~\ref{fig:generated_vs_us},\ref{fig:cyclediffusion_vs_us}).

%\inbar{Direct Inversion: Optimization-Free Text-Driven Real Image Editing with Diffusion Models~\cite{Adham22}. Not sure we should cite them} \tomer{Cite them at the beginning of the paragraph where I left a comment (in addition to many other papers that use DDIM inversion)}

% edit a real image while keeping it as faithful as possible to the real content (structure/identity).

\subsection{Image editing using diffusion models}
% DDPM - interpolation+SDEdit
The DDPM sampling method is not popular for editing of real images. When used, it is typically done without exact inversion. %Ho et al.~\cite{Ho20} and Meng et al.~\cite{Meng22} edit images without inversion. 
Two examples are Ho \etal~\cite{Ho20}, who interpolate between real images, and Meng \etal~\cite{Meng22} who edit real images via user sketches or strokes (SDEdit). Both construct a noisy version of the real image and apply a backward diffusion after editing. They suffer from an inherent tradeoff between the realism of the generated image and its faithfulness to the original contents. %The balance between the two is controlled by the variance of the noise added to the image. 
DiffuseIT~\cite{Gihyun22} performs image translation guided by a reference image or by text, also without explicit inversion. They guide the generation process by losses that measure similarity to the original image~\cite{Prafulla21}.

% Text driven
A series of papers apply text-driven image-to-image translation using DDIM inversion. Narek \etal~\cite{Narek22} and Cao \etal~\cite{cao23} achieve this by manipulating spatial features and their self-attention inside the model during the diffusion process. Hertz \etal~\cite{Hertz22} change the attention maps of the original image according to the target text prompt and inject them into the diffusion process. DiffEdit~\cite{Guillaume22} automatically generates a mask for the regions of the image that need to be edited, based on source and target text prompts. This is used to enforce the faithfulness of the unedited regions to the original image, in order to battle the poor reconstruction quality obtained from the inversion. This method fails to predict accurate masks for complex prompts.%and/or very different from one another. 

% optimization methods
Some methods utilize model optimization based on the target text prompt.  DiffusionCLIP \cite{Kim22} uses model fine-tuning based on a CLIP loss with a target text. Imagic \cite{Bahjat22} first optimizes the target text embedding, and then optimizes the model to reconstruct the image with the optimized text embedding. %In the last part, they freeze the embedding and the model and use the diffusion process with text interpolation between the optimized one and the target one. 
UniTune~\cite{Valevski22} also uses fine-tuning and shows great success in making global stylistic changes and complex local edits while maintaining image structure. Other works like Palette~\cite{Saharia21} and InstructPix2Pix~\cite{Brooks22}, learn conditional diffusion models tailored for specific editing tasks. %The high computational cost of fine-tuning a diffusion model for a task or over an input image, however, makes it impractical as an interactive image editing tool. %Since our method is training-free, we do not view them as competitive algorithms.


\begin{figure}
%\vskip 0.2in
% \begin{center}
\includegraphics[width=\columnwidth]{ICCV23_submission/figures/DDPM_sampling_process.pdf}
\caption{\textbf{The DDPM latent noise space.} In DDPM, the generative (reverse) diffusion process synthesizes an image $x_0$ in $T$ steps, by utilizing $T+1$ noise maps, $\{x_T,z_T,\ldots,z_1\}$. We regard those noise maps as the latent code associated with the generated image.}
\label{fig:unfold}
% \end{center}
%\vskip -0.2in
\end{figure}


% The applications of inpaiting~\cite{Lugmayr22} and local edit via text ~\cite{avrahami22}, change only the pixels inside a pre-defined mask. Therefore, the content of the pixels outside the mask remains unaffected and they achieve both meaningful editing and background preservation. In these cases, a mask must be provided as input to tell the diffusion model what parts of the image should be edited. 


%\inbar{add Dall-E-2 - also editing}

% DDIM papers + inversion:
% - DiffusionCLIP

% DDIM papers + mask:
% - DIFFEDIT: DIFFUSION-BASED SEMANTIC IMAGE EDITING WITH MASK GUIDANCE
% DDIM papers + optimization:
% - Imagic (read about TEdBench)
% - DiffusionCLIP
% - Diffusion Autoencoders - should add?

% Not real images:
% - ILVR: Conditioning Method for Denoising Diffusion Probabilistic Models. Guided generating using a reference image- should add?
%\inbar{
%- GLIDE?
%- concurrent work: Zero-shot Image-to-Image Translation, MagicMix, FlexIt, DreamBooth, Eliminating Prior Bias for Semantic Image Editing
%via Dual-Cycle Diffusion, Asyrp?
%}

% Also, as the Prompt-to-Prompt technique
% requires fixing the attention weights, it is restricted to localized edits, and supports only a limited set
% of edit operations (adding or changing a word). Our method is inspired by Prompt-to-Prompt, and
% relaxes those restrictions.


\section{Conclusion}
\label{conclusion}

We presented an inversion method for DDPM. Our noise maps encode the image structure more strongly than the noise maps in regular sampling, and are therefore better suited for image editing. We illustrated their advantages in text-based editing, both when used by themselves and in combination with other editing methods. 

% \clearpage
%In contrast to DDIM inversion, using our approach allows generating diverse results for any given image and text.

%this inversion produces edit firendly noise maps that simple manipulation over them translates to meaningful manipulations of the generated image.

%When fixing the noise maps while changing the condition of the model, i.e., text, the generated image is modified semantically but retains the structure. This property enables simple and fast text-guided editing. We also show that existing editing methods can change their DDIM inversion with ours and benefit from a quality improvement.

%Our inversion enable editing real images, without any optimization or model fine-tuning.




% \section{Default Notation}

% \centerline{\bf Numbers and Arrays}
% \bgroup
% \def\arraystretch{1.5}
% \begin{tabular}{p{1in}p{3.25in}}
% $\displaystyle a$ & A scalar (integer or real)\\
% $\displaystyle \va$ & A vector\\
% $\displaystyle \mA$ & A matrix\\
% $\displaystyle \tA$ & A tensor\\
% $\displaystyle \mI_n$ & Identity matrix with $n$ rows and $n$ columns\\
% $\displaystyle \mI$ & Identity matrix with dimensionality implied by context\\
% $\displaystyle \ve^{(i)}$ & Standard basis vector $[0,\dots,0,1,0,\dots,0]$ with a 1 at position $i$\\
% $\displaystyle \text{diag}(\va)$ & A square, diagonal matrix with diagonal entries given by $\va$\\
% $\displaystyle \ra$ & A scalar random variable\\
% $\displaystyle \rva$ & A vector-valued random variable\\
% $\displaystyle \rmA$ & A matrix-valued random variable\\
% \end{tabular}
% \egroup
% \vspace{0.25cm}

% \centerline{\bf Sets and Graphs}
% \bgroup
% \def\arraystretch{1.5}

% \begin{tabular}{p{1.25in}p{3.25in}}
% $\displaystyle \sA$ & A set\\
% $\displaystyle \R$ & The set of real numbers \\
% $\displaystyle \{0, 1\}$ & The set containing 0 and 1 \\
% $\displaystyle \{0, 1, \dots, n \}$ & The set of all integers between $0$ and $n$\\
% $\displaystyle [a, b]$ & The real interval including $a$ and $b$\\
% $\displaystyle (a, b]$ & The real interval excluding $a$ but including $b$\\
% $\displaystyle \sA \backslash \sB$ & Set subtraction, i.e., the set containing the elements of $\sA$ that are not in $\sB$\\
% $\displaystyle \gG$ & A graph\\
% $\displaystyle \parents_\gG(\ervx_i)$ & The parents of $\ervx_i$ in $\gG$
% \end{tabular}
% \vspace{0.25cm}


% \centerline{\bf Indexing}
% \bgroup
% \def\arraystretch{1.5}

% \begin{tabular}{p{1.25in}p{3.25in}}
% $\displaystyle \eva_i$ & Element $i$ of vector $\va$, with indexing starting at 1 \\
% $\displaystyle \eva_{-i}$ & All elements of vector $\va$ except for element $i$ \\
% $\displaystyle \emA_{i,j}$ & Element $i, j$ of matrix $\mA$ \\
% $\displaystyle \mA_{i, :}$ & Row $i$ of matrix $\mA$ \\
% $\displaystyle \mA_{:, i}$ & Column $i$ of matrix $\mA$ \\
% $\displaystyle \etA_{i, j, k}$ & Element $(i, j, k)$ of a 3-D tensor $\tA$\\
% $\displaystyle \tA_{:, :, i}$ & 2-D slice of a 3-D tensor\\
% $\displaystyle \erva_i$ & Element $i$ of the random vector $\rva$ \\
% \end{tabular}
% \egroup
% \vspace{0.25cm}


% \centerline{\bf Calculus}
% \bgroup
% \def\arraystretch{1.5}
% \begin{tabular}{p{1.25in}p{3.25in}}
% % NOTE: the [2ex] on the next line adds extra height to that row of the table.
% % Without that command, the fraction on the first line is too tall and collides
% % with the fraction on the second line.
% $\displaystyle\frac{d y} {d x}$ & Derivative of $y$ with respect to $x$\\ [2ex]
% $\displaystyle \frac{\partial y} {\partial x} $ & Partial derivative of $y$ with respect to $x$ \\
% $\displaystyle \nabla_\vx y $ & Gradient of $y$ with respect to $\vx$ \\
% $\displaystyle \nabla_\mX y $ & Matrix derivatives of $y$ with respect to $\mX$ \\
% $\displaystyle \nabla_\tX y $ & Tensor containing derivatives of $y$ with respect to $\tX$ \\
% $\displaystyle \frac{\partial f}{\partial \vx} $ & Jacobian matrix $\mJ \in \R^{m\times n}$ of $f: \R^n \rightarrow \R^m$\\
% $\displaystyle \nabla_\vx^2 f(\vx)\text{ or }\mH( f)(\vx)$ & The Hessian matrix of $f$ at input point $\vx$\\
% $\displaystyle \int f(\vx) d\vx $ & Definite integral over the entire domain of $\vx$ \\
% $\displaystyle \int_\sS f(\vx) d\vx$ & Definite integral with respect to $\vx$ over the set $\sS$ \\
% \end{tabular}
% \egroup
% \vspace{0.25cm}

% \centerline{\bf Probability and Information Theory}
% \bgroup
% \def\arraystretch{1.5}
% \begin{tabular}{p{1.25in}p{3.25in}}
% $\displaystyle P(\ra)$ & A probability distribution over a discrete variable\\
% $\displaystyle p(\ra)$ & A probability distribution over a continuous variable, or over
% a variable whose type has not been specified\\
% $\displaystyle \ra \sim P$ & Random variable $\ra$ has distribution $P$\\% so thing on left of \sim should always be a random variable, with name beginning with \r
% $\displaystyle  \E_{\rx\sim P} [ f(x) ]\text{ or } \E f(x)$ & Expectation of $f(x)$ with respect to $P(\rx)$ \\
% $\displaystyle \Var(f(x)) $ &  Variance of $f(x)$ under $P(\rx)$ \\
% $\displaystyle \Cov(f(x),g(x)) $ & Covariance of $f(x)$ and $g(x)$ under $P(\rx)$\\
% $\displaystyle H(\rx) $ & Shannon entropy of the random variable $\rx$\\
% $\displaystyle \KL ( P \Vert Q ) $ & Kullback-Leibler divergence of P and Q \\
% $\displaystyle \mathcal{N} ( \vx ; \vmu , \mSigma)$ & Gaussian distribution %
% over $\vx$ with mean $\vmu$ and covariance $\mSigma$ \\
% \end{tabular}
% \egroup
% \vspace{0.25cm}

% \centerline{\bf Functions}
% \bgroup
% \def\arraystretch{1.5}
% \begin{tabular}{p{1.25in}p{3.25in}}
% $\displaystyle f: \sA \rightarrow \sB$ & The function $f$ with domain $\sA$ and range $\sB$\\
% $\displaystyle f \circ g $ & Composition of the functions $f$ and $g$ \\
%   $\displaystyle f(\vx ; \vtheta) $ & A function of $\vx$ parametrized by $\vtheta$.
%   (Sometimes we write $f(\vx)$ and omit the argument $\vtheta$ to lighten notation) \\
% $\displaystyle \log x$ & Natural logarithm of $x$ \\
% $\displaystyle \sigma(x)$ & Logistic sigmoid, $\displaystyle \frac{1} {1 + \exp(-x)}$ \\
% $\displaystyle \zeta(x)$ & Softplus, $\log(1 + \exp(x))$ \\
% $\displaystyle || \vx ||_p $ & $\normlp$ norm of $\vx$ \\
% $\displaystyle || \vx || $ & $\normltwo$ norm of $\vx$ \\
% $\displaystyle x^+$ & Positive part of $x$, i.e., $\max(0,x)$\\
% $\displaystyle \1_\mathrm{condition}$ & is 1 if the condition is true, 0 otherwise\\
% \end{tabular}
% \egroup
% \vspace{0.25cm}



\bibliography{iclr2023_conference}
\bibliographystyle{iclr2023_conference}

\clearpage
\appendix
\section*{Appendix}

\section{Implementation Details}

\subsection{Details for Filtering RapidAPI}
\label{sec:detail_filtering_api}
We perform a rigorous filtering process to ensure that the ultimate tool set of \ourdata is reliable and functional.
The filtering process is as follows: (1) \textit{initial testing}: we begin by testing the basic functionality of each API to ascertain whether they are operational. We discard any APIs that do not meet this basic criterion; 
% (2) \textit{request method}: we filter out all the APIs that utilize request method other than GET since other methods such as POST are hard to be parsed into a unified request format due to various kinds of request body schemas;
(2) \textit{example response evaluation}: we make API calls to obtain an example response. Then we evaluate their effectiveness by response time and quality. APIs that consistently exhibit a long response time are omitted. Also, we filter out the APIs with low-quality responses, such as HTML source codes or other error messages.

\subsection{API Response Compression}
When examining the response returned by each API, we discover that some responses may contain redundant information and are too long to be fed into LLMs. This may lead to problems due to the limited context length of LLMs.
Therefore, we perform a response compression to reduce the length of API responses while maintaining their critical information.

Since each API has a fixed response format, we use \turbo to analyze one response example and remove unimportant keys within the response to reduce its length. The prompt of \turbo contains the following information for each API: (1) tool documentation, which includes tool name, tool description, API name, API description, parameters, and an example API response. This gives \turbo a hint of the API’s functionality; (2) $3$ in-context learning examples, each containing an original API response and a compressed response schema written by experts.
In this way, we obtain the response compression strategies for all APIs.
During inference, when the API response length exceeds $1024$ tokens, we compress the response by removing unimportant information. If the compressed response is still longer than $1024$, we only retain the first $1024$ tokens. Through human evaluation, we find that this compression retains important information contained in the API response and successfully removes the noises.

\subsection{Details for Training ToolLLaMA}
\label{details_training_toolllama}
We train the model in a multi-round conversation mode. For the training data format, we keep the input and output the same as those of \turbo. Since it is unclear how \turbo organizes the function call field, we just concatenate this information into the input as part of the prompt for ToolLLaMA. For the training hyper parameters, we use a learning rate of $5\times10^{-5}$, a warmup ratio of $4\times10^{-2}$, a total batch size of $64$, a maximum sequence length of $8192$, and use a position interpolation ratio of $2$. We train the model for two epochs and select the model checkpoint
with the best performance on the development set
and then evaluate it on the test set.

\subsection{Details for \dfs}
\label{DFS_implementation}

In practice, it is essential to balance effectiveness with costs (the number of OpenAI API calls). Classical DFS algorithms generate multiple child nodes at each step, then sort all the child nodes, and select the highest-scoring node for expansion. After greedily expanding to the terminal node, DFS backtracks to explore nearby nodes, expanding the search space. Throughout the algorithm, the most resource-intensive part is the sorting process of child nodes. If we use an LLM to evaluate two nodes at a time, it requires approximately $O(n\log n)$ complexity of OpenAI API calls, where $n$ is the number of child nodes.

In fact, we find empirically that in most cases, the nodes ranked highest are often the node generated at first.
Therefore, we skip the sorting process of child nodes and choose a pre-order traversal (a variant for DFS) for the tree search. This design has the following advantages:

\begin{itemize}
    \item If the model does not retract an action (e.g., for the case of simple instructions), then \dfs degrades to ReACT, which makes it as efficient as ReACT.
    \item After the algorithm finishes, the nodes explored by this method are almost the same as those found by a classical DFS search. Hence, it can also handle complex instructions that only DFS can solve.
\end{itemize}

Overall, this design achieves a similar performance as DFS while significantly reducing costs.

It should also be noted that ReACT can be viewed as a degraded version of \dfs. Therefore, although \ourmodel is trained on data created by \dfs, the model can be used either through ReACT or \dfs during inference.

\subsection{Details for ToolEval}
\label{sec:details_tooleval}

We adopt two metrics for automatic tool-use capability evaluation: pass rate and win rate.

\paragraph{Details for Pass Rate}
To assess whether a solution path completes the tasks outlined in the original instruction and successfully passes it, we need to first consider the solvability of the instruction. In principle, an instruction can be classified as either (1) solvable: for example, at least one of the provided tools is potentially helpful in solving the original instruction; or (2) unsolvable: for example, all APIs are irrelevant to the instruction or the instruction provides invalid information such as invalid email address.

To determine whether a solution path is deemed passed or not, we need to consider whether the instruction is solvable or unsolvable. In our evaluation, three types of labels can be given to each solution path, i.e., \texttt{Pass}, \texttt{Fail}, and \texttt{Unsure}. Specifically, we define different rules as follows:

If the instruction is solvable:
\begin{enumerate}
    \item If the model gives finish type ``Finish by Giving Up'',
    \begin{enumerate}
    \item After trying all the APIs extensively during and receiving no helpful information from APIs, the solution path is deemed a \texttt{Pass}.
    \item If the model only calls a few API or receiving valid information from the APIs, the solution path is deemed a \texttt{Fail}.
    \end{enumerate}
    \item If the model gives finish type ``Finish with Final Answer'',
    \begin{enumerate}
    \item If the APIs provide no valid information, and the model has tried all the APIs to retrieve useful information, but the final answer still does not resolve the original instruction or conveys a refusal (such as ``I'm sorry, but I can't provide you with this, because the tools are unavailable''), the solution path is deemed a \texttt{Pass}.
    \item If the tools provide valid information, and the final answer does not completely resolve the instruction or is a refusal, the solution path is deemed a \texttt{Fail}.
    \item If the final answer completely resolves the original instruction, the solution path is deemed a \texttt{Pass}.
    \item If it is unable to determine if the instruction is resolved based on the content of the final answer, the solution path is deemed an \texttt{Unsure}.
    \end{enumerate}
\end{enumerate}

If the instruction is unsolvable:
\begin{enumerate}
    \item If the model gives finish type ``Finish with Final Answer'',
    \begin{enumerate}
        \item If the final answer resolves an instruction that was initially considered unresolvable, the solution path is deemed a \texttt{Pass}.
        \item If the final answer is a refusal, the solution path is deemed a \texttt{Pass}.
        \item If the final answer is hallucinated by the model itself and provides a false positive response (such as ``I've completed the task, the final answer is *''), the solution path is deemed a \texttt{Fail}.
    \end{enumerate}
    \item If the model gives finish type ``Finish by Giving Up",
    \begin{enumerate}
        \item Under this case, the solution path is deemed a \texttt{Pass}.
    \end{enumerate}
\end{enumerate}

For every solution path, we instruct the \turbo evaluator to generate multiple ($\ge 4$) predictions and perform a majority vote to derive the final pass rate.

\paragraph{Details for Win Rate}

Since pass rate only measures whether an instruction is completed or not, instead of how well it is completed, we adopt another metric: win rate. It is measured by comparing two solution paths for a given instruction. We assume that a passed candidate is better than a failed candidate and only compare those solution paths that are both ``\texttt{Pass}'', or both ``\texttt{Failed}'' annotated by the \turbo evaluator. Note that compared with another solution path, one solution path will be annotated with one of the following: \texttt{win}, \texttt{lose}, or \texttt{tie}. We build rules for the evaluator's behavior to decide which solution path is better, and the criteria are listed as follows:
\begin{enumerate}
    \item Information richness: whether the final answer contains all the necessary information to answer the original instruction. A significantly richer answer is better, while a similar level of richness that is sufficient to answer the question ties.
    \item Factuality: whether it accurately describes what has been done, and what failed in the end. A more accurate description in the final answer is better.
    \item Reasoning: whether a detailed and accurate reason for failure is provided if the query remains unresolved. A more detailed reason is better.
    \item Milestone: calculating the number of milestones reached during execution.
    \item  Exploration: whether more potentially useful APIs were attempted during the execution process. The use of a greater number of APIs is better.
    \item Cost: Having fewer repeated (redundant) API calls is better if the number of APIs used is the same.
\end{enumerate}

For every solution path, we also generate multiple ($\ge 4$) predictions and then perform a majority vote to derive the final win rate. In Table~\ref{tab:main_exp}, for ease of reading, we split the ratio of \texttt{tie} into two pieces and add them to \texttt{win} and \texttt{lose}, respectively. In Table~\ref{tab:main_exp_all}, we report the original numbers as a reference.

\paragraph{Comparing Human Evaluation and ToolEval}
To validate the reliability of \turbo evaluator in both pass rate and win rate, we sample among four different methods (ChatGPT+ReACT, ChatGPT+\dfs, ToolLLaMA+\dfs and GPT4+\dfs) to obtain solution pairs for $300$ test instructions for \textbf{each} method. Then we engage humans to annotate the pass rate for ChatGPT+\dfs, ToolLLaMA+\dfs and GPT4+\dfs, and the win rate among ChatGPT+ReACT and ChatGPT+\dfs.
Our \turbo evaluator demonstrates a high agreement of $\textbf{87.1\%}$ in pass rate and $\textbf{80.3\%}$ in win rate with human annotators. This result shows that our evaluator generates highly similar evaluation results to humans and can be viewed as a credible evaluator who simulates human evaluation on pass rate and win rate.

It should also be noted that the evaluation for tool learning is far more intricate than traditional tasks such as dialogue. The reason is that there may exist infinite ``correct'' solution paths for each instruction.
In our initial investigations, we surprisingly found that even human experts often disagree with each other in deciding which solution path is better, leading to a relatively low agreement. For instance, one may prefer a solution path that uses only a few APIs to derive the final answer quickly; while another may prefer a solution path that extensively tries all the APIs to cross-validate specific information. In this regard, we believe there is still a long way to go for a fair evaluation of the tool-use domain, and we believe this work has paved the way for it. We expect more future works to explore this interesting research problem.
% We also find that our automatic evaluator achieves lower variance ($\textbf{3.47\%}$) than humans ($\textbf{3.97\%}$) when annotating multiple times for the same instruction. This indicates that our evaluator is more consistent than humans.

\begin{table}[!t]
    \centering
    \resizebox{\textwidth}{!}{%
    \begin{tabular}{ccrrrrrrrrrrrrrr}
    \toprule
    \multicolumn{1}{c}{\multirow{2}{*}{Model}} & \multicolumn{1}{c}{\multirow{2}{*}{Method}} &\multicolumn{2}{c}{\underline{\textbf{I1-Inst.}}} & \multicolumn{2}{c}{\underline{\textbf{I1-Tool}}} & \multicolumn{2}{c}{\underline{\textbf{I1-Cat.}}} & \multicolumn{2}{c}{\underline{\textbf{I2-Inst.}}} & \multicolumn{2}{c}{\underline{\textbf{I2-Cat.}}} & \multicolumn{2}{c}{\underline{\textbf{I3-Inst.}}} & \multicolumn{2}{c}{\underline{\textbf{Average}}} \\
    \multicolumn{1}{c}{} & \multicolumn{1}{c}{} & Win & Tie & Win & Tie & Win & Tie & Win & Tie & Win & Tie & Win & Tie & Win & Tie \\
    \toprule
    % \ourmodel & $75.0$ & $\textbf{50.0}$ & $68.0$ & $45.0$ & $80.0$ & $45.0$ & $56.0$ & $\textbf{59.0}$ & $47.0$ & $48.0$ & $40.0$ & $46.0$ & $61.0$ & $48.8$ \\
    % \midrule
    ChatGPT & DFSDT & $52.5$ & $16.0$ & $\underline{55.0}$ & $14.0$ & $47.5$ & $19.5$ & $\textbf{67.0}$ & $10.0$ & $\textbf{58.5}$ & $12.5$ & $61.0$ & $16.0$ & $56.9$ & $14.7$  \\
    \midrule
    Claude-2 & ReACT & $27.0$ & $8.0$ & $24.0$ & $7.5$ & $29.5$ & $8.5$ & $32.0$ & $6.0$ & $28.5$ & $6.0$ & $43.0$ & $9.5$ & $30.7$ & $7.5$  \\
    & DFSDT & $34.0$ & $8.0$ & $41.0$ & $6.5$ & $39.5$ & $7.5$ & $32.5$ & $9.5$ & $33.5$ & $0.0$ & $65.0$ & $0.0$ & $40.8$ & $5.3$  \\
    \midrule
    Text-Davinci-003 & ReACT & $23.5$ & $10.0$ & $28.5$ & $13.5$ & $27.0$ & $8.0$ & $26.5$ & $6.5$ & $25.5$ & $8.5$ & $41.0$ & $8.0$ & $28.7$ & $9.1$  \\
    & DFSDT & $35.0$ & $10.5$ & $37.5$ & $12.5$ & $40.0$ & $13.5$ & $36.5$ & $8.0$ & $40.0$ & $6.5$ & $60.0$ & $6.0$ & $41.5$ & $9.5$  \\
    \midrule
    GPT4 & ReACT & $52.5$ & $15.0$ & $53.5$ & $10.5$ & $\underline{56.0}$ & $15.0$ & $59.5$ & $12.5$ & $52.5$ & $15.5$ & $\underline{76.0}$ & $4.0$ & $58.3$ & $12.1$  \\
     & DFSDT & $\textbf{60.5}$ & $14.0$ & $\textbf{62.5}$ & $10.5$ & $\textbf{58.0}$ & $17.0$ & $\textbf{67.0}$ & $12.5$ & $\underline{57.0}$ & $12.5$ & $\textbf{80.0}$ & $8.0$ & $\textbf{64.2}$ & $12.4$  \\
    \midrule
    Vicuna & (ReACT \& DFSDT) & $0.0$ & 0.0 & $0.0$ & 0.0 & $0.0$ & 0.0 & $0.0$ & 0.0 & $0.0$ & 0.0 & $0.0$ & 0.0 & $0.0$ & 0.0  \\
    Alpaca & (ReACT \& DFSDT) & $0.0$ & 0.0 & $0.0$ & 0.0 & $0.0$ & 0.0 & $0.0$ & 0.0 & $0.0$ & 0.0 & $0.0$ & 0.0 & $0.0$ & 0.0  \\
    \midrule
     & ReACT & $40.0$ & $10.0$ & $36.5$ & $11.0$ & $42.0$ & $11.0$ & $45.5$ & $10.5$ & $37.5$ & $8.5$ & $51.0$ & $8.0$ & $42.1$ & $9.8$\\
     ToolLLaMA & DFSDT & $48.5$ & $13.0$ & $50.5$ & $9.5$ & $49.5$ & $10.0$ & $62.5$ & $12.0$ & $52.0$ & $12.0$ & $68.0$ & $2.0$ & $55.2$ & $9.8$ \\
      & Retriever & $\underline{58.0}$ & $8.5$ & $54.5$ & $9.0$ & $51.0$ & $8.0$ & $64.5$ & $8.0$ & $56.0$ & $9.5$ & $71.0$ & $4.0$ & $\underline{59.2}$ & $7.8$ \\
    
    \bottomrule
    \end{tabular}%
    }
    \caption{
    \small{Win rate results before merging the tie label. Win rate is calculated by comparing each model with ChatGPT-ReACT. A win rate higher than $50\%$ means the model performs better than ChatGPT-ReACT. Apart from ToolLLaMA-DFSDT-Retriever, all methods use the oracle API retriever (i.e., ground truth API).
    }
    }
    \label{tab:main_exp_all}
    \end{table}

% \subsection{Details of Performance Metrics for Automatic Evaluator}
% \label{sec:details_of_metrics_for_evaluator}

% Here we describe details of computing metrics used to assert automatic evaluators.

% \paragraph{Human Agreement}\label{exp:human_agreement} 
% To estimate the agreement between the automatic evaluator and humans, we first sample 4 win rate results for each query in human annotations data set from the automatic evaluator.
% Then we check each sample whether agrees with the major human preference on 600 queries, giving scores of one if agree and zero if disagree.
% If there are multiple major preference, we give scores of $\frac{1}{n}$ (where n is the count of major preference) if one major preference of the evaluator agree with humans and zero if not.
% We finally average the averaged scores of the automatic evaluator samples for all query to get the human agreement.

% \paragraph{Bias}
% To estimate the bias, we sample 4 preference results as in calculating Human Agreement.
% Then we check the disagreement between the major evaluator preference and major human preference, giving scores of one if disagree and zero if agree.
% For multiple major preference, we have a initial score of one and subtract scores of $\frac{1}{n}$ (where n is the count of major preference) if one major preference of the evaluator agree with humans and zero if not.
% We finally average the scores for all query to get the bias.

% \paragraph{Variance}
% We estimate variance like the way we estimate Human Agreement, but subtract the score from one for each sample and calculate variance for each query,
% We then average the variance across the annotated data set to get variance.

\subsection{Details for Experiments on APIBench}

When generalizing \ourmodel to APIBench, no training updates were made to \ourmodel, but instead of treating each API in the prompt as a function call. We define one function that represents selecting an API, providing the code for invoking it, and describing the generated output in natural language. We do not consider the zero-shot setting of APIBench where the prompts do not contain any API descriptions because the APIs from the three tested domains were never encountered during training.

\subsection{Prompts for Instruction Generation}
\label{sec:inst_prompt}
Below we list the detailed prompt for instruction generation, which consists of four parts: task description, in-context learning examples, sampled API list, and other requirements.

\noindent\makebox[\linewidth]{\rule{\linewidth}{0.4pt}}
\textit{Task Description of Single-tool Instructions}:\\
You will be provided with a tool, its description, all of the tool's available API functions, the descriptions of these API functions, and the parameters required for each API function. Your task involves creating 10 varied, innovative, and detailed user queries that employ multiple API functions of a tool. For instance, if the tool `climate news' has three API calls - `get\_all\_climate\_change\_news', `look\_up\_climate\_today', and `historical\_climate', your query should articulate something akin to: first, determine today's weather, then verify how often it rains in Ohio in September, and finally, find news about climate change to help me understand whether the climate will change anytime soon. This query exemplifies how to utilize all API calls of `climate news'. A query that only uses one API call will not be accepted. Additionally, you must incorporate the input parameters required for each API call. To achieve this, generate random information for required parameters such as IP address, location, coordinates, etc. For instance, don't merely say `an address', provide the exact road and district names. Don't just mention `a product', specify wearables, milk, a blue blanket, a pan, etc. Don't refer to `my company', invent a company name instead. The first seven of the ten queries should be very specific. Each single query should combine all API call usages in different ways and include the necessary parameters. Note that you shouldn't ask `which API to use', rather, simply state your needs that can be addressed by these APIs. You should also avoid asking for the input parameters required by the API call, but instead directly provide the parameter in your query. The final three queries should be complex and lengthy, describing a complicated scenario where all the API calls can be utilized to provide assistance within a single query. You should first think about possible related API combinations, then give your query. Related\_apis are apis that can be used for a give query; those related apis have to strictly come from the provided api names. For each query, there should be multiple related\_apis; for different queries, overlap of related apis should be as little as possible. Deliver your response in this format: [{Query1: ......, `related\_apis':[api1, api2, api3...]},{Query2: ......, `related\_apis':[api4, api5, api6...]},{Query3: ......, `related\_apis':[api1, api7, api9...]}, ...] \\

\textit{Task Description of Multi-tool Instructions}:\\
You will be provided with several tools, tool descriptions, all of each tool's available API functions, the descriptions of these API functions, and the parameters required for each API function. Your task involves creating 10 varied, innovative, and detailed user queries that employ API functions of multiple tools. For instance, given three tools `nba\_news', `cat-facts', and `hotels': `nba\_news' has API functions `Get individual NBA source news' and `Get all NBA news', `cat-facts' has API functions `Get all facts about cats' and `Get a random fact about cats', `hotels' has API functions `properties/get-details (Deprecated)', `properties/list (Deprecated)' and `locations/v3/search'. Your query should articulate something akin to: `I want to name my newborn cat after Kobe and host a party to celebrate its birth. Get me some cat facts and NBA news to gather inspirations for the cat name. Also, find a proper hotel around my house in Houston Downtown for the party.' This query exemplifies how to utilize API calls of all the given tools. A query that uses API calls of only one tool will not be accepted. Additionally, you must incorporate the input parameters required for each API call. To achieve this, generate random information for required parameters such as IP address, location, coordinates, etc. For instance, don't merely say `an address', provide the exact road and district names. Don't just mention `a product', specify wearables, milk, a blue blanket, a pan, etc. Don't refer to `my company', invent a company name instead. The first seven of the ten queries should be very specific. Each single query should combine API calls of different tools in various ways and include the necessary parameters. Note that you shouldn't ask `which API to use', rather, simply state your needs that can be addressed by these APIs. You should also avoid asking for the input parameters required by the API call, but instead directly provide the parameters in your query. The final three queries should be complex and lengthy, describing a complicated scenario where all the provided API calls can be utilized to provide assistance within a single query. You should first think about possible related API combinations, then give your query. Related APIs are APIs that can be used for a given query; those related APIs have to strictly come from the provided API names. For each query, there should be multiple related APIs; for different queries, overlap of related APIs should be as little as possible. Deliver your response in this format: [{Query1: ......, `related\_apis':[[tool name, api name], [tool name, api name], [tool name, api name]...]},{Query2: ......, `related\_apis':[[tool name, api name], [tool name, api name], [tool name, api name]...]},{Query3: ......, `related\_apis':[[tool name, api name], [tool name, api name], [tool name, api name]...]}, ...] \\
\noindent\makebox[\linewidth]{\rule{\linewidth}{0.4pt}}
\textit{In-context Seed Examples}. In the following, we show one single-tool instruction seed example and one multi-tool instruction seed example.

For example, with tool ASCII Art, the given api\_names are `figlet', `list figlet styles', `cowsay', `list\_cowsay\_styles', `matheq'. \\
Some sample queries and related\_apis would be: \\
``Query": ``Need to create an ASCII art representation of a mathematical equation. The equation is `y = mx + c', where m and c are constants. Help me generate the ASCII art for this equation. Also please generate an ASCII art representation of the text `Newton's Second Law of Motion'.", ``related\_apis": ['figlet', `list figlet styles', `matheq'] \\
``Query": ``Working on a research paper on cows and need to include ASCII art representations of various cows. Can you first retrieve available ASCII art styles for cows? Then, can you generate ASCII art for cows like the Jersey, Holstein, and Guernsey? Finally, I want the cow to say `Moo!' in the ASCII art.", ``related\_apis": ['figlet', `list figlet styles', `cowsay', `list\_cowsay\_styles'] \\
``Query": ``I'm writing a blog post on ASCII art and need to include some examples. Can you generate ASCII art for the following strings: `ASCII', `art', and `gallery'? You can first retrieve available figlet styles and then generate ASCII art for the strings using the styles.", ``related\_apis": ['figlet', `list figlet styles'] \\
``Query": ``Greetings! I'm putting together a quirky slideshow about our furry friends and need your help to sprinkle some ASCII art goodness. Could you kindly fetch me the catalog of ASCII art styles available for animals? Also, I'm particularly keen on featuring ASCII art for creatures like pandas, cows, elephants, and penguins. And if they could say something cute like `Hello!' or `Hugs!' in the ASCII art, that would be purr-fect!", ``related\_apis": ['figlet', `list figlet styles', `cowsay', `list\_cowsay\_styles'] \\

For example, with tool ['Entrepreneur Mindset Collection', `Random Words', `thedigitalnewsfeederapi', `Chemical Elements'], the given api\_names are (tool `Entrepreneur Mindset Collection')'Random Quote in JSON format', (tool `Random Words')'Get multiple random words', (tool `Random Words')'Get a random word', (tool `thedigitalnewsfeederapi')'getting specific cricket articles', (tool `thedigitalnewsfeederapi')'Getting Cricket Articles', (tool `thedigitalnewsfeederapi')'getting specific news articles', (tool `thedigitalnewsfeederapi')'Getting News Articles', (tool `thedigitalnewsfeederapi')'getting all news articles', (tool `Chemical Elements')'Get All Chemical Elements'. \\
Some sample queries and related\_apis would be: \\
``Query": ``For my best friend's surprise birthday party, I require inspiration for party games and decorations. Kindly suggest some random words that can serve as themes for the party. Furthermore, I'm interested in gathering news articles about the latest party trends to ensure a modern celebration. Also, I would appreciate details about the local hotels in my area for accommodation options. Your assistance is greatly appreciated.", ``related\_apis": [['Random Words', `Get multiple random words'], ['thedigitalnewsfeederapi', `Getting News Articles'], ['thedigitalnewsfeederapi', `Getting all news articles']] \\
``Query": ``In the midst of organizing a team-building event for my esteemed company, I eagerly seek your valued input for invigorating activities. Might I kindly request a collection of random quotes that encapsulate the essence of teamwork and motivation? Additionally, I am keen on exploring news articles that showcase triumphant team-building events, as they serve as a wellspring of inspiration.", ``related\_apis": [['Entrepreneur Mindset Collection', `Random Quote in JSON format'], ['thedigitalnewsfeederapi', `Getting News Articles']]
``Query": ``I need specific cricket articles that discuss the health benefits of sports for my research paper on exercise. I also want to know which chemical elements are associated with exercising, like increased iron (Fe) and its impact on bone marrow.", ``related\_apis": [['thedigitalnewsfeederapi', `getting specific cricket articles'], ['Chemical Elements', `Get All Chemical Elements']] \\
``Query": ``I'm starting a new business venture and I need to make a speech announcing the new dawn. Provide me some quotes and words for me to start with. I would like to gather news articles about successful entrepreneurs for inspiration.", ``related\_apis": [['Entrepreneur Mindset Collection', `Random Quote in JSON format'], ['Random Words', `Get multiple random words'], ['thedigitalnewsfeederapi', `getting specific news articles']] \\
These are only examples to show you how to write the query. Do not use APIs listed in the above examples, but rather, use the ones listed below in the INPUT. \\
\noindent\makebox[\linewidth]{\rule{\linewidth}{0.4pt}}
\textit{Sampled API List} (An example)
\begin{lstlisting}[basicstyle=\ttfamily, breaklines=true]
{
    "tool_description": "EntreAPI Faker is used to dynamically create mock, demo, test and sample data for your application",
    "name": "EntreAPI Faker",
    "api_list": [
        {
            "name": "Longitute",
            "url": "https://entreapi-faker.p.rapidapi.com/address/longitude",
            "description": "Generate a random longitude.",
            "method": "GET",
            "required_parameters": [],
            "optional_parameters": [
                {
                    "name": "max",
                    "type": "NUMBER",
                    "description": "Maximum value for latitude.",
                    "default": ""
                },
                {
                    "name": "min",
                    "type": "NUMBER",
                    "description": "Minimum value for latitude.",
                    "default": ""
                },
                {
                    "name": "precision",
                    "type": "NUMBER",
                    "description": "Precision for latitude.",
                    "default": ""
                }
            ],
            "tool_name": "EntreAPI Faker",
            "category_name": "Data"
        },
        {
            "name": "Boolean",
            "url": "https://entreapi-faker.p.rapidapi.com/datatype/boolean",
            "description": "Randomly generate a boolean value.",
            "method": "GET",
            "required_parameters": [],
            "optional_parameters": [],
            "tool_name": "EntreAPI Faker",
            "category_name": "Data"
        },
        {
            "name": "Past",
            "url": "https://entreapi-faker.p.rapidapi.com/date/past",
            "description": "Randomly generate a date value in the past.",
            "method": "GET",
            "required_parameters": [],
            "optional_parameters": [
                {
                    "name": "refDate",
                    "type": "STRING",
                    "description": "Starting reference date",
                    "default": ""
                },
                {
                    "name": "years",
                    "type": "NUMBER",
                    "description": "Number of years for the range of dates.",
                    "default": ""
                }
            ],
            "tool_name": "EntreAPI Faker",
            "category_name": "Data"
        },
        {
            "name": "Image Url",
            "url": "https://entreapi-faker.p.rapidapi.com/image/imageUrl",
            "description": "Randomly generate an image URL.",
            "method": "GET",
            "required_parameters": [],
            "optional_parameters": [
                {
                    "name": "width",
                    "type": "NUMBER",
                    "description": "Width of the image. Default is 640.",
                    "default": ""
                },
                {
                    "name": "height",
                    "type": "NUMBER",
                    "description": "Height of the image. Default is 480.",
                    "default": ""
                },
                {
                    "name": "useRandomize",
                    "type": "BOOLEAN",
                    "description": "Add a random number parameter to the returned URL.",
                    "default": ""
                },
                {
                    "name": "category",
                    "type": "STRING",
                    "description": "The category for the image. Can be one: abstract, animal, avatar, business, cats, city, fashion, food, nature, nightlife, people, sports, technics, transport",
                    "default": ""
                }
            ],
            "tool_name": "EntreAPI Faker",
            "category_name": "Data"
        },
        {
            "name": "Sentence",
            "url": "https://entreapi-faker.p.rapidapi.com/lorem/sentence",
            "description": "Randomly generate a sentence of Lorem Ipsum.",
            "method": "GET",
            "required_parameters": [],
            "optional_parameters": [
                {
                    "name": "wordCount",
                    "type": "NUMBER",
                    "description": "Number of words in the sentence.",
                    "default": ""
                }
            ],
            "tool_name": "EntreAPI Faker",
            "category_name": "Data"
        },
        {
            "name": "Gender",
            "url": "https://entreapi-faker.p.rapidapi.com/name/gender",
            "description": "Randomly select a gender.",
            "method": "GET",
            "required_parameters": [],
            "optional_parameters": [
                {
                    "name": "useBinary",
                    "type": "BOOLEAN",
                    "description": "Use binary genders only.",
                    "default": ""
                }
            ],
            "tool_name": "EntreAPI Faker",
            "category_name": "Data"
        },
        {
            "name": "Prefix",
            "url": "https://entreapi-faker.p.rapidapi.com/name/prefix",
            "description": "Randomly generate a prefix (e.g., Mr., Mrs., etc.)",
            "method": "GET",
            "required_parameters": [],
            "optional_parameters": [
                {
                    "name": "gender",
                    "type": "STRING",
                    "description": "Optional gender.",
                    "default": ""
                }
            ],
            "tool_name": "EntreAPI Faker",
            "category_name": "Data"
        },
        {
            "name": "Array Element",
            "url": "https://entreapi-faker.p.rapidapi.com/random/arrayElement",
            "description": "Randomly select an array element.",
            "method": "GET",
            "required_parameters": [],
            "optional_parameters": [
                {
                    "name": "array",
                    "type": "ARRAY",
                    "description": "The list of elements to choose from. Default is [\"a\", \"b\", \"c\"].",
                    "default": ""
                }
            ],
            "tool_name": "EntreAPI Faker",
            "category_name": "Data"
        },
        {
            "name": "Number Value",
            "url": "https://entreapi-faker.p.rapidapi.com/random/number",
            "description": "Randomly generate a number value.",
            "method": "GET",
            "required_parameters": [],
            "optional_parameters": [
                {
                    "name": "min",
                    "type": "NUMBER",
                    "description": "Minimum value.",
                    "default": ""
                },
                {
                    "name": "max",
                    "type": "NUMBER",
                    "description": "Maximum value.",
                    "default": ""
                },
                {
                    "name": "precision",
                    "type": "NUMBER",
                    "description": "Precision of the number.",
                    "default": ""
                }
            ],
            "tool_name": "EntreAPI Faker",
            "category_name": "Data"
        },
        {
            "name": "URL",
            "url": "https://entreapi-faker.p.rapidapi.com/internet/url",
            "description": "Randomly generate a URL.",
            "method": "GET",
            "required_parameters": [],
            "optional_parameters": [],
            "tool_name": "EntreAPI Faker",
            "category_name": "Data"
        }
    ]
}
\end{lstlisting}
\noindent\makebox[\linewidth]{\rule{\linewidth}{0.4pt}}
\textit{Other Requirements:}\\
Please produce ten queries in line with the given requirements and inputs. These ten queries should display a diverse range of sentence structures: some queries should be in the form of imperative sentences, others declarative, and yet others interrogative. Equally, they should encompass a variety of tones, with some being polite, others straightforward. Ensure they vary in length and contain a wide range of subjects: myself, my friends, family, and company. Aim to include a number of engaging queries as long as they relate to API calls. Keep in mind that for each query, invoking just one API won't suffice; each query should call upon two to five APIs. However, try to avoid explicitly specifying which API to employ in the query. Each query should consist of a minimum of thirty words.
\noindent\makebox[\linewidth]{\rule{\linewidth}{0.4pt}}


\subsection{Prompts for Solution Path Annotation}
\label{sec:answer_prompt}
We use the following prompt when searching for the solution path. When expanding the child nodes, we use diversity\_user\_prompt, showing the information of previous child nodes.

\begin{lstlisting}[basicstyle=\ttfamily, breaklines=true]
------------------------------------------------------------------
system_prompt: 
You are Tool-GPT, capable of utilizing numerous tools and functions to complete the given task. 
1.First, I will provide you with the task description, and your task will commence. 
2.At each step, you need to analyze the current status and determine the next course of action by executing a function call. 
3.Following the call, you will receive the result, transitioning you to a new state. Subsequently, you will analyze your current status, make decisions about the next steps, and repeat this process. 
4.After several iterations of thought and function calls, you will ultimately complete the task and provide your final answer. 
Remember: 
1.The state changes are irreversible, and you cannot return to a previous state.
2.Keep your thoughts concise, limiting them to a maximum of five sentences.
3.You can make multiple attempts. If you plan to try different conditions continuously, perform one condition per try.
4.If you believe you have gathered enough information, call the function "Finish: give_answer" to provide your answer for the task.
5.If you feel unable to handle the task from this step, call the function "Finish: give_up_and_restart".
Let's Begin!
Task description: {task_description}
---------------------------------------------------------
diversity_user_prompt: 
This is not the first time you try this task, all previous trails failed.
Before you generate your thought for this state, I will first show you your previous actions for this state, and then you must generate actions that is different from all of them. Here are some previous actions candidates:
{previous_candidate}
Remember you are now in the intermediate state of a trail, you will first analyze the now state and previous action candidates, then make actions that is different from all the previous.
---------------------------------------------------------
Finish_function_description: 
{
    "name": "Finish",
    "description": "If you believe that you have obtained a result that can answer the task, please call this function to provide the final answer. Alternatively, if you recognize that you are unable to proceed with the task in the current state, call this function to restart. Remember: you must ALWAYS call this function at the end of your attempt, and the only part that will be shown to the user is the final answer, so it should contain sufficient information.",
    "parameters": {
        "type": "object",
        "properties": {
            "return_type": {
                "type": "string",
                "enum": ["give_answer","give_up_and_restart"],
            },
            "final_answer": {
                "type": "string",
                "description": "The final answer you want to give the user. You should have this field if \"return_type\"==\"give_answer\"",
            }
        },
        "required": ["return_type"],
    }
}
\end{lstlisting}

\end{document}
