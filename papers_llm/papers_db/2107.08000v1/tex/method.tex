\section{Global-local attention}
\label{sec:method}

We design a \emph{global-local attention module} (GLAM), which is attached at the end of a backbone network. \autoref{fig:glam} illustrates its main components. We are given a $c\times h \times w$ feature tensor $\vF$, where $c$ is the number of channels, and $h \times w$ is the spatial resolution. Local attention collects context from the image and applies pooling to obtain a $c \times 1 \times 1$ \emph{local channel attention map} $\vA_c^l$ and a $1 \times h \times w$ \emph{local spatial attention map} $\vA_s^l$. Global attention allows interaction between channels, resulting in a $c \times c$ \emph{global channel attention map} $\vA_c^g$, and between spatial locations, resulting in a $hw \times hw$ \emph{global spatial attention map} $\vA_s^g$. The feature maps produced by the two attention streams are combined with the original one by a learned fusion mechanism into the \emph{global-local attention feature map} $\vF^{gl}$ before being spatially pooled into a global image descriptor.

%------------------------------------------------------------------------------
\begin{figure}
\centering
%------------------------------------------------------------------------------
\begin{tikzpicture}[
	font={\footnotesize},
]
\matrix[
	row sep=15pt,column sep=10pt,cells={scale=1,},
]{
	\& \node[inp](i){feature map}; \\
	\& \node[down](d1){conv $1 \times 1$}; \\
	\node[fix](c3){conv $3 \times 3$}; \&
	\node[fix](c5){conv $5 \times 5$}; \&
	\node[fix](c7){conv $7 \times 7$}; \\
	\& \node[cat](ca){concat}; \\
	\& \node[down](d2){conv $1 \times 1$}; \\
	\& \node[outp](o){attention map}; \\
};
\coordinate(s) at($(d1.south)!.5!(c5.north)$); % split
\coordinate(m) at($(c5.south)!.5!(ca.north)$); % merge
\node(left)[left=5pt of c3.west]{};
\coordinate(l) at(left.center);
\draw[->]
	(i) edge node[dim,midway,right]{$c \times h \times w$} (d1)
	(d1) edge (c5)
	(c5) edge (ca)
	(ca) edge node[dim,midway,right]{$4c' \times h \times w$} (d2)
	(d2) edge node[dim,midway,right]{$1 \times h \times w$} (o)
	;
\draw[->] (s) -| (c3);
\draw[->] (s) -| node[dim,midway,above]{$c' \times h \times w$} (c7);
\draw (s) -| (l) (l) |- (m);
\draw (c3) |- (m) (c7) |- (m);
\node[dim,below right=2pt of c7.south](dil) {\emph{dilated} \\ \emph{conv}};
\node[left=1pt of i] {$\vF$};
\node[left=1pt of d1] {$\vF'$};
\node[left=1pt of o] {$\vA_s^l$};
\end{tikzpicture}
%------------------------------------------------------------------------------

\caption{Local spatial attention. Convolutional layers in blue implemented by dilated convolutions with kernel size $3 \times 3$ and dilation factors $1,3,5$.}
\label{fig:fig3}
\end{figure}
%------------------------------------------------------------------------------

\subsection{Local attention}
\label{sec:local}

We extract an 1D channel and a 2D spatial attention map to weigh the feature map in the corresponding dimensions.

\paragraph{Local channel attention}

Following ECA-Net~\cite{wang01}, this attention captures local channel information. As shown in \autoref{fig:fig4}, we are given a $c\times h\times w$ feature tensor $\vF$ from our backbone. We first reduce it to a $c \times 1 \times 1$ tensor by \emph{global average pooling} (GAP). Channel attention is then captured by a 1D convolution of kernel size $k$ along the channel dimension, where $k$ controls the extent of cross-channel interaction. This is followed by a sigmoid function, resulting in the $c\times 1\times 1$ \emph{local channel attention map} $\vA_c^l$.

%------------------------------------------------------------------------------

\paragraph{Local spatial attention}

Inspired by the inception module~\cite{Szegedy01} and similar to~\cite{Kim01}, this attention map captures local spatial information at different scales. As shown in \autoref{fig:fig3},
given the same $c\times h\times w$ feature tensor $\vF$ from our backbone, we obtain a new tensor $\vF'$ with channels reduced to $c'$, using a ${1 \times 1}$ convolution. We then extract local spatial contextual information using convolutional filters of kernel size ${3\times 3}$, ${5\times 5}$, and ${7\times 7}$, which are efficiently implemented by ${3\times 3}$ dilated convolutions~\cite{chen2017rethinking,Yu_2017_CVPR} with dilation parameter 1, 2, and 3 respectively. The resulting features, along with one obtained by ${1\times 1}$ convolution on $\vF'$, are concatenated into a $4c' \times h \times w$ tensor. Finally, we obtain the $1 \times h \times w$ \emph{local spatial attention map} $\vA_s^l$ by a ${1\times 1}$ convolution that reduces the channel dimension to $1$.

The middle column of \autoref{fig:fig7} shows heat maps of local spatial attention, localizing target objects in images.

%------------------------------------------------------------------------------

\paragraph{Local attention feature map}

We use the local channel attention map $\vA_c^l$ to weigh $\vF$ in the channel dimension
\begin{equation}
	\vF_c^l \defn \vF \odot \vA_c^l + \vF.
\label{eq:eq3}
\end{equation}
We then use local spatial attention map $\vA_s^l$ to weigh $\vF_c^l$ in the spatial dimensions, resulting in the $c \times h \times w$ \emph{local attention feature map}
\begin{equation}
	\vF^l = \vF_c^l \odot \vA_s^l + \vF_c^l.
\label{eq:eq3-1}
\end{equation}
Here, $\vA \odot \vB$ denotes an element-wise multiplication of tensors $\vA$ and $\vB$, with broadcasting when one tensor is smaller. We adopt the choice of applying channel followed by spatial attention from \emph{convolutional block attention module} CBAM~\cite{woo01}. However, apart from computing $\vA_s^l$ at different scales, both attention maps are obtained from the original tensor $\vF$ rather than sequentially. In addition, both~\eq{eq3} and~\eq{eq3-1} include residual connections, while CBAM includes a single residual connection over both steps.

%------------------------------------------------------------------------------
\begin{figure}
\centering
\begin{tikzpicture}[
	font={\footnotesize},
]
\matrix[
	row sep=15pt,column sep=10pt,cells={scale=1,},
]{
	\node[inp](i){feature map}; \&
	\node[gap](g){GAP}; \\
	\& \node[fix](c2){conv1d($k$)};
	\& \node[fix](c3){conv1d($k$)}; \\
	\& \node[fun](s1){$\sigmoid$};
	\& \node[fun](s2){$\sigmoid$}; \\
	\& \node[op](m1){$\times$}; \\
	\node[op](m2){$\times$}; \&
	\node[fun](sm){$\softmax$}; \\
	\node[outp](o){attention feature map}; \\
};
\coordinate(s) at($(g.south)!.5!(c2.north)$); % split
\draw[->]
	(i) edge (g)
	(g)  edge (c2)
	(c2) edge node[dim,midway,right]{$1 \times c$} (s1)
	(c3) edge node[dim,midway,right]{$1 \times c$} (s2)
	(s1) edge node[dim,midway,right]{$1 \times c$} node[key,midway,left]{$\vQ_c$} (m1)
	(m1) edge node[dim,midway,right]{$c \times c$} (sm)
	(i) edge node[dim,midway,right]{$hw \times c$} node[key,midway,left]{$\vV_c$} (m2)
	(sm) edge node[midway,above]{$\vA_c^g$} (m2)
	(m2) edge node[dim,midway,right]{$c \times h \times w$} (o)
	;
\draw[->] (s) -| node[dim,midway,above]{$1 \times c$} (c3);
\draw[->] (s2) |- node[dim,near start,right]{$1 \times c$} node[key,near start,left]{$\vK_c$} (m1);
\node[left=1pt of i]{$\vF$};
\node[left=1pt of o]{$\vG_c$};
\end{tikzpicture}

\caption{Global channel attention.}
\label{fig:fig6}
\end{figure}
%------------------------------------------------------------------------------

%------------------------------------------------------------------------------

\subsection{Global attention}
\label{sec:global}

We extract two matrices capturing global pairwise channel and spatial interaction to weigh the feature map.

%------------------------------------------------------------------------------

\paragraph{Global channel attention}

We introduce a \emph{global channel attention} mechanism that captures global channel interaction. This mechanism is based on the non-local neural network~\cite{Wang02}, but with the idea of 1D convolution from ECA-Net~\cite{wang01}. As shown in \autoref{fig:fig6}, we are given the $c\times h\times w$ feature tensor $\vF$ from our backbone. We apply GAP and squeeze spatial dimensions, followed by a 1D convolution of kernel size $k$ and a sigmoid function, to obtain $1 \times c$ \emph{query} $\vQ_c$ and \emph{key} $\vK_c$ tensors. The \emph{value} tensor $\vV_c$ is obtained by mere reshaping of $\vF$ to $hw \times c$, without GAP. Next, we form the outer product of $\vK_c$ and $\vQ_c$, followed by softmax over channels to obtain a $c \times c$ \emph{global channel attention map}
\begin{equation}
	\vA_c^g = \softmax({\vK_c}\tran \vQ_c).
\label{eq:eq6-1}
\end{equation}
Finally, this attention map is multiplied with $\vV_c$ and the matrix product $\vV_c \vA_c^g$ is reshaped back to $c \times h \times w$ to give the \emph{global channel attention feature map} $\vG_c$. In GSoP~\cite{Gao_2019_CVPR} and A$^2$-Net~\cite{ChenKLYF18}, a $c \times c$ global channel attention map is obtained by multiplication of $hw \times c$ matrices; \eq{eq6-1} is more efficient, using only an outer product of $1 \times c$ vectors.

%------------------------------------------------------------------------------
\begin{figure}
\centering
\begin{tikzpicture}[
	font={\footnotesize},
]
\matrix[
	row sep=15pt,column sep=10pt,cells={scale=1,},
]{
	\& \node[inp](i){feature map}; \\
	\node[down](c1){conv $1 \times 1$}; \&
	\node[down](c2){conv $1 \times 1$}; \&
	\node[down](c3){conv $1 \times 1$}; \\
	\& \node[op](m1){$\times$}; \\
	\node[op](m2){$\times$}; \&
	\node[fun](sm){$\softmax$}; \\
	\node[up](up){conv $1 \times 1$}; \\
	\node[outp](o){attention feature map}; \\
};
\coordinate(s) at($(i.south)!.5!(c2.north)$); % split
\draw[->]
	(i) edge (c2)
	(c2) edge node[dim,midway,right]{$c' \times hw$} node[key,midway,left]{$\vQ_s$} (m1)
	(m1) edge node[dim,midway,right]{$hw \times hw$} (sm)
	(up) edge node[dim,midway,right]{$c \times h \times w$} (o)
	(c1) edge node[dim,midway,right]{$c' \times hw$} node[key,midway,left]{$\vV_s$} (m2)
	(m2) edge node[dim,midway,right]{$c' \times h \times w$} (up)
	(sm) edge node[midway,above]{$\vA_s^g$} (m2)
	;
\draw[->] (s) -| (c1);
\draw[->] (s) -| node[dim,midway,above]{$c \times h \times w$} (c3);
\draw[->] (c3) |- node[dim,near start,right]{$c' \times hw$} node[key,near start,left]{$\vK_c$} (m1);
\node[left=1pt of i]{$\vF$};
\node[left=1pt of o]{$\vG_s$};
\end{tikzpicture}

\caption{Global spatial attention.}
\label{fig:fig5}
\end{figure}
%------------------------------------------------------------------------------

\paragraph{Global spatial attention}

Since ordinary convolution applies only a local neighborhood at a time, it cannot capture global contextual information. Thus, we apply \emph{non-local filtering}~\cite{Wang02}, which is a form of \emph{self-attention}~\cite{Vaswani01} in the spatial dimensions. As shown in \autoref{fig:fig5}, we are given the same $c\times h\times w$ feature tensor $\vF$ from our backbone. By using three $1\times 1$ convolutions, which reduce channels to $c'$, and flattening spatial dimensions to $hw$, we obtain $c' \times hw$ \emph{query} $\vQ_s$, \emph{key} $\vK_s$, and \emph{value} $\vV_s$ tensors, where each column is a feature vector corresponding to a particular spatial location. We capture pairwise similarities of these vectors by matrix multiplication of $\vK_s$ and $\vQ_s$, followed by softmax over locations to obtain a $hw \times hw$ \emph{global spatial attention map}:
\begin{equation}
	\vA_s^g = \softmax(\vK_s\tran \vQ_s).
\label{eq:eq4-1}
\end{equation}
This attention map is multiplied with $\vV_s$ and the matrix product $\vV_s \vA_s^g$ is reshaped back to $c' \times h \times w$ by expanding the spatial dimensions. Finally, using a ${1\times 1}$ convolution, which increases channels back to $c$, we obtain the $c \times h\times w$ \emph{global spatial attention feature map} $\vG_s$.

The right column of \autoref{fig:fig7} shows heat maps for global spatial attention, localizing target objects in images.

%------------------------------------------------------------------------------

\paragraph{Global attention feature map}

We use the global channel attention feature map $\vF_c$ to weigh $\vF$ element-wise
\begin{equation}
	\vF_c^g = \vF \odot \vG_c.
\label{eq:eq8}
\end{equation}
We then use global spatial attention feature map $\vG_s$ to weigh $\vF_c^g$ element-wise, resulting in the $c \times h \times w$ \emph{global attention feature map}
\begin{equation}
	\vF^g = \vF_c^g \odot \vG_s + \vF_c^g.
\label{eq:eq8-1}
\end{equation}
Similarly to $\mathbf{F}^{l}$ in~\eq{eq3} and~\eq{eq3-1}, we apply channel attention first, followed by spatial attention. However, unlike \eq{eq3}, there is no residual connection in~\eq{eq8}. This choice is supported by early experiments.

%------------------------------------------------------------------------------
\begin{figure}
\centering
\small
\setlength{\tabcolsep}{2pt}
\newcommand{\heat}[1]{%
	\fig[.28]{heatmap/#1/src.png} &
	\fig[.28]{heatmap/#1/l.png} &
	\fig[.28]{heatmap/#1/g.png} \\
}
\begin{tabular}{ccc}
	\heat{1}
	\heat{2}
	\heat{4}
	(a) input &
	(b) local &
	(c) global
\end{tabular}
\caption{\emph{Local and global spatial attention}. Left: input images. Middle: local spatial attention heat maps. Right: global spatial attention heat maps. Red (blue) means higher (lower) attention weight.}
\label{fig:fig7}
\end{figure}
%------------------------------------------------------------------------------

\subsection{Global-local attention}
\label{sec:embed}

\paragraph{Feature fusion}

As shown in \autoref{fig:glam}, we combine the local and global attention feature maps, $\vF^l$ and $\vF^g$, with the original feature $\vF$. While concatenation and summation are common operations for feature combination, we use a weighted average with weights $w_l$, $w_g$, $w$ respectively, obtained by softmax over three learnable scalar parameters, to obtain a $c \times h \times w$ \emph{global-local attention feature map}
\begin{equation}
	\vF^{gl} = w_l \vF^l + w_g \vF^l + w \vF.
\label{eq:eq10}
\end{equation}
EfficientDet~\cite{Tan01} has shown that this is the most effective, among a number of choices, for fusion of features across different scales.

%------------------------------------------------------------------------------

\paragraph{Pooling}

We apply GeM~\cite{Radenovic01}, a learnable spatial pooling mechanism, to feature map $\vF^{gl}$~\eq{eq10}, followed by a fully-connected (FC) layer with dropout and batch normalization. The final embedding is obtained by $\ell_2$-normalization.
