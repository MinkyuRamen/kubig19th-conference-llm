\section{Introduction}
\label{sec:intro}

%------------------------------------------------------------------------------
\begin{figure*}
\centering
\tikzfading[
	name=fade out,
	inner color=transparent!90,
	outer color=transparent!10,
]

%------------------------------------------------------------------------------
\begin{tikzpicture}[
	scale=.3,
	font={\footnotesize},
	node distance=.5,
	ovr/.style={fill=white,fill opacity=.9},
	ten/.style={draw,ovr},
	ops/.style={op,ovr},
	rec/.style args={(#1/#2)}{draw,rectc,minimum width=#1cm,minimum height=#2cm,preaction={ovr}},
	att/.style={fill,path fading=fade out},
	atth/.style={att,fading transform={yscale=10}},
	dim/.style={text opacity=.5,inner sep=2pt,below=2pt of #1},
	sym/.style={above=3pt of #1},
	symt/.style={above=8pt of #1-northwest},
	key/.style={red,left,near end},
	back/.style={draw=#1!60,fill=#1!30,fill opacity=.5,inner sep=4},
	xback/.style={back=#1,inner xsep=8},
	yback/.style={back=#1,inner ysep=8},
]
\matrix[
	tight,
	row sep={38,between origins},column sep=12,
	cells={scale=.3,},
	nodes={node distance=.5},
] {
	\&[4]\&
% 	local channel
	\node[rec=(4/1),atth] (lc) {};
	\node[sym=lc] {$\vA_c^l$};
	\node[dim=lc] {$c \times 1 \times 1$};
	\&
	\node[ops] (lc1) {$\times$};
	\&
	\node[ops] (lc2) {$+$};
	\&
	\para[ten]{lcf}{0,-2.5,-2}{1,5,4};
	\node[symt=lcf] {$\vF_c^l$};
	\&\&
% 	local spatial
	\node[rec=(4/4),att] (ls) {};
	\node[sym=ls] {$\vA_s^l$};
	\node[dim=ls] {$1 \times h \times w$};
	\node[ops,right=of ls] (ls1) {$\times$};
	\&
	\node[ops] (ls2) {$+$};
	\&
	\para[ten]{lsf}{0,-2.5,-2}{1,5,4};
	\node[symt=lsf] {$\vF^l$};
	\&[5]
% 	local output
	\node[ops] (l1) {$\times$};
	\\
% 	backbone
	\para[ten]{if}{0,-2.5,-2}{1,5,4};
	\node[ovr,dim=if-south] {$c \times h \times w$};
	\node[symt=if] {$\vF$};
	\&
	\node[dot](s) at(0,0){};
	\&\&\&\&\&
	\node[dot] (c){};
	\&\&\&\&
% 	skip
	\node[ops] (f1) {$\times$};
	\&
	\node[ops] (f) {$+$};
	\&
% 	glam output
	\para[ten]{of}{0,-2.5,-2}{1,5,4};
	\node[dim=of-south] {$c \times h \times w$};
	\node[symt=of] {$\vF^{gl}$};
	\\
	\&\&
% 	global channel
	\node[rec=(4/4),att] (gc) {};
	\node[sym=gc] {$\vA_c^g$};
	\node[dim=gc] {$c \times c$};
	\&
	\node[ops] (gc1) {$\times$};
	\&\&
	\para[ten]{gcf}{0,-2.5,-2}{1,5,4};
	\node[symt=gcf] {$\vF_c^g$};
	\&\&
% 	global spatial
	\node[rec=(5/5),att] (gs) {};
	\node[sym=gs] {$\vA_s^g$};
	\node[dim=gs] {$hw \times hw$};
	\node[ops,right=of gs] (gs1) {$\times$};
	\&
	\node[ops] (gs2) {$+$};
	\&
	\para[ten]{gsf}{0,-2.5,-2}{1,5,4};
	\node[symt=gsf] {$\vF^g$};
	\&
% 	global output
	\node[ops] (g1) {$\times$};
	\\
};

% weights
\node[above=of l1] (wl) {$w_l$};
\node[above=of f1] (w)  {$w$};
\node[above=of g1] (wg) {$w_g$};

% input / output
\draw (if-east)--(s);
\draw[->] (s) |- (lc);
\draw[->] (s) |- (gc);
\draw[->] (l1) -| (f);
\draw[->] (g1) -| (f);

% local stream
\draw[->]
	(lc) edge (lc1)
	(lc1) edge (lc2)
	(lc2) edge (lcf-west)
	(ls) edge (ls1)
	(ls1) edge (ls2)
	(ls2) edge (lsf-west)
	(lsf-east)--(l1)
	;

% skip stream
\draw[->]
	(s)--(f1)
	(f1) edge (f)
	(f) edge (of-west)
	;

% center connections
\draw[->] (c) |- (ls);
\draw[->] (c) |- (gs);

% global stream
\draw[->]
	(gc) edge (gc1)
	(gc1) edge (gcf-west)
	(gs) edge (gs1)
	(gs1) edge (gs2)
	(gs2) edge (gsf-west)
	(gsf-east)--(g1)
	;

% local skips
\draw (lcf-north) -- +(0,.6) coordinate (lcf-n);
\draw[->] (lcf-n) -| (ls1);
\draw[->] (lcf-n) -| (ls2);

% global skips
\draw (gcf-south) -- +(0,-.6) coordinate (gcf-s);
\draw[->] (gcf-s) -| (gs1);
\draw[->] (gcf-s) -| (gs2);

% input skips
\path
	(s |- lcf-n) coordinate(s-n)
	(s |- gcf-s) coordinate(s-s)
	;
\draw[->] (s-n) -| (lc1);
\draw[->] (if-north) |- (s-n) -| (lc2);
\begin{pgfonlayer}{bg1}
	\draw[->] (if-south) |- (s-s) -| (gc1);
\end{pgfonlayer}

% weighting
\draw[->]
	(wl) edge (l1)
	(w) edge (f1)
	(wg) edge (g1)
	;

\begin{pgfonlayer}{bg2}
	\node[yback=blue,fit=(s-n) (s-s) (lc2)] (channel) {};
	\node[yback=red,fit=(lcf-n -| ls1) (gcf-s -| gs1) (c) (gs2)] (spatial) {};
	\node[back=black,fit=(l1) (g1) (f)] (fusion) {};
	\node[xback=yellow,fit=(s-n) (lcf-n) (lcf-south) (lsf-east)] (local) {};
	\node[xback=green,fit=(s-s) (gcf-north) (gcf-s) (gsf-east)] (global) {};
\end{pgfonlayer}

\node[blue,below=1pt of channel]{channel attention};
\node[red,below=1pt of spatial]{spatial attention};
\node[black,below=1pt of fusion]{fusion};
\node[yellow!60!red,above=1pt of local.north east]{local attention};
\node[green!60!black,below=1pt of global.south east]{global attention};

\end{tikzpicture}
%------------------------------------------------------------------------------

\caption{Our \emph{global-local attention module} (GLAM) involves both {\color{blue}channel} and {\color{red}spatial} attention, as well as both {\color{yellow!60!red}local} attention (channels/locations weighted independently, based on contextual information obtained by pooling) and {\color{green!60!black}global} attention (based on pairwise interaction between channels/locations). As a result, four attention maps are used: \emph{local channel} ($\vA_c^l$), \emph{local spatial} ($\vA_s^l$), \emph{global channel} ($\vA_c^g$) and \emph{global spatial} ($\vA_s^g$). The input feature map $\vF$ is weighted into local ($\vF^l$) and global ($\vF^g$) attention feature maps, which are fused with $\vF$ to yield the \emph{global-local attention feature map} $\vF^{gl}$. The diagram is abstract: The four attention modules are shown in more detail in Figures \ref{fig:fig4}, \ref{fig:fig3}, \ref{fig:fig6}, \ref{fig:fig5}.}
\label{fig:glam}
\end{figure*}
%------------------------------------------------------------------------------

Instance-level image retrieval is at the core of visual representation learning and is connected with many problems of visual recognition and machine learning, for instance \emph{metric learning}~\cite{oh2016deep,KKCK20}, \emph{few-shot learning}~\cite{SnellSZ17} and \emph{unsupervised learning}~\cite{chen2020simple}. Many large-scale open datasets~\cite{Babenko01, Radenovic01, Gordo01, Noh01, Weyand01}, and competitions\footnote{https://www.kaggle.com/c/landmark-retrieval-2020} have accelerated progress in instance-level image retrieval, which has been transformed by deep learning~\cite{Babenko01}.

Many studies on instance-level image retrieval focus on learning features from \emph{convolutional neural networks} (CNN), while others focus on \emph{re-ranking}, for instance by graph-based methods~\cite{Donoser01}. The former can be distinguished according to feature types: \emph{local descriptors}, reminiscent of SIFT~\cite{Lowe01}, where an image is mapped to a few hundred vectors; and \emph{global descriptors}, where an image is mapped to a single vector. In fact, deep learning has brought global descriptors with astounding performance, while allowing efficient search. Our study belongs to this type.

Studies on global descriptors have focused on \emph{spatial pooling}~\cite{Babenko03,Radenovic01}. The need for compact, discriminative representations that are resistant to clutter has naturally given rise to \emph{spatial attention} methods~\cite{Kalantidis01,Ng01}. Different kinds of attention have been studied in many areas of computer vision research. There is also \emph{channel attention}~\cite{Hu01,ChenKLYF18}; \emph{local attention}, applied independently to elements of the representation (feature map)~\cite{woo01,Kim01}; \emph{global attention}, based on interaction between elements~\cite{Wang02,ChenKLYF18}; and combinations thereof. Unfortunately, each study has been limited to one or two kinds of attention only; attention is not always learned; and applications vary.

It is the objective of our work to perform a comprehensive study of all forms of attention above, apply them to instance-level image retrieval and provide a detailed account of their interaction and impact on performance. As shown in \autoref{fig:glam}, we collect contextual information from images with both \emph{local} and \emph{global} attention, giving rise to two parallel network streams. Importantly, each operates on both \emph{spatial locations} and \emph{feature channels}. Local attention is about individual locations and channels; global is about interaction between locations and between channels. The extracted information is separately embedded in local and global attention feature maps, which are combined in a \emph{global-local attention feature map} before pooling.

Our contributions can be summarized as follows:
\begin{enumerate}[itemsep=2pt, parsep=0pt, topsep=0pt]
	 \item We propose a novel network that consists of both global and local attention for image retrieval. This is the first study that employs both mechanisms.
	 \item Each of the global and local attention mechanisms comprises both spatial and channel attention.
	 \item Focusing on global descriptors, we provide empirical evidence of the interaction of all forms of attention and improve the state of the art on standard benchmarks.
\end{enumerate}
