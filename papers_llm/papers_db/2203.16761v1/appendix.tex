\vspace{-0.5mm}
\section{Appendix}

\subsection{Algorithm}
\vspace{-2.5mm}
\begin{algorithm}
    \small
    \caption{MeMOT Algorithm}
    \hspace*{\algorithmicindent} \textbf{Input}: A sequence of video frames $\mathbi{I} = \{I^0, I^1, \cdots, I^T\}$.\\
    \hspace*{\algorithmicindent} \textbf{Memory}: A set of track states $\mathbi{X} = \{X_0, X_1, \cdots, X_K\}$ with $X_k = \{\widehat{\mathbi{q}}_k^{t_0}, \widehat{\mathbi{q}}_k^{t_1}, \cdots, \widehat{\mathbi{q}}_k^{t_N}\}$ of identity embeddings $\widehat{\mathbi{q}}_k^t \in \mathbb{R}^d$.\\
    % $\widehat{\mathbi{q}}_k^t \leftarrow \mathbi{0}$ if object $k$ does not appear in $I^t$. \\
    \hspace*{\algorithmicindent} \textbf{Output}: A set of trajectories $\mathbfcal{T} = \{\mathcal{T}_0, \mathcal{T}_1, \cdots, \mathcal{T}_K\}$ with $\mathcal{T}_k = \{\mathbi{b}_k^{t_0}, \mathbi{b}_k^{t_1}, \cdots, \mathbi{b}_k^{t_N}\}$ of bounding boxes $\mathbi{b}_k^{t} = (x, y, w, h)$.
    \begin{algorithmic}[1]
    \State $\mathbi{X} \leftarrow \O$
    \For{$I^t \in \mathbi{I}$}
        \State $\mathbfcal{T}_{new}, \mathbi{X}_{new} \leftarrow \O, \O$
        
        % hypothesis generation
        \State $z_1^t \leftarrow \Theta_H$.encoder($I^t$)
        \State $\mathbi{Q}_{pro}^t \leftarrow \Theta_H$.decoder($z_1^t$)
        
        % encoding
        \State $\mathbi{Q}^t_{AST} \leftarrow \Theta_E$.$f_{short}(\mathbi{X}^{t-1}, \mathbi{X}^{t-1-T_s:t-1})$
        \State $\mathbi{Q}^t_{ALT} \leftarrow \Theta_E$.$f_{long}(\mathbi{Q}_{dmat}^{t-1}, \mathbi{X}^{t-1-T_l:t-1})$
        \State $[\mathbi{Q}_{tck}^t, \mathbi{Q}_{dmat}^{t}] \leftarrow \Theta_E$.$f_{fuse}([\mathbi{Q}^t_{AST}, \mathbi{Q}^t_{ALT}])$
        
        % decoding
        \State $[\widehat{\mathbi{Q}}_{pro}^t, \widehat{\mathbi{Q}}_{tck}^t] \leftarrow \Theta_D$.association\_solver$([\mathbi{Q}_{pro}^t, \mathbi{Q}_{tck}^{t}], z_1^t)$
        \State $[\mathbi{B}^t_{pro}, \mathbi{B}^t_{tck}], [\mathbi{S}^t_{pro}, \mathbi{S}^t_{tck}] \leftarrow \Theta_D$.predictor$([\widehat{\mathbi{Q}}_{pro}^t, \widehat{\mathbi{Q}}_{tck}^t])$
        
        % update
        \For{$\mathbi{b}^t_{k} \in \mathbi{B}^t_{tck}$, $\mathbi{s}^t_{k} \in \mathbi{S}^t_{tck}$, $\widehat{\mathbi{q}}^t_{k} \in \widehat{\mathbi{Q}}^t_{tck}$}
            \If{$\mathbi{s}^t_{k} \ge$ $\epsilon_{tck}$}
                \State $\mathcal{T}_k \leftarrow \mathcal{T}_k + \{\mathbi{b}^t_{k}\}$
                \State $X_k \leftarrow X_k + \{\widehat{\mathbi{q}}^t_{k}\}$
            \EndIf
        \EndFor
        \For{$\mathbi{b}^t_{k} \in \mathbi{B}^t_{pro}$, $\mathbi{s}^t_{k} \in \mathbi{S}^t_{pro}$, $\widehat{\mathbi{q}}^t_{k} \in \widehat{\mathbi{Q}}^t_{pro}$}
            \If{$\mathbi{s}^t_{k} \ge$ $\epsilon_{pro}$}
                \State $\mathbfcal{T}_{new} \leftarrow \mathbfcal{T}_{new} + \{\{\mathbi{b}^t_{k}\}\}$
                \State $\mathbi{X}_{new} \leftarrow \mathbi{X}_{new} + \{\{\widehat{\mathbi{q}}^t_{k}\}\}$
            \EndIf
        \EndFor
        \State $\mathbfcal{T} \leftarrow \mathbfcal{T} + \mathbfcal{T}_{new}$
        \State $\mathbi{X} \leftarrow \mathbi{X} + \mathbi{X}_{new}$
    \EndFor
    \end{algorithmic}
    \label{alg:memot}
\end{algorithm}

% \begin{algorithm}
%     \small
%     \caption{Workflow of MeMOT}
%     \hspace*{\algorithmicindent} \textbf{Input}: input video frame $I^t$, external spatio-temporal memory $\mathbi{X}^{t-1-T:t-1} \in \mathbb{R}^{N^t_{tck} \times T \times d}$, and DMAT $\mathbi{Q}_{dmat}^{t-1} \in \mathbb{R}^{N^t_{tck} \times d}$ \\
%     \hspace*{\algorithmicindent} \textbf{Output}: bounding boxes $[\mathbi{B}^t_{pro}, \mathbi{B}^t_{tck}]$ and confidence scores $[\mathbi{S}^t_{pro}, \mathbi{S}^t_{tck}]$ for new and tracked objects
%     \begin{algorithmic}[1]
%         \FUNCDO{$\Theta_H$ generates proposal embeddings}
%             \State Extract image feature $z_0^t$ using $I^t$
%             \State Encode image feature $z_1^t$ using $z_0^t$
%             \State Decode proposal embeddings $\mathbi{Q}_{pro}^t \in \mathbb{R}^{N^t_{pro} \times d}$
%         \ENDFUNCDO
        
%         \FUNCDO{$\Theta_E$ aggregates embeddings for each track}
%             \State Compute AST $q^t_{AST}$ using $\mathbi{X}^{t-1-T_s:t-1}$ and $\mathbi{X}^{t-1}$
%             \State Compute ALT $q^t_{ALT}$ using $\mathbi{X}^{t-1-T_l:t-1}$ and $\mathbi{Q}_{dmat}^{t-1}$
%             \State Compute track embeddings $\mathbi{Q}_{tck}^t$ and new DMAT $\mathbi{Q}_{dmat}^{t}$ using $f_{fuse}([q^t_{AST}, q^t_{ALT}])$ as in Eq.~\ref{eq:memory}
%             \State Update DMAT with $\mathbi{Q}_{dmat}^{t}$ for next step $t+1$
%         \ENDFUNCDO
        
%         \FUNCDO{$\Theta_D$ updates tracked objects and initiates new objects}
%             \State Compute $[\widehat{\mathbi{Q}}_{pro}^t, \widehat{\mathbi{Q}}_{tck}^t]$ using $[\mathbi{Q}_{pro}^t, \mathbi{Q}_{tck}^t]$ and $z_1^t$
%             \State Compute bounding boxes $[\mathbi{B}^t_{pro}, \mathbi{B}^t_{tck}]$ and confidence scores $[\mathbi{S}^t_{pro}, \mathbi{S}^t_{tck}]$ for $[\widehat{\mathbi{Q}}_{pro}^t, \widehat{\mathbi{Q}}_{tck}^t]$ using MLPs
%             \State Update $\mathbi{X}^{t-T:t}$ with $\widehat{\mathbi{Q}}_{pro}^t$ for $t+1$
%             \State Update $\mathbi{X}^{t-T:t}$ with $\mathbi{X}^{t-1-T:t-1}$ and $\widehat{\mathbi{Q}}_{tck}^t$ for $t+1$
%         \ENDFUNCDO
        
%     \end{algorithmic}
%     \label{alg:memot}
% \end{algorithm}
The workflow of our proposed MeMOT is shown in Algorithm~\ref{alg:memot}. 
MeMOT takes a sequence of video frames $\mathbi{I} = \{I^0, I^1, \cdots, I^T\}$ as input, and outputs trajectories $\mathbfcal{T} = \{\mathcal{T}_0, \mathcal{T}_1, \cdots, \mathcal{T}_K\}$ for $K$ objects. The track states $\mathbi{X} = \{X_0, X_1, \cdots, X_K\}$, represented as embeddings for each object at its own active timestamps, are maintained and updated in a spatio-temporal memory buffer. 
MeMOT contains three Transformer-based network modules: 1) a hypothesis generation module $\Theta_H$ for extracting the frame feature $z_1$ and producing the proposal embeddings $\mathbi{q}_{pro}$, 2) a memory encoding module $\Theta_E$ that aggregates the previous states to track embeddings $\mathbi{q}_{tck}$ for each object, and 3) a memory decoding module $\Theta_D$ that predicts the current states of tracked objects and initializes new objects. 

Concretely, at time $t$, the encoder of $\Theta_H$ translates image $I^t$ to features $z_1^t \in \mathbb{R}^{d \times HW}$, which are then decoded to a set of proposal embeddings $\mathbi{Q}_{pro}^t$ by $\Theta_H$'s decoder.
At the same time, the short-term aggregation module $f_{short}$ in $\Theta_E$ queries the past $T_s$ memory $\mathbi{X}^{t-1-T_s:t-1}$ with the latest observation $\mathbi{X}^{t-1}$ and obtains the aggregated short-term queries $\mathbi{Q}^t_{AST}$.
The long-term aggregation module $f_{long}$ uses a set of learnable queries, called dynamic memory aggregation token (DMAT) $\mathbi{Q}_{dmat}^{t-1}$, and takes advantages of a longer time period $T_l$ to produce the aggregated long-term queries $\mathbi{Q}^t_{ALT}$. $\mathbi{Q}^t_{AST}$ and $\mathbi{Q}^t_{ALT}$ are fused by a self-attention module $f_{fuse}$, which outputs the track query $\mathbi{Q}_{tck}^t$ and updated $\mathbi{Q}_{dmat}^{t}$. 
$\Theta_D$ takes the concatenated set of $\mathbi{Q}_{pro}^t$ and $\mathbi{Q}_{tck}^t$ as query set and the frame feature $z_1^t$ as key-value, generating the estimated states $\widehat{\mathbi{Q}}_{pro}^t$ and $\widehat{\mathbi{Q}}_{tck}^t$. 
Object bounding boxes $\textbf{B}_{pro}^t$, $\textbf{B}_{tck}^t$ and confidence scores $\textbf{S}_{pro}^t$, $\textbf{S}_{tck}^t$ are obtained from $\widehat{\textbf{Q}}_{pro}^t$ and $\widehat{\textbf{Q}}_{tck}^t$ through FFN. 
For all the tracked objects, the states will be updated if their confidence scores are above a threshold $\epsilon_{tck}$. Similarly, proposal queries will be initialized as new tracks if the confidence scores are higher than $\epsilon_{pro}$. 
As discussed in the paper, $T_s$ is selected as 3 as an accuracy-efficiency trade-off; while $T_l$ is 24 frames due to hardware limitation. We select $\epsilon_{pro}$, $\epsilon_{tck}$ as 0.7 and 0.6, respectively.
